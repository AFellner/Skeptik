\documentclass[a4paper]{article}

\usepackage{kpfonts}
\usepackage[left=2.8cm, right=4cm, bottom=2.5cm]{geometry}

\usepackage{amsthm}
\theoremstyle{definition}
\newtheorem{thm}{Theorem}
\newtheorem{defn}{Definition}
\newtheorem*{nota}{Notation}

\usepackage{bussproofs}
\def\defaultHypSeparation{\hskip.1in}

\begin{document}

\section{Introduction}

We would like to extend the unit concept for lowering more nodes. Intuitively,
if a node with a clause $a$ is already marked to be lowered, another node with
a clause $\bar{a},b$ might be lowered too.

\begin{nota}[Resolution]
Resolution of nodes $\eta_0$ and $\eta_1$ is written $\eta_0 \odot_a
\eta_1$, $a$ being the pivot literal such that $\bar{a} \in \eta_0$ and $a
\in \eta_1$.
\end{nota}

\section{Lowering node}

The problem of lowering a node $\eta$ in a proof of the form $\psi[\eta]
\odot_{a_0} \eta_0 \odot_{a_1} \cdots \odot_{a_n} \eta_n$ is to decide
whether a literal $a$ such that the proof $\psi[] \odot_a \eta
\odot_{a_0} \eta_0 \odot_{a_1} \cdots \odot_{a_n} \eta_n$ is equivalent to
the original proof exists.

This problem doesn't depend on the sequence $\eta_{i \in 0..n}$ but only on
$\psi[\eta]$ and $a_{i \in 0..n}$. This last sequence might be seen as a
clause. We call it the clause of lowered pivots.

\section{Univalent clause}

The extended unit concept intuitively presented above can be formely defined as
univalent clause.

\begin{defn}[Valent literal]
A literal $a$ is said to be valent w.r.t. a clause $\Delta$ iff $\bar{a}$ does
not occur in $\Delta$.
\end{defn}

\begin{defn}[Univalent clause]
A clause $\Gamma$ is said to be univalent w.r.t. another clause $\Delta$ iff
there is exactely one literal $a$ in $\Gamma$ which is valent w.r.t. $\Delta$.
\end{defn}

Units have univalent clause w.r.t. a clause $\Delta$ unless their sole literal
is a dual of a $\Delta$'s literal. Therefore univalent clauses are not a
strict extension of units.


\section{Lowering nodes with univalent clause}

The question now is whether nodes with a univalent clause w.r.t. the clause of
lowered pivots can be lowered.

Actually this isn't always the case. For example let's consider a node
$\eta$ with clause $a,b$, the clause $\Delta$ of lowered pivots being
$\bar{a}$. Obviously, $\eta$'s clause is univalent. But if an $\eta$'s
direct child has $a$ as pivot, removing $\eta$ from $\psi$ will introduce
the literal $\bar{a}$ which will propagate down the proof.

Hence we need a condition on the children of the node we want to lower.

\begin{defn}[Active literal]
Let's consider a node $\eta$ with clause $\Gamma$. A literal $a$ from
$\Gamma$ is said to be an active literal of $\eta$ iff $a$ is the pivot of
one of $\eta$'s child.
\end{defn}

\begin{thm}
If all the active literals of a node $\eta$ with clause $\Gamma$ univalent
w.r.t. the clause $\Delta$ of lowered literals either are the valent
literal $a$ or belong to $\Delta$, then $\eta$ can be lowered.
\end{thm}

\begin{proof}
Let $\bar{\Delta}$ be the clause of all $\Delta$'s literals duals.
$\psi[\eta]$'s clause subsumes $\bar{\Delta}$. For $\eta$ to be lowerable,
the conclusion of $\psi[] \odot_a \eta$ has to subsume $\bar{\Delta}$ too.

By removing $\eta$ from $\psi$ only the dual of every $\eta$'s active literal
might be introduced. If the active literal is $a$, its dual will be removed
by the resolution on $a$. If the active literal belongs to $\Delta$, its
dual belongs to $\bar{\Delta}$.
\end{proof}

\end{document}
