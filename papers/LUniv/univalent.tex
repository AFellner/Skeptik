\documentclass{llncs}

\spnewtheorem*{notation}{Notation}{\bfseries}{\itshape}

\usepackage{bussproofs}
\def\defaultHypSeparation{\hskip.1in}

\usepackage{enumitem,amsmath}

% The following is to comment draft source file. It has to be deleted for the
% final document.
\usepackage{xcolor}
\newenvironment{jogo}{\color{teal}}{}

\title{Propositional Resolution Proof Compression by Lowering Nodes}
\author{Joseph Boudou \and Bruno Woltzenlogel Paleo}
\institute{Vienna University of Technology, Austria}

\begin{document}

\maketitle

\begin{abstract}
\begin{jogo}
  To do.
\end{jogo}
\end{abstract}

\section{Introduction}

\begin{jogo}
In this section we introduce the domain : first SAT then compression.
I could be a good idea to talk about the problem of measuring a proof.
We could explain the use case of verifying a proof and how we plan to measure its complexity.
\end{jogo}

\section{Propositional Resolution Calculus}

\begin{jogo}
In this section we introduce the definitions and notations we'll use trough this paper.

Here is a list of things to define (maybe elsewhere) : \begin{itemize}[nosep]
    \item variable, dual, literal, clause, node, proof
    \item premise, children, parent
    \item regularity (vertical and horizontal)
    \item safe literals
\end{itemize}
\end{jogo}

\begin{notation}[Resolution]
Resolution of nodes $\eta_0$ and $\eta_1$ is written $\eta_0 \odot_a \eta_1$, $a$ being the pivot
literal such that $\bar{a} \in \eta_0$ and $a \in \eta_1$.
\end{notation}

\section{Lowering Nodes}

\begin{jogo}
In this section we define formely what we mean by lowering a node. Then we explain why it's
interessing for proof compression. Finaly we briefly describe LowerUnits.
\end{jogo}

When a node $\eta$ as more than one child in a proof $\psi[\eta]$ it might be convenient to factor
the corresponding resolutions. Lowering $\eta$ is such a factorisation. We construct a new
equivalent proof by removing $\eta$ from $\psi$, fixing $\psi[]$ such that it remains a valid proof
and then reintroducing $\eta$ at the bottom of the proof. Formaly, $\eta$ can be lowered in
$\psi[\eta]$ if there exists a literal $a$ such that $\operatorname{fix}\left(\psi[]\right) \odot_a
\eta$ is equivalent to $\psi[\eta]$. A possible implementation of the $\operatorname{fix}$ function
is proposed as the \ref{algo:fix}.

\begin{algorithm}[hbt]
  \For{every resolution node $\eta$ with in a top-down traversal}{
    Let $p$ be $\eta$'s pivot \;
    \uIf{the left premise of $\eta$ is deleted or its conclusion doesn't contain $\dual{p}$}{
      Replace $\eta$ by its right premise \; }
    \uElseIf{the right premise of $\eta$ is deleted or its conclusion doesn't contain $p$}{
      Replace $\eta$ by its left premise \; }
    \Else{
      Recompute $\eta$'s clause based on the current premises \; }
  }
  \label{algo:fix}
  \caption{A possible \operatorname{fix} implementation}
\end{algorithm}

\begin{jogo}
This introduction to lowering is not the same as the one of LU's paper. The drawback is that we need
to cope with the fix function which is not nice. But the advantage is that it's simpler and not
limited to the 2 children case. An alternative solution would be to use multi-arguments environments
and define lowering as transforming $\psi\left[\eta_0 \odot \eta, \cdots, \eta_n \odot \eta\right]$
into $\psi\left[\eta_0, \cdots, \eta_n\right] \odot \eta$. But anyway we'll need a fix function to
introduce LowerUnivalent.
\end{jogo}

It's obvious that lowering nodes with more than one child leads to an smaller equivalent proof.
These idea has been introduce in \cite{LURPI} for the LowerUnits algorithm. Units are nodes with a
clause consisting of only one literal. Such nodes can always be lowered. The proposed algorithm
lowers every unit with more than one child. Care is taken to reintroduce units in an order
corresponding to the transitive closure of the premise relation : if a unit $\eta$ is a parent of a
unit $\varsigma$ then $\eta$ has to be reintroduced after (ie below) $\varsigma$.

\begin{algorithm}[htb]
  Units $\leftarrow \varnothing$ \;

  \For{every node $\eta$ in a bottom-up traversal}{
    \If{$\eta$ is a unit}{Enqueue $\eta$ in Units \; }
  }

  \For{every node $\eta$ in a top-down traversal}{
    \uIf{the right premise of $\eta$ belongs to Units}{
      Replace $\eta$ by its left premise \; }
    \uElseIf{the left premise of $\eta$ belongs to Units}{
      Replace $\eta$ by its left premise \; }
    \Else{
      Recompute $\eta$'s clause based on the current premises \; }
  }

  $\eta \leftarrow$ the new root of the proof \;
  \For{every unit $\upsilon$ in Units}{
    $\eta \leftarrow \eta \odot \upsilon$ \;
  }

  \label{algo:LU}
  \caption{A possible LowerUnit implementation}
\end{algorithm}

A possible implementation of LowerUnits is shown in \ref{algo:LU}. Units are collected during a
first traversal. As this traversal is bottom up, units are stored in a queue. The traversal could
have been top-down and units stored in a stack. Units are effectively removed during a second,
top-down traversal. The last step is the reintroduction of units. There is no need here for such a
complicated fix implementation as \ref{algo:fix}, because deleting units only introduce literals in
the children's clauses and therefore all the resolutions are still valid.


\section{LoweringUnivalents}

\begin{jogo}
This is the main section. First we define univalent clause and we prove that it's not sufficent for
a node to be lowered. Second we introduce active literals and the main theorem (with proof). Third
we explain the partial regularization done by "Deletable" nodes. Fourth we give the algorithm. Fifth
we prove it's always better than LowerUnits.
\end{jogo}

We would like to extend LowerUnits presented in \cite{LURPI}. Intuitively,
if a node with a clause $a$ is already marked to be lowered, another node with
a clause $\bar{a},b$ might be lowered too.

The problem of lowering a node $\eta$ in a proof of the form
\begin{equation}
\psi[\eta] \odot_{a_0} \eta_0 \odot_{a_1} \cdots \odot_{a_n} \eta_n
\end{equation}
is to decide whether a literal $a$ such that the proof 
\begin{equation}
\text{Fix}\left(\psi[]\right) \odot_a \eta \odot_{a_0} \eta_0 \odot_{a_1} \cdots \odot_{a_n} \eta_n
\end{equation}
is equivalent to
the original proof exists.
\subsection{Univalent Clauses}

The extended unit concept intuitively presented above can be formely defined as
univalent clause.

\begin{definition}[Valent literal]
A literal $a$ is said to be valent w.r.t. a clause $\Delta$ iff $\bar{a}$ does
not occur in $\Delta$.
\end{definition}

\begin{definition}[Univalent clause]
A clause $\Gamma$ is said to be univalent w.r.t. another clause $\Delta$ iff
there is exactly one literal $a$ in $\Gamma$ which is valent w.r.t. $\Delta$.
\end{definition}

Units have univalent clause w.r.t. a clause $\Delta$ unless their sole literal
is a dual of a $\Delta$'s literal. Therefore univalent clauses are not a
strict extension of units.


\subsection{Lowering Nodes with Univalent Clause}

The question now is whether nodes with a univalent clause w.r.t. the clause of
lowered pivots can be lowered.

Actually this isn't always the case. For example let's consider a node
$\eta$ with clause $a,b$, the clause $\Delta$ of lowered pivots being
$\bar{a}$. Obviously, $\eta$'s clause is univalent. But if an $\eta$'s
direct child has $a$ as pivot, removing $\eta$ from $\psi$ will introduce
the literal $\bar{a}$ which will propagate down the proof.

Hence we need a condition on the children of the node we want to lower.

\begin{definition}[Active literal]
Let's consider a node $\eta$ with clause $\Gamma$. A literal $a$ from
$\Gamma$ is said to be an active literal of $\eta$ iff $a$ is the pivot of
one of $\eta$'s child.
\end{definition}

\begin{theorem}
If all the active literals of a node $\eta$ with clause $\Gamma$ univalent
w.r.t. the clause $\Delta$ of lowered literals either are the valent
literal $a$ or belong to $\Delta$, then $\eta$ can be lowered.
\end{theorem}

\begin{proof}
Let $\bar{\Delta}$ be the clause of all $\Delta$'s literals duals.
$\psi[\eta]$'s clause subsumes $\bar{\Delta}$. For $\eta$ to be lowerable,
the conclusion of $\psi[] \odot_a \eta$ has to subsume $\bar{\Delta}$ too.

By removing $\eta$ from $\psi$ only the dual of every $\eta$'s active literal
might be introduced. If the active literal is $a$, its dual will be removed
by the resolution on $a$. If the active literal belongs to $\Delta$, its
dual belongs to $\bar{\Delta}$.
\qed
\end{proof}

\section{Experiments}

\begin{jogo}
LU vs LUniv ; RPI[3]LU vs RPI[3]LUniv ; LUnivRPI vs LU.RPI.
\end{jogo}

\section{Conclusions}

\bibliographystyle{splncs}
\bibliography{../biblio}

\end{document}
