\documentclass{easychair}

\usepackage{xspace}

\usepackage{prooftheory}

\title{Compression of Propositional Resolution Proofs by Lowering Subproofs}
\titlerunning{Compression of Propositional Resolution Proofs}

\author{
  Joseph Boudou\inst{1}
  \thanks{Supported by the Google Summer of Code 2012 program.}
  \and 
  Bruno Woltzenlogel Paleo\inst{2}
  \thanks{Supported by the Austrian Science Fund, project P24300.}
}

\authorrunning{J.\~Boudou \and B.\~Woltzenlogel Paleo}

\institute{
  Universit\'e Paul Sabatier, Toulouse \\
  \email{joseph.boudou@matabio.net}
  \and 
  Vienna University of Technology \\
  \email{bruno@logic.at}
}

\begin{document}

\maketitle

Sat-solvers are among the most efficient automated deduction tools available today.
%As standalone tools, they can already be applied to a wide range of problems.
%, especially considering that, due to
%the NP-completeness of propositional satisfiability \cite{cook}, any NP problem can be encoded as a
%propositional formula.
%Despite the theoretical difficulty associated with NP
%problems, state-of-the-art sat-solving techniques perform surprisingly well in practice.
Leveraging this efficiency, sat-solvers have been embedded
into various other automated deduction tools
%that target problems described in more expressivelogics.
%The most prominent examples are SMT-solvers.
%, for checking satisfiability modulo
%theories for equality, linear arithmetic, bit-vectors, arrays and others.
%More recently, 
like interactive proof assistants and automated first-order
and even higher-order theorem provers.
%have taken advantage of embedding sat-solvers too.
In such a scenario, proofs of unsastifiability produced by the sat-solver are analysed by the frontend tools. The quality and
size of this proofs has thus become of critical importance.

%it is essential that sat-solvers output not only a \emph{yes} or \emph{no}
%answer, but also a model (in case of satisfiability) or a refutation (in case of unsatisfiability).
%For DPLL- and CDCL-based sat-solvers, propositional resolution is an excellent proof system, since
%refutations in this system can be generated with an acceptable efficiency overhead and they are
%detailed enough to allow easy implementation of efficient proof checkers.


%With the increase in the demand for proofs from sat-solvers, there has been a surge of techniques to
%compress such proofs in a post-processing stage.
Techniques to compress propositional resolution proofs in a post-procecession stage have been
proposed.
\RecyclePivotsIntersection and \LowerUnits \cite{LURPI} are such proofs
compression algorithms. They both achieve good compression ratio in linear time w.r.t. the number of
resolution steps. The former reduces proofs' irregularity as described by Tseitin \cite{Tseitin}. The
latter extends the concept of irregularity to redundances between branches and reduces
this \emph{horizontal irregularity} by lowering subproofs of a single literal (called units).
Experiments showed \cite{LURPI} that sequentialy composing this two algorithms leverages their
efficiency.

In this talk, we'll describe a new proof compression algorithm, called \LowerUnivalents~\footnote{An article describing
this algorithm has just been submitted to the CADE 23 conference.}, which extends \LowerUnits. Its
purposes are twofold. First, it compress further than \LowerUnits by lowering
more subproofs. Second, a non-sequential combination of \LowerUnivalents after
\RecyclePivotsIntersection may be easily implemented. This combination is both faster and more
efficient than the sequential composition of \LowerUnits after \RecyclePivotsIntersection.

The principle of \LowerUnivalents is to take already lowered subproofs into account in order to
allow more proofs to be lowered. For instance, if a subproof of $p$ has already been selected for
lowering, a subproof of $\neg p, q$ may be lowered above the previous subproofs, provided its
lowering would not introduce unwanted literals in the conclusion. The formal properties for such
lowering will be exposed along with practical optimizations which make the algorithm reduce both
traditional and horizontal irregularities.


\bibliographystyle{plain}
\bibliography{../biblio}

\end{document}

% vim: tw=100
