\documentclass[compress]{beamer}
\setbeamercovered{transparent}
%\useoutertheme{infolines}
%\useoutertheme[footline=authortitle,subsection=false]{miniframes}
\setbeamertemplate{navigation symbols}{}

\definecolor{vertmoyen}{HTML}{A9CB60}
\definecolor{vertclair}{HTML}{D3F192}
\definecolor{vertfonce}{HTML}{347003}
\definecolor{deletecol}{HTML}{3B8A13}
\definecolor{addcolor} {HTML}{9E0270}

\usecolortheme[named=vertfonce]{structure}

\setbeamercolor{section in head/foot}{fg=vertfonce,bg=vertmoyen}
\setbeamercolor{subsection in head/foot}{bg=vertclair}

\setbeamercolor{author in head/foot}{fg=vertfonce, bg=vertclair}
\setbeamercolor{title in head/foot}{fg=vertfonce, bg=vertmoyen}
\setbeamercolor{date in head/foot}{fg=vertfonce, bg=vertclair}

\setbeamercolor{separation line}{bg=vertfonce}


\setbeamercolor{titlelike}{bg=vertclair}
\setbeamerfont{frametitle}{size=\normalsize}

\makeatletter

\defbeamertemplate*{frametitle}{jogo theme}
{
  \ifbeamercolorempty[bg]{frametitle}{}{\nointerlineskip}%
  \ifx\insertsectionhead\@empty\usebeamerfont{headline}\vspace{-3.45ex}\fi%
  \@tempdima=\textwidth%
  \advance\@tempdima by\beamer@leftmargin%
  \advance\@tempdima by\beamer@rightmargin%
  \begin{beamercolorbox}[left,wd=\the\@tempdima]{frametitle}
    \usebeamerfont{frametitle}%
%    \vbox{}\vskip-0.5ex%
%    \if@tempswa\else\csname beamer@fte#1\endcsname\fi%
    \strut\hspace{1em}\insertframetitle\strut\par%
    {%
      \ifx\insertframesubtitle\@empty%
      \else%
      {\usebeamerfont{framesubtitle}\usebeamercolor[fg]{framesubtitle}\insertframesubtitle\strut\par}%
      \fi
    }%
%    \vskip-1ex%
%    \if@tempswa\else\vskip-.3cm\fi% set inside beamercolorbox... evil here...
  \end{beamercolorbox}%
}

\defbeamertemplate*{headline}{jogo theme}
{
    \leavevmode%
  \ifx\insertsectionhead\@empty\vspace{3.45ex}%
  \else%
  {%
    \hbox{%
    \begin{beamercolorbox}[wd=.5\paperwidth,ht=2.25ex,dp=1ex,right]{section in head/foot}%
      \usebeamerfont{section in head/foot}\insertsectionhead\hspace*{2ex}
    \end{beamercolorbox}%
    \begin{beamercolorbox}[wd=.5\paperwidth,ht=2.25ex,dp=1ex,left]{subsection in head/foot}%
      \usebeamerfont{subsection in head/foot}\hspace*{2ex}\insertsubsectionhead
    \end{beamercolorbox}%
    }%
    \vskip0pt%
    \begin{beamercolorbox}[wd=\paperwidth,colsep=.05ex]{separation line}
    \end{beamercolorbox}
  }%
  \fi
}

\defbeamertemplate*{footline}{jogo theme}
{
  \leavevmode%
    \begin{beamercolorbox}[wd=\paperwidth,colsep=.05ex]{separation line}
    \end{beamercolorbox}
  \hbox{%
  \begin{beamercolorbox}[wd=.25\paperwidth,ht=2.25ex,dp=1ex,left]{author in head/foot}%
    \usebeamerfont{author in head/foot}\insertshortauthor
  \end{beamercolorbox}%
  \begin{beamercolorbox}[wd=.5\paperwidth,ht=2.25ex,dp=1ex,center]{title in head/foot}%
    \usebeamerfont{title in head/foot}\insertshorttitle
  \end{beamercolorbox}%
  \begin{beamercolorbox}[wd=.25\paperwidth,ht=2.25ex,dp=1ex,right]{date in head/foot}%
    \usebeamerfont{date in head/foot}\insertshortdate{}\hspace*{2em}
    \insertframenumber{} / \inserttotalframenumber\hspace*{2ex} 
  \end{beamercolorbox}}%
  \vskip0pt%
}

\makeatother

\usepackage[utf8]{inputenc}
\usepackage[T1]{fontenc}

\usepackage{kpfonts}

\usepackage{amsmath}

\newcommand{\dual}[1]{\ensuremath{\bar{#1}}}

\newcommand{\set}[2]{\ensuremath{\left\{#1\mid#2\right\}}}

\newenvironment<>{subpart}[1]
{ \begin{block}#2{#1}
  \begin{itemize}
}{
  \end{itemize}
  \end{block}
}

\newcommand<>{\singleline}[1]{\begin{block}{\only#2#1}\end{block}}

\usepackage{bussproofs}
\newcommand{\Resolution}[2]{\RightLabel{\footnotesize{$#1$}} \BinaryInfC{$#2$}}

\usepackage[vlined]{algorithm2e}

\usepackage{multicol}

\usepackage{tikz}
\usetikzlibrary{positioning}

\newcommand{\basecharttitle}[1]{{\color{vertfonce}#1}\vspace{-0.8em}}
\makeatletter
\newcommand{\charttitle}{\@ifstar\charttitlestar\charttitleplain}
\makeatother
\newcommand{\charttitleplain}[1]{\hspace{2em}\basecharttitle{#1}}
\newcommand{\charttitlestar}[1]{\centering\basecharttitle{#1}}

\newcommand{\compC}{Compression ratio comparison}
\newcommand{\timeC}{Compression time comparison \scriptsize (in seconds)}
\newcommand{\axioC}{Axiom compression ratio comparison}

\newcommand{\asGoodRPILU}{For a combined algorithm to be always at least as
good as RPI.LU it has to compute root and units safe literals.}
\newcommand{\asGoodLURPI}{For a combined algorithm to be always at least as
good as LU.RPI it has to be able to lower units introduced by RPI.}

\newcommand{\bottomup}{\item[$\uparrow$]}
\newcommand{\topdown} {\item[$\downarrow$]}
\newcommand{\anydir}  {\item[$\updownarrow$]}

\newcommand{\chartscale}{1}

\tikzstyle{proof edge}=[->,thick,cap=round]
\tikzstyle{deleted edge}=[proof edge, dashed, color=deletecol]

\newcommand{\proofnode}[3][]{
  \node [anchor=mid, #1] (#2) {#3}
}

\newcommand{\rootnode}[1][]{
  \proofnode[#1]{root}{$\bot$}
}

\newcommand<>{\edgewithlabel}[3]{
  \draw#4 [proof edge, color=vertfonce] (#1) -- (#2) node [above, pos=0.3] {\footnotesize #3}
}

\newcommand<>{\drawchildren}[3]{
  \draw#4 [proof edge] (#1) -- (#2);
  \draw#4 [proof edge] (#1) -- (#3)
}

\newcommand{\addchildren}[5]{
  \proofnode[above left  of=#1]{#2}{#3};
  \proofnode[above right of=#1]{#4}{#5}
}

\newcommand{\withchildren}[5]{
  \addchildren{#1}{#2}{#3}{#4}{#5};
  \drawchildren{#1}{#2}{#4}
}

\newcommand<>{\crossnode}[2][]{
  \draw#3 [color=deletecol,thick,cap=round,#1] (#2.mid) ++(10:0.3) -- ++(190:0.6);
}

\newcommand<>{\pivot}[3][]{
  \node#4 [anchor=base,color=vertfonce,#1] at (#2.north) {\scriptsize #3};
}

\newcommand<>{\safelit}[3][]{
  \node#4 [anchor=west,color=vertfonce,#1] at (#2.mid) {\hspace{0.2em} \scriptsize $\left\{#3\right\}$};
}

\title[Compression of Propositional Resolution Proofs]{Compression of Propositional Resolution Proofs by Lowering Subproofs}
\author[J. Boudou, B. Woltzenlogel Paleo]{
  Joseph Boudou\inst{1}
  \and 
  Bruno Woltzenlogel Paleo\inst{2}
}
\institute{
  \inst{1}Universit\'e Paul Sabatier, Toulouse
  \and 
  \inst{2}Vienna University of Technology
}
\date[PCC 2013]{PCC Workshop, 2013}

\includeonly{}
%%%%%%%%%%%%%%%%%%
% Begin Document %
%%%%%%%%%%%%%%%%%%
\begin{document}

\begin{frame}[plain]
\titlepage
\end{frame}

\begin{frame}{Overview}
\tableofcontents
\end{frame}

\section{Introduction}

\subsection{Propositional resolution calculus}


\begin{frame}
\begin{subpart}{Literals}
  \item A \emph{literal} is an atomic formula or the negation of an atomic formula.
  \item The dual of a literal $\ell$, denoted by $\dual{\ell}$, is such that
        $\dual{p} = \neg p$ and $\dual{\neg p} = p$.
\end{subpart}
\begin{subpart}{Clauses}
  \item A \emph{clause} is a disjonctive set of literals.
  \item The empty clause corresponds to false.% and is denoted by $\bot$.
  \item Tautologies contain both a literal $\ell$ and its dual $\dual{\ell}$.
\end{subpart}
%\begin{subpart}{Instance}
%  \item An \emph{instance} is a conjonctive set of clauses.
%\end{subpart}
%  \begin{subpart}{SAT context}
%    \item Axioms are instance's clauses.
%    \item Proofs of unsatisfiability.
%    \item Tautologies are prohibited.
%  \end{subpart}
  \begin{block}{Robinson's Resolution Principle}
    \vspace{-.5em}
    \begin{prooftree}
      \AxiomC{$\Gamma, \dual{\ell}$}
      \AxiomC{$\ell, \Delta$}
      \Resolution{\ell}{\Gamma, \Delta}
    \end{prooftree}
  \end{block}
\end{frame}

\subsection{Proofs as DAG}

\begin{frame}
  \begin{block}{Proof as a tree}
    \vspace{-.5em}
    \begin{prooftree}
      \AxiomC{$\dual{b}, c$}
      \AxiomC{$\dual{a}, b$}
      \Resolution{b}{\dual{a}, c}
      \AxiomC{$\dual{a}, b$}
      \AxiomC{$\dual{a}, \dual{b}, \dual{c}$}
      \Resolution{\dual{b}}{\dual{a}, \dual{c}}
      \Resolution{\dual{c}}{\dual{a}}
      \AxiomC{$a$}
      \Resolution{a}{\bot}
    \end{prooftree}
  \end{block}
  \begin{block}{Proof as a directed acyclic graph (DAG)}
    \vspace{.5em}
      \centering
      \begin{tikzpicture}[node distance=1.2cm]
        \rootnode;
        \withchildren{root} {r0}{\dual{a}}  {unit}{a};
        \withchildren{r0}   {r1}{\dual{a},c} {r2}{\dual{a},\dual{c}};
        \withchildren{r1}   {a0}{\dual{b},c} {low}{\dual{a},b};
        \proofnode[above right of=r2] {a1} {\dual{a},\dual{b},\dual{c}};
        \drawchildren {r2} {low} {a1};
      \end{tikzpicture}
  \end{block}
\end{frame}

\newcommand{\edge}[3]{\ensuremath{#1 \xrightarrow{#2} #3}}

\begin{frame}
  \begin{definition}[Proof]
    A graph $\varphi = \langle V, E, L \rangle$ where $V$ is a set of node,
    $E \subset V \times \mathcal{L} \times V$ a set of edges labeled by literals and $L: V
    \longrightarrow 2^\mathcal{L}$ an edge labeling function, is a proof of $\Gamma$ iff either : \begin{itemize}
    \item $v$ is a new node, $V=\{v\}$, $E=\varnothing$ and $L=\{ v \longmapsto \Gamma \}$;
    \item $\varphi_L = \langle V_L, E_L, L_L \rangle$ is a proof of $\Gamma_L$, $\varphi_R = \langle V_R, E_R, L_R \rangle$
      is a proof of $\Gamma_R$, $\ell$ is a literal s.t. $\dual{\ell} \in \Gamma_L$ and $\ell \in
      \Gamma_R$, $v$ is a new node and \begin{align*}
        V &= V_L \cup V_R \cup \{v\} \\
        E &= E_L \cup E_R \cup
             \{ \edge{v}{\dual{\ell}}{\rho(\varphi_L)}, \edge{v}{\ell}{\rho(\varphi_R)} \} \\
        L &= L_L \cup L_R \cup \{v \longmapsto \Gamma\} \\
   \Gamma &= \left( \Gamma_L \setminus \{\dual{\ell}\} \right) \cup
             \left( \Gamma_R \setminus \{\ell\} \right)
      \end{align*} where $\rho(\varphi)$ denotes the root of $\varphi$.
    \end{itemize}
  \end{definition}
\end{frame}

\begin{frame}
  \begin{definition}[Directed Acyclic Graph with root]
    A graph $\gamma = \langle V, E, L \rangle$ where $V$ is a set of node,
    $E \subset V \times X \times V$ a set of labeled directed edges and $L: V
    \longrightarrow Y$ a node labeling function, is a directed acyclic graph with root iff
    there exists a node $\rho(\gamma)$ such that each other node is reachable from $\rho(\gamma)$
    but $\rho(\gamma)$ is not reachable from any node.
  \end{definition}

  \begin{definition}[Proof]
    A directed acyclic graph $\langle V, E, L \rangle$ where $V$ is a set of node,
    $E \subset V \times \mathcal{L} \times V$ a set of edges labeled by literals and $L: V
    \longrightarrow 2^\mathcal{L}$ an edge labeling function, is a proof iff it is inductively
    constructible according to the following cases : \begin{itemize}
    \item if $\Gamma$ is a clause then $\langle \{v\}, \varnothing, v \longmapsto \Gamma \rangle$ is a
      proof for any new $v$;
    \item if $\langle V_L, E_L, L_L \rangle$ and $\langle V_R, E_R, L_R \rangle$ are proofs and
      for a given literal $\ell$, $\ell \in L$
    \end{itemize}
  \end{definition}
\end{frame}


\begin{frame}{Proof as DAG}
  \begin{definition}[Proof]
    A proof is a tuple $\langle V, E, \lambda_V, \lambda_E, \rho \rangle$ s.t.: \begin{itemize}
    \item $\langle V, E \rangle$ is a directed acyclic graph (DAG);
    \item $\lambda_V$ is a function labeling nodes with clauses;
    \item $\lambda_E$ is a function labeling edges with literals;
    \item $\rho \in V$ and $\forall \eta \in V \setminus \{\rho\}$, a path exists from $\rho$ to $\eta$;
    \item for each incoming edge $\edge{\eta'}{\ell}{\eta}$, $\ell \in \lambda_V(\eta)$;
    \item a node $\eta \in V$ without any outgoing edge is called an axiom;
    \item a node $\eta$ with exactely two outgoing edges $\edge{\eta}{\ell_L}{\eta_L}$ and
          $\edge{\eta}{\ell_R}{\eta_R}$ is called a resolution step and is s.t. $\dual{\ell_L} = \ell_R$ and
          $ \lambda_V(\eta) = \left( \lambda_V(\eta_L) \setminus \{ \ell_L \} \right)
                         \cup \left( \lambda_V(\eta_R) \setminus \{ \ell_R \} \right) $;
    \item each node $\eta \in V$ is either an axiom or a resolution step.
    \end{itemize}
  \end{definition}
\end{frame}
    

\begin{frame}{Subproofs}
  \begin{definition}[Subproof]
    A proof $\psi' = \langle V', E', \lambda_V', \lambda_E', \rho' \rangle$ is a subproof of a proof
    $\psi = \langle V, E, \lambda_V, \lambda_E, \rho \rangle$ iff there exists an injective
    function $f$ from $V'$ to $V$ such that: \begin{itemize}
    \item $\forall \eta' \in V',~ \lambda_V'(\eta') = \lambda_V(f(\eta'))$;
    \item $\forall (\eta_F', \eta_T') \in E',~ (f(\eta_F'), f(\eta_T')) \in E$ and
          $\lambda_E'(\eta_F', \eta_T') = \lambda_E(f(\eta_F'), f(\eta_T'))$;
    \end{itemize}
  \end{definition}
  \begin{subpart}{Properties}
    \item Subproof is a partial order relation on proofs, noted $\sqsubseteq$.
    \item The relation $\equiv$ such that
          $\psi' \equiv \psi \Leftrightarrow \psi' \sqsubseteq \psi \wedge \psi \sqsubseteq \psi'$
          is an equivalence relation on proofs.
  \end{subpart}
\end{frame}

\begin{frame}{Resolution operator}
  \begin{definition}
    TODO
  \end{definition}
\end{frame}

\section{Redundancies and corresponding algorithms}

\frame{\tableofcontents[sectionstyle=show/shaded,subsectionstyle=show/show/hide]}

\subsection{Vertical redundancy}

\begin{frame}{Regular proof}
\begin{definition}[Tseitin 1970]
  A proof is regular iff on every path from its root to any of its axiom, any literal
  appears at most once as edge label.
\end{definition}
\begin{theorem}[Goerdt 1990]
  Given a set of axioms and a clause $\Gamma$, the smallest regular proof of $\Gamma$ might be exponentialy bigger
  than the smallest irregular proof of $\Gamma$.
\end{theorem}
\end{frame}

\begin{frame}{RecyclePivotsWithIntersection (RPI)}
  \begin{subpart}{Partial Regularization}
    \item Delete an outgoing edge labeled with $\ell$ iff $\dual{\ell}$ appears on \textbf{every} path from the root to the node.
  \end{subpart}
  \begin{definition}[Safe literal]
    A literal is safe for a node $\eta$ if it can be added to $\eta$'s clause without changing proof's conclusion.
  \end{definition}
  \begin{subpart}{Two traversals}
    \bottomup Collect safe literals and mark edges to delete.
    \topdown Delete edges and fix the proof.
  \end{subpart}
\end{frame}

\begin{frame}
  \begin{columns}
    \alt<3->{\column{0.5\textwidth}}{\column{1\textwidth}}
    \begin{center}
    \begin{tikzpicture}[node distance=1.5cm]

      \rootnode;
      \addchildren{root} {a4}{$\dual{c}$} {r2}{$c$};
      \draw [proof edge] (root) -- (a4);
      \draw<1> [proof edge] (root) -- (r2);
      \edgewithlabel<2->{root}{r2}{$c$};

      \withchildren{r2} {r0}{$b$} {r1}{$\dual{b},c$};

      \withchildren{r0} {a0}{$\dual{a},b$} {iu}{$a$};

      \addchildren{iu} {a2}{$a,\dual{c}$} {a1}{$a,c$};
      \draw [proof edge] (iu) -- (a2);
      \draw<1> [proof edge] (iu) -- (a1);
      \edgewithlabel<2->{iu}{a1}{$c$};

      \proofnode[above right of=r1]{a3}{$\dual{a},\dual{b},c$};
      \draw [proof edge] (r1) -- (a3);
      \draw [proof edge] (r1) -- (iu);
    \end{tikzpicture}
    \only<3->{Original proof}
    \end{center}
%    \begin{onlyenv}<3->
    \only<3->{
    \column{0.5\textwidth}
    \begin{center}
    \begin{tikzpicture}[node distance=1.5cm]

      \rootnode;
      \withchildren{root} {a4}{$\dual{c}$} {r2}{$c$};

      \withchildren{r2} {r0}{\alt<4>{$b,c$}{$b$}} {r1}{$\dual{b},c$};

      \withchildren{r0} {a0}{$\dual{a},b$} {iu}{\alt<4>{$a,c$}{$a$}};

      \addchildren{iu} {a2}{$a,\dual{c}$} {a1}{\alt<4>{$a$}{$a,c$}};
      \draw<3> [deleted edge] (iu) -- (a2);
      \draw<3> [proof edge] (iu) -- (a1);

      \proofnode[above right of=r1]{a3}{$\dual{a},\dual{b},c$};
      \draw [proof edge] (r1) -- (a3);
      \draw [proof edge] (r1) -- (iu);

      \crossnode<4>{a2}
      \crossnode<4>{a1}
    \end{tikzpicture}
    Compressed proof
    \end{center}
  }
%    \end{onlyenv}
  \end{columns}
\end{frame}


\subsection{Horizontal redundancy}

\begin{frame}{Extending Irregularity}
\begin{definition}[Fully regular proof]
  A proof is fully regular if for each variable there is at most one resolution node with this variable as pivot.
\end{definition}
\begin{subpart}{Conventions}
  \item Usual irregularities are called \emph{vertical irregularities}.
  \item Other irregularities are called \emph{horizontal irregularities}.
\end{subpart}
\end{frame}

\begin{frame}{LowerUnits (LU)}
  \begin{subpart}{Lowering}
    \item Moving a node down the proof to resolve it only once.
  \end{subpart}
  \begin{subpart}{Lowering Units}
    \item Units can always be lowered.
    \item Reduces horizontal irregularities.
  \end{subpart}
  \begin{subpart}{Two traversals}
    \anydir Collect units with more than one child.
    \topdown Delete units, fix the proof and then reintroduce the units at the bottom of the proof.
  \end{subpart}
\end{frame}

\begin{frame}{Sequential Composition}
\centering
\charttitle{\compC}
\begin{tikzpicture}

\draw (0,0) -- (5,0);
\node at (2.5,-0.6) {LU.RPI};
\node [anchor=north] at (0.714285714285714,0) {\small 0.1};
\draw (0.714285714285714,0) -- (0.714285714285714,0.1);
\draw [style=help lines] (0.714285714285714,0) -- (0.714285714285714,5);
\node [anchor=north] at (1.42857142857143,0) {\small 0.2};
\draw (1.42857142857143,0) -- (1.42857142857143,0.1);
\draw [style=help lines] (1.42857142857143,0) -- (1.42857142857143,5);
\node [anchor=north] at (2.14285714285714,0) {\small 0.3};
\draw (2.14285714285714,0) -- (2.14285714285714,0.1);
\draw [style=help lines] (2.14285714285714,0) -- (2.14285714285714,5);
\node [anchor=north] at (2.85714285714286,0) {\small 0.4};
\draw (2.85714285714286,0) -- (2.85714285714286,0.1);
\draw [style=help lines] (2.85714285714286,0) -- (2.85714285714286,5);
\node [anchor=north] at (3.57142857142857,0) {\small 0.5};
\draw (3.57142857142857,0) -- (3.57142857142857,0.1);
\draw [style=help lines] (3.57142857142857,0) -- (3.57142857142857,5);
\node [anchor=north] at (4.28571428571429,0) {\small 0.6};
\draw (4.28571428571429,0) -- (4.28571428571429,0.1);
\draw [style=help lines] (4.28571428571429,0) -- (4.28571428571429,5);
\node [anchor=north] at (5,0) {\small 0.7};
\draw (5,0) -- (5,0.1);
\draw [style=help lines] (5,0) -- (5,5);
\draw (0,0) -- (0,5);
\node [rotate=90] at (-2.5em,2.5) {RPI.LU};
\node [anchor=east] at (0,0.713285714285714) {\small 0.1};
\draw (0,0.713285714285714) -- (0.1,0.713285714285714);
\draw [style=help lines] (0,0.713285714285714) -- (5,0.713285714285714);
\node [anchor=east] at (0,1.42657142857143) {\small 0.2};
\draw (0,1.42657142857143) -- (0.1,1.42657142857143);
\draw [style=help lines] (0,1.42657142857143) -- (5,1.42657142857143);
\node [anchor=east] at (0,2.13985714285714) {\small 0.3};
\draw (0,2.13985714285714) -- (0.1,2.13985714285714);
\draw [style=help lines] (0,2.13985714285714) -- (5,2.13985714285714);
\node [anchor=east] at (0,2.85314285714286) {\small 0.4};
\draw (0,2.85314285714286) -- (0.1,2.85314285714286);
\draw [style=help lines] (0,2.85314285714286) -- (5,2.85314285714286);
\node [anchor=east] at (0,3.56642857142857) {\small 0.5};
\draw (0,3.56642857142857) -- (0.1,3.56642857142857);
\draw [style=help lines] (0,3.56642857142857) -- (5,3.56642857142857);
\node [anchor=east] at (0,4.27971428571429) {\small 0.6};
\draw (0,4.27971428571429) -- (0.1,4.27971428571429);
\draw [style=help lines] (0,4.27971428571429) -- (5,4.27971428571429);
\node [anchor=east] at (0,4.993) {\small 0.7};
\draw (0,4.993) -- (0.1,4.993);
\draw [style=help lines] (0,4.993) -- (5,4.993);

\foreach \pos in {
	(1.556186, 1.556186),
	(2.287890, 2.342277),
	(2.034038, 2.062508),
	(1.070455, 1.158038),
	(1.190476, 1.201201),
	(0.480619, 0.484526),
	(1.349544, 1.513678),
	(0.250054, 0.250054),
	(1.972249, 2.046426),
	(1.440906, 1.732451),
	(1.363287, 1.420891),
	(1.745928, 1.802097),
	(1.032574, 1.165638),
	(2.179622, 2.242647),
	(1.552430, 1.604737),
	(0.898964, 0.908736),
	(1.510241, 1.756547),
	(1.728723, 2.026342),
	(1.602943, 1.684679),
	(2.587883, 3.246073),
	(2.077456, 2.115927),
	(1.750266, 1.900922),
	(1.620730, 1.926436),
	(1.996380, 2.245221),
	(1.712040, 2.002058),
	(1.686183, 1.861827),
	(1.246753, 1.474026),
	(2.119953, 2.173218),
	(1.521739, 1.310559),
	(1.839907, 1.903352),
	(1.668702, 1.729752),
	(1.642036, 2.151067),
	(1.760131, 1.794242),
	(0.994575, 1.410488),
	(1.714286, 1.755952),
	(2.059437, 2.075078),
	(1.361273, 1.827776),
	(1.560806, 1.560806),
	(1.321892, 1.321892),
	(1.634847, 1.634847),
	(1.104901, 1.104901),
	(1.719177, 2.208315),
	(1.754729, 2.100457),
	(1.396348, 1.450054),
	(1.004318, 1.210813),
	(1.517165, 1.860465),
	(2.024264, 2.167394),
	(1.326885, 1.444692),
	(1.729911, 2.032844),
	(1.400308, 1.464543),
	(1.448652, 1.627940),
	(1.495197, 1.634645),
	(1.783416, 2.307409),
	(1.308216, 1.613466),
	(1.342857, 1.585714),
	(1.631117, 1.971680),
	(1.489621, 1.636142),
	(0.129359, 0.134983),
	(0.469764, 0.474354),
	(1.801242, 2.236025),
	(1.866475, 1.920316),
	(1.403908, 1.811800),
	(1.522052, 1.605944),
	(1.630435, 1.727484),
	(1.250955, 1.250955),
	(1.785714, 1.800115),
	(1.618804, 1.463325),
	(0.933442, 0.963880),
	(1.243622, 1.530612),
	(1.538951, 1.560178),
	(1.838755, 1.867044),
	(1.957224, 1.997579),
	(1.535598, 1.713975),
	(1.710076, 1.710076),
	(1.394286, 1.577143),
	(1.392857, 1.464286),
	(1.553288, 1.575964),
	(1.815062, 1.887304),
	(1.617681, 1.822739),
	(1.251476, 1.227863),
	(1.194162, 1.194162),
	(1.661753, 1.755562),
	(1.813616, 1.813616),
	(1.624030, 1.925841),
	(2.376477, 2.537594),
	(1.551459, 1.628264),
	(1.720210, 1.742994),
	(1.655052, 1.916376),
	(1.941610, 2.097506),
	(1.562500, 1.596841),
	(0.267681, 0.267681),
	(1.926164, 1.926164),
	(0.267857, 0.287698),
	(0.267857, 0.287698),
	(1.251618, 1.273198),
	(1.071429, 1.071429),
	(1.020408, 1.916980),
	(0.850181, 0.853528),
	(0.676692, 0.789474),
	(0.875576, 0.875576),
	(0.115830, 0.115830),
	(1.392857, 1.392857),
	(1.344086, 1.344086),
	(1.344086, 1.344086),
	(0.418443, 0.418443),
	(1.561022, 1.655629),
	(0.315770, 0.324863),
	(0.127767, 0.127767),
	(0.127767, 0.129112),
	(0.545889, 0.548326),
	(0.369922, 0.410277),
	(1.034964, 1.099314),
	(0.294922, 0.294922),
	(1.059771, 1.066836),
	(1.378245, 1.355370),
	(0.473934, 3.368314),
	(0.527024, 0.527024),
	(0.683230, 0.708075),
	(1.104323, 1.409774),
	(1.235231, 1.235231),
	(0.253593, 0.253593),
	(1.751152, 1.751152),
	(1.095994, 1.927438),
	(0.992063, 0.396825),
	(0.601504, 0.751880),
	(0.480769, 0.480769),
	(0.514801, 0.772201),
	(1.749271, 1.749271),
	(0.588235, 1.512605),
	(1.632653, 1.632653),
	(0.000000, 0.000000),
	(0.148810, 0.148810),
	(1.020408, 1.020408),
	(0.696864, 1.219512),
	(0.744048, 0.744048),
	(0.533049, 0.533049),
	(1.166181, 1.166181),
	(1.288056, 1.522248),
	(0.420168, 0.840336),
	(0.000000, 0.000000),
	(1.428571, 2.142857),
	(3.125000, 2.678571),
	(4.395604, 4.395604),
	(1.586670, 1.899454),
	(2.175244, 1.864130),
	(1.744072, 1.949833),
	(1.187591, 1.371096),
	(1.531900, 1.728143),
	(1.572865, 1.692539),
	(2.033262, 2.120792),
	(1.142061, 1.388778),
	(1.338802, 1.547748),
	(1.843231, 1.940670),
	(1.676689, 1.836503),
	(1.694212, 1.792318),
	(1.365061, 1.542395),
	(1.412464, 1.455830),
	(1.198943, 1.283614),
	(2.406748, 2.816901),
	(1.285925, 1.563488),
	(1.802911, 2.038230),
	(1.820728, 1.884921),
	(1.651348, 1.701979),
	(1.515290, 1.538110),
	(1.600731, 1.709805),
	(2.004808, 2.051342),
	(1.034028, 1.078029),
	(1.837379, 1.837379),
	(1.882698, 2.205523),
	(2.475999, 2.646262),
	(1.381034, 1.651504),
	(2.128173, 2.134208),
	(1.537839, 1.537839),
	(1.987146, 2.164359),
	(2.105427, 2.105427),
	(1.496313, 1.607786),
	(1.839767, 2.386194),
	(1.464179, 1.597045),
	(1.369963, 1.853480),
	(1.587302, 1.591711),
	(2.073922, 2.340862),
	(1.983941, 2.134493),
	(2.447045, 2.809749),
	(1.678257, 1.713848),
	(1.998149, 2.082282),
	(1.675048, 1.791912),
	(1.698166, 1.806559),
	(1.962624, 2.099460),
	(1.515869, 2.108006),
	(2.403414, 2.425876),
	(1.434531, 1.890005),
	(1.657754, 1.791444),
	(1.776185, 1.802866),
	(1.966279, 1.985603),
	(1.306973, 1.480685),
	(1.954115, 1.984127),
	(1.344390, 1.470034),
	(2.321249, 2.335578),
	(1.254181, 1.230291),
	(2.496633, 1.419248),
	(1.817867, 1.936585),
	(2.262094, 2.502839),
	(1.531737, 1.539613),
	(1.336779, 1.813564),
	(1.675898, 2.029221),
	(1.692737, 1.936119),
	(1.324194, 1.417666),
	(1.982212, 2.261445),
	(2.040241, 2.140845),
	(1.766600, 2.305835),
	(1.443093, 1.443093),
	(1.378106, 1.378106),
	(1.453373, 1.661706),
	(2.125066, 2.477596),
	(1.979658, 1.988277),
	(1.890915, 1.901705),
	(0.113773, 0.114527),
	(1.503759, 1.597744),
	(1.051709, 1.139351),
	(1.331558, 1.464714),
	(1.735834, 1.900439),
	(1.394155, 1.493321),
	(1.761834, 1.793674),
	(1.387363, 1.575092),
	(2.060762, 2.306489),
	(1.640146, 1.675705),
	(1.797424, 1.812061),
	(1.141527, 1.163693),
	(2.555408, 1.842842),
	(1.376804, 1.508977),
	(1.220488, 1.293851),
	(2.119514, 2.131964),
	(1.355534, 1.850706),
	(2.098404, 2.275112),
	(1.591441, 1.873536),
	(1.325848, 1.391543),
	(2.142560, 2.175203),
	(1.675557, 1.808905),
	(1.501966, 1.627785),
	(1.764079, 1.810678),
	(2.097527, 2.228958),
	(1.968380, 2.037667),
	(1.300044, 1.440390),
	(2.382885, 2.481447),
	(1.808395, 1.814443),
	(1.498423, 1.464624),
	(1.566475, 1.566475),
	(1.361139, 1.467283),
	(1.575290, 1.575290),
	(1.503759, 1.564395),
	(1.487100, 1.684185),
	(1.749971, 1.772847),
	(2.077654, 2.188168),
	(1.442929, 1.514716),
	(1.594286, 1.634286),
	(1.604295, 1.539212),
	(1.675406, 1.710577),
	(2.318548, 2.599366),
	(1.401230, 1.196172),
	(1.791675, 1.784863),
	(1.217965, 1.243339),
	(2.038509, 2.161272),
	(1.897547, 1.897547),
	(0.825688, 1.215596),
	(1.504414, 1.647395),
	(1.037102, 1.080923),
	(1.362835, 1.507135),
	(1.657716, 1.672037),
	(2.333044, 2.356441),
	(1.515152, 1.515152),
	(2.114967, 1.576542),
	(1.782247, 1.782247),
	(1.319261, 1.319261),
	(2.327554, 2.336977),
	(1.888574, 2.977877),
	(2.023875, 2.221519),
	(2.492651, 3.409759),
	(2.294343, 2.470830),
	(2.592257, 2.705456),
	(1.762918, 1.863602),
	(2.055279, 2.076169),
	(1.451696, 1.622987),
	(1.189882, 1.339508),
	(2.147440, 2.223823),
	(1.845883, 1.881462),
	(1.967478, 2.241101),
	(1.729115, 2.145121),
	(1.612379, 2.344689),
	(1.632377, 1.728967),
	(1.420904, 1.440073),
	(1.578216, 1.950282),
	(1.495215, 1.677489),
	(1.869398, 2.165722),
	(1.772814, 1.818885),
	(1.454784, 1.643512),
	(1.788087, 1.808967),
	(2.412670, 2.429586),
	(1.472370, 1.480135),
	(1.466783, 1.488676),
	(2.495288, 2.212590),
	(1.894320, 1.917962),
	(1.689890, 1.713996),
	(1.626184, 1.821737),
	(1.783634, 1.864078),
	(2.031056, 2.105590),
	(1.663276, 1.845992),
	(1.762005, 1.791954),
	(2.076180, 2.164018),
	(2.183990, 2.209045),
	(1.851003, 1.887657),
	(2.065121, 2.363503),
	(2.375992, 2.415675),
	(1.808556, 1.932172),
	(2.382475, 2.665723),
	(1.611868, 1.676057),
	(2.135383, 2.188768),
	(1.633177, 1.536838),
	(1.887631, 1.957185),
	(1.629223, 1.660664),
	(1.117300, 1.376147),
	(1.710821, 1.712625),
	(1.226593, 1.238858),
	(1.700334, 1.759919),
	(2.072264, 2.732655),
	(1.839259, 1.858000),
	(1.789455, 1.852295),
	(1.378043, 1.351542),
	(1.756292, 1.946406),
	(2.057873, 2.196001),
	(2.836015, 2.862425),
	(1.572823, 1.606158),
	(1.489970, 1.491578),
	(2.032367, 2.131835),
	(1.847635, 1.905561),
	(2.011952, 2.046101),
	(1.591822, 1.622336),
	(1.531188, 1.670875),
	(1.384083, 1.586752),
	(1.765717, 1.833707),
	(1.519914, 1.684428),
	(1.718278, 1.872033),
	(1.893842, 1.865691),
	(2.379232, 2.477291),
	(1.808954, 1.874278),
	(1.875119, 1.881448),
	(2.374535, 2.445129),
	(1.893018, 2.079095),
	(1.862976, 2.028380),
	(1.153822, 1.545866),
	(1.203071, 1.130524),
	(2.146034, 2.212466),
	(1.383163, 1.532533),
	(1.254010, 1.254010),
	(2.077160, 2.289628),
	(2.205987, 2.310611),
	(1.442211, 1.470293),
	(1.397789, 1.451187),
	(1.668802, 1.832159),
	(1.499353, 1.624933),
	(1.626941, 1.717108),
	(2.449194, 2.489140),
	(1.383526, 1.544402),
	(1.391891, 1.535993),
	(1.646168, 1.632923),
	(1.880081, 2.217625),
	(1.771255, 1.818248),
	(1.052265, 1.379791),
	(1.341132, 1.559731),
	(1.654280, 1.929535),
	(1.611082, 1.625194),
	(1.474886, 1.499697),
	(2.043613, 2.098422),
	(2.159548, 2.179985),
	(1.849571, 1.897464),
	(1.539288, 1.793890),
	(1.238938, 1.396966),
	(1.323676, 1.402764),
	(1.505266, 1.817193),
	(1.795777, 1.866989),
	(1.045770, 1.059639),
	(1.630812, 1.670467),
	(1.537232, 1.550016),
	(1.580370, 1.701670),
	(1.794193, 1.872202),
	(1.705473, 1.523229),
	(1.469150, 1.522930),
	(1.353591, 1.515391),
	(1.579652, 1.661519),
	(1.427123, 1.470588),
	(1.531051, 1.564801),
	(1.729160, 1.803533),
	(2.365037, 2.371403),
	(1.422329, 1.582843),
	(1.893524, 1.897444),
	(1.809618, 1.905234),
	(2.219283, 2.333186),
	(2.200772, 2.209965),
	(2.506510, 2.622768),
	(2.193017, 2.445758),
	(1.802089, 1.887060),
	(2.171492, 2.569201),
	(1.412873, 1.439037),
	(2.511808, 2.527142),
	(1.045296, 1.447333),
	(2.203826, 2.304508),
	(1.647555, 1.941144),
	(1.700159, 1.648141),
	(1.456767, 1.625157),
	(2.092555, 2.363271),
	(1.682253, 1.709126),
	(1.786857, 1.896536),
	(1.785094, 1.947713),
	(1.036933, 1.127820),
	(2.099666, 2.113921),
	(2.311745, 2.351874),
	(1.826880, 1.873678),
	(1.637749, 1.790174),
	(1.400232, 1.493174),
	(2.098353, 2.162884),
	(1.669440, 1.712924),
	(1.828328, 1.816153),
	(1.768833, 2.037065),
	(1.747482, 1.609474),
	(1.560796, 1.665304),
	(1.840812, 1.858253),
	(1.751492, 1.797122),
	(1.807419, 2.178374),
	(1.856951, 1.962605),
	(1.627968, 1.762656),
	(2.299766, 2.309133),
	(1.541943, 1.590930),
	(1.628007, 1.711998),
	(2.560922, 2.885598),
	(2.243590, 2.389674),
	(1.767968, 1.867791),
	(1.645574, 1.696880),
	(1.452330, 1.486818),
	(1.192591, 1.236996),
	(1.266373, 1.270211),
	(1.923408, 1.942054),
	(1.577164, 2.009906),
	(1.302232, 1.527108),
	(2.507015, 2.515148),
	(1.898485, 1.966366),
	(2.132984, 1.800064),
	(2.036762, 2.189574),
	(2.285101, 2.453800),
	(2.113763, 2.668923),
	(1.964602, 2.252845),
	(1.738473, 1.747366),
	(1.897526, 1.933306),
	(1.702393, 1.721575),
	(1.736094, 1.824308),
	(1.364210, 1.504327),
	(1.234538, 1.489207),
	(2.087097, 2.087097),
	(1.698806, 1.744720),
	(1.977848, 2.158680),
	(1.814809, 1.848541),
	(1.899673, 1.948437),
	(1.693841, 1.702289),
	(1.998251, 2.431524),
	(1.943060, 2.012757),
	(1.070006, 1.092772),
	(1.355271, 1.537204),
	(2.621746, 2.669490),
	(1.995565, 2.038328),
	(1.469311, 1.916784),
	(1.627434, 1.656275),
	(1.391290, 1.431813),
	(1.898734, 2.022303),
	(2.286076, 2.304147),
	(1.791097, 1.877227),
	(2.086301, 2.086301),
	(1.625631, 1.705403),
	(1.431413, 1.584144),
	(1.613008, 2.043359),
	(1.836461, 1.920720),
	(1.299369, 2.121568),
	(1.546518, 1.580776),
	(1.259863, 1.296918),
	(1.832942, 1.036455),
	(1.438383, 1.517222),
	(1.677041, 1.839086),
	(1.930023, 2.069906),
	(1.987215, 2.016994),
	(2.062723, 2.193807),
	(1.832516, 1.840426),
	(2.083051, 2.141729),
	(1.572899, 1.692849),
	(2.519119, 2.549709),
	(1.335205, 1.808566),
	(2.249473, 2.265094),
	(1.826061, 1.876868),
	(1.796442, 2.092923),
	(1.719945, 1.830481),
	(1.775939, 1.802692),
	(1.546425, 1.634327),
	(1.414824, 1.646809),
	(1.887039, 1.965484),
	(2.387984, 2.463927),
	(1.482290, 1.505792),
	(1.394069, 1.538605),
	(1.801298, 1.846438),
	(1.984347, 2.120515),
	(2.048752, 2.192524),
	(1.589338, 1.853791),
	(1.610707, 1.727733),
	(1.632170, 1.684760),
	(1.549480, 1.857197),
	(1.957408, 2.050429),
	(1.858006, 1.879586),
	(1.430515, 1.593780),
	(1.340111, 1.350013),
	(1.641545, 1.690364),
	(1.263142, 1.270616),
	(1.612848, 1.647164),
	(1.909477, 1.942480),
	(2.635892, 2.697350),
	(1.602488, 1.620315),
	(1.412663, 1.568565),
	(2.073593, 2.145743),
	(1.517067, 1.580278),
	(1.863580, 1.889535),
	(1.510949, 1.533717),
	(1.973684, 2.030430),
	(2.073234, 2.088706),
	(2.236386, 2.236386),
	(1.619337, 1.643182),
	(1.801837, 1.899551),
	(1.288265, 1.353635),
	(2.105394, 2.860858),
	(1.682300, 1.817758),
	(1.053384, 1.054300),
	(1.510470, 1.851307),
	(1.288154, 1.306003),
	(1.546919, 1.655820),
	(2.843310, 2.948944),
	(1.884841, 2.031376),
	(2.009978, 2.067543),
	(1.478823, 1.574539),
	(2.304933, 2.380525),
	(1.614907, 1.784897),
	(1.464765, 1.541409),
	(1.483227, 1.560458),
	(1.645591, 1.714617),
	(1.566166, 1.602625),
	(1.917649, 2.051117),
	(1.602816, 1.638216),
	(2.142751, 2.174653),
	(1.941406, 1.952237),
	(1.871650, 1.880735),
	(1.917606, 1.793473),
	(1.750732, 1.856492),
	(2.246004, 2.355136),
	(2.322498, 2.398061),
	(1.775784, 1.915977),
	(1.504827, 1.636651),
	(1.594460, 1.716863),
	(1.555071, 1.648855),
	(1.481285, 1.847009),
	(3.137066, 3.156938),
	(1.358589, 1.503245),
	(2.208992, 2.312902),
	(1.868323, 1.869565),
	(1.514856, 1.559755),
	(1.609885, 1.684088),
	(1.913709, 2.044604),
	(2.226589, 2.435065),
	(1.711722, 1.755080),
	(1.835585, 2.051832),
	(1.326324, 1.367319),
	(1.484980, 1.571632),
	(2.045954, 2.047952),
	(1.886705, 1.993499),
	(1.417445, 1.494848),
	(2.052899, 2.095187),
	(1.700106, 1.719716),
	(2.000471, 2.009885),
	(1.931649, 1.972334),
	(2.221111, 2.358564),
	(1.272718, 1.341118),
	(2.388601, 2.415721),
	(1.442239, 1.460222),
	(1.817961, 1.878661),
	(1.421137, 1.478132),
	(1.225555, 1.281505),
	(1.902421, 1.898290),
	(2.156244, 2.226366),
	(1.588889, 1.787500),
	(1.743085, 1.764136),
	(1.563544, 1.618669),
	(1.914553, 2.132176),
	(2.378319, 2.413085),
	(1.389245, 1.427717),
	(0.946980, 1.153696),
	(2.223814, 2.352079),
	(2.126636, 2.135549),
	(1.511381, 1.767048),
	(1.310380, 1.376756),
	(2.211860, 2.251245),
	(2.122959, 2.447279),
	(1.670795, 1.703826),
	(2.443057, 2.988029),
	(1.476190, 1.680556),
	(2.012490, 2.053514),
	(1.937736, 1.944159),
	(1.624234, 1.685526),
	(1.902349, 1.918436),
	(1.441603, 1.504724),
	(1.651948, 1.688312),
	(1.448953, 1.532333),
	(1.849003, 1.887458),
	(1.614231, 1.673678),
	(3.158141, 3.155542),
	(1.985400, 2.013225),
	(2.536341, 2.537594),
	(2.523984, 3.314720),
	(1.487741, 1.555688),
	(1.638916, 1.766122),
	(1.791375, 1.894530),
	(1.533923, 1.620312),
	(2.242884, 2.276268),
	(1.532980, 1.740139),
	(1.870102, 2.095135),
	(1.328904, 1.819516),
	(1.418847, 1.617751),
	(1.912342, 1.933016),
	(1.511370, 1.543063),
	(2.566674, 2.616819),
	(1.848164, 1.893735),
	(1.448312, 1.651761),
	(1.754951, 1.772530),
	(1.785714, 1.785714),
	(2.211038, 2.292769),
	(1.917073, 1.950051),
	(1.912613, 1.919381),
	(2.218768, 2.271259),
	(1.783962, 1.845297),
	(1.291069, 1.334180),
	(1.573098, 1.643535),
	(1.748905, 1.791999),
	(1.519003, 1.581514),
	(1.414709, 1.556890),
	(1.769512, 1.967870),
	(1.730722, 1.771966),
	(2.222737, 2.242031),
	(2.094718, 2.130858),
	(1.748478, 1.778082),
	(2.393368, 2.400053),
	(1.845539, 1.985402),
	(1.897679, 1.925922),
	(2.050594, 2.131358),
	(1.601597, 1.739130),
	(2.371576, 2.427833),
	(1.408304, 1.943564),
	(1.945764, 2.016897),
	(1.958933, 1.957603),
	(1.484368, 1.580203),
	(1.694675, 2.309487),
	(2.108148, 2.183714),
	(2.746722, 2.812284),
	(1.829441, 2.001198),
	(2.730487, 2.843097),
	(1.990447, 2.096017),
	(1.462054, 1.495536),
	(1.276596, 1.297558),
	(2.287893, 2.291298),
	(1.789421, 1.799798),
	(1.342433, 1.429266),
	(1.513538, 1.556139),
	(2.045410, 2.098538),
	(1.532143, 1.673214),
	(1.566048, 1.596900),
	(1.700316, 1.707956),
	(1.681095, 1.727113),
	(1.554635, 1.623680),
	(1.913406, 2.024726),
	(1.698084, 1.701979),
	(1.456541, 1.540978),
	(1.942895, 2.097318),
	(2.001840, 2.069246),
	(2.180196, 2.197535),
	(2.169258, 2.242256),
	(1.596237, 1.685188),
	(2.093787, 1.992326),
	(2.093833, 2.290846),
	(1.903236, 2.109229),
	(1.762077, 1.792775),
	(2.535368, 2.640777),
	(2.037206, 2.113032),
	(1.261012, 1.900155),
	(2.041033, 2.063796),
	(1.722429, 1.806450),
	(1.845939, 1.888702),
	(2.196621, 2.230756),
	(1.593880, 1.783102),
	(2.156024, 2.315120),
	(1.587302, 1.701528),
	(1.970894, 2.018823),
	(1.978670, 2.054098),
	(1.624225, 1.680233),
	(1.896240, 1.923212),
	(2.007136, 2.086148),
	(1.644238, 1.667869),
	(1.568570, 1.625376),
	(1.674945, 1.645312),
	(2.064494, 2.154679),
	(1.541637, 1.679828),
	(2.516153, 2.543193),
	(2.060639, 2.099988),
	(1.181365, 1.308916),
	(1.852363, 1.906371),
	(1.724199, 1.789229),
	(1.776190, 2.007410),
	(1.567874, 1.749949),
	(1.597106, 2.170652),
	(2.144300, 2.630592),
	(1.727490, 1.810523),
	(1.717991, 1.822853),
	(1.779086, 2.248383),
	(1.963587, 2.021065),
	(1.719331, 1.776863),
	(1.539936, 1.549656),
	(1.429029, 1.445034),
	(1.933861, 1.976297),
	(1.791537, 2.067158),
	(2.200963, 2.391713),
	(1.720409, 1.745688),
	(1.302496, 1.352592),
	(2.149709, 2.225363),
	(2.403138, 2.449870),
	(1.561605, 1.801683),
	(1.575414, 1.578561),
	(1.843799, 1.848769),
	(2.343906, 2.400100),
	(1.943103, 1.974778),
	(1.410962, 1.522413),
	(2.011352, 2.074933),
	(2.017722, 2.317088),
	(1.349903, 1.697507),
	(2.015327, 2.072365),
	(2.314453, 2.371067),
	(1.594474, 1.726809),
	(1.781893, 1.923833),
	(1.854314, 2.267327),
	(1.791626, 1.797190),
	(1.616367, 1.622793),
	(2.227188, 2.276200),
	(1.902490, 2.161100),
	(1.457396, 1.538281),
	(1.940402, 1.975052),
	(1.519791, 1.536081),
	(1.587667, 1.636425),
	(1.207614, 1.208387),
	(3.201123, 3.639874),
	(1.543350, 1.550483),
	(0.473271, 0.474155),
	(1.760097, 1.873022),
	(1.647522, 1.666088),
	(1.530171, 1.549277),
	(1.416229, 1.416229),
	(0.754647, 0.755194),
	(1.307947, 1.308831),
	(2.563598, 2.689109),
	(1.982926, 2.284598),
	(1.649334, 1.687628),
	(1.493414, 1.579833),
	(1.913486, 2.152980),
	(1.507043, 1.510873),
	(1.910425, 1.932046),
	(1.771486, 1.808629),
	(1.232687, 1.255111),
	(1.704853, 1.712349),
	(1.786143, 1.861663),
	(1.656956, 1.707801),
	(1.356705, 1.364668),
	(1.339286, 1.365248),
	(1.990393, 2.027209),
	(1.408920, 1.413438),
	(1.339947, 1.372354),
	(1.060665, 1.062902),
	(2.131391, 2.156643),
	(1.659589, 1.665612),
	(2.032351, 2.077194),
	(2.794407, 2.825160),
	(1.033744, 1.307738),
	(1.235944, 1.287174),
	(2.635914, 2.727112),
	(2.220332, 2.262869),
	(1.921268, 1.974911),
	(2.452853, 2.133613),
	(1.156451, 1.160151),
	(2.149875, 2.149875),
	(2.811018, 2.811018),
	(1.956766, 2.128644),
	(1.871162, 1.798749),
	(1.849905, 1.914527),
	(1.897690, 2.613744),
	(1.442733, 1.703841),
	(1.840725, 1.861825),
	(1.956908, 2.857525),
	(1.515348, 1.527374),
	(2.071268, 2.187902),
	(1.541468, 1.587991),
	(2.370698, 2.820611),
	(1.933800, 1.945374),
	(1.896654, 1.963491),
	(1.513891, 1.524586),
	(1.783617, 1.854338),
	(1.849737, 1.927602),
	(1.881852, 1.970945),
	(1.261186, 1.544026),
	(1.693828, 1.890894),
	(1.854093, 2.002953),
	(1.795388, 1.797967),
	(1.831472, 1.880663),
	(1.479904, 1.771516),
	(1.829553, 1.845141),
	(2.306199, 2.335310),
	(1.570205, 1.572059),
	(2.017395, 2.016006),
	(2.114387, 2.132102),
	(1.548001, 1.608863),
	(2.145476, 2.163423),
	(1.777246, 1.791297),
	(1.980084, 1.985829),
	(1.641643, 1.788124),
	(1.153357, 1.225095),
	(1.940642, 1.940907),
	(2.547906, 2.661356),
	(2.247022, 2.258527),
	(1.788491, 1.905669),
	(1.652420, 1.658077),
	(1.772939, 2.161942),
	(1.060540, 1.077195),
	(1.272011, 1.325243),
	(1.817597, 1.895109),
	(1.162976, 1.182797),
	(1.303850, 1.308206),
	(1.402132, 1.462897),
	(1.772444, 1.777269),
	(1.299856, 1.316872),
	(1.154559, 1.176763),
	(1.515685, 1.633466),
	(2.176489, 1.524880),
	(2.346517, 2.376033),
	(1.616848, 1.791758),
	(2.355282, 2.679374),
	(1.228278, 1.238410),
	(1.331963, 1.359541),
	(2.707968, 2.755216),
	(1.504362, 1.524401),
	(2.572688, 2.839572),
	(2.304355, 2.336539),
	(1.253043, 1.301596),
	(1.051973, 1.058548),
	(1.574651, 1.581604),
	(1.377014, 1.391316),
	(3.134557, 3.232853),
	(1.171827, 1.193389),
	(1.429962, 1.446871),
	(1.036725, 1.051891),
	(2.090560, 2.210339),
	(1.575183, 1.626349),
	(0.280112, 0.280112),
	(1.340367, 1.369925),
	(1.562908, 1.575139),
	(2.092219, 2.078069),
	(1.648816, 2.107155),
	(1.772586, 1.894260),
	(2.182940, 2.238143),
	(2.058135, 2.393958),
	(1.890587, 1.867601),
	(1.989763, 1.992069),
	(1.798711, 1.802870),
	(2.101905, 2.106430),
	(1.696480, 1.812243),
	(2.283578, 2.371529),
	(2.134378, 2.179697),
	(1.879130, 1.967142),
	(1.832168, 1.876348),
	(1.623926, 1.711939),
	(1.901861, 2.263083),
	(2.031832, 2.102724),
	(1.849870, 2.071222),
	(1.392554, 1.547514),
	(1.998750, 2.044422),
	(1.893699, 1.969349),
	(2.050183, 2.252188),
	(2.206681, 2.284764),
	(2.393489, 2.489517),
	(1.480314, 1.503657),
	(1.717090, 1.730161),
	(2.346541, 2.635364),
	(1.983615, 2.124936),
	(1.849063, 1.969784),
	(1.909869, 1.924919),
	(1.213367, 1.229835),
	(2.505967, 2.561979),
	(2.118378, 2.122305),
	(2.313152, 2.352855),
	(2.763797, 2.793730),
	(1.965494, 1.965967),
	(1.878462, 2.011704),
	(1.227904, 1.315237),
	(2.201716, 2.527642),
	(1.466313, 1.495667),
	(1.630454, 1.674587),
	(1.325795, 1.342955),
	(1.372929, 1.439132),
	(1.521653, 1.523548),
	(1.567786, 1.654166),
	(0.911701, 0.916913),
	(1.545672, 1.668971),
	(1.178899, 1.209515),
	(1.384740, 1.390191),
	(2.085506, 2.193040),
	(1.514841, 1.536779),
	(1.409673, 1.424210),
	(1.991162, 2.009253),
	(2.368085, 2.385389),
	(1.039881, 1.041705),
	(1.606614, 1.494771),
	(2.290779, 2.300907),
	(2.835160, 2.980770),
	(0.926814, 1.214446),
	(2.062145, 2.157710),
	(1.673108, 1.682132),
	(1.736780, 1.787293),
	(2.797087, 2.799451),
	(0.906042, 0.937365),
	(2.043012, 2.093888),
	(2.174230, 2.183958),
	(1.329249, 1.367908),
	(1.560969, 1.596560),
	(1.937046, 1.957596),
	(1.853702, 1.862027),
	(2.772275, 2.789159),
	(2.203896, 2.313731),
	(0.902256, 0.902256),
	(1.724138, 1.736453),
	(1.265408, 1.364948),
	(1.450768, 1.556410),
	(1.547633, 1.562474),
	(1.501233, 1.510074),
	(1.962608, 1.979731),
	(2.138215, 2.183313),
	(2.456187, 2.615507),
	(1.372267, 1.421625),
	(2.397984, 2.444485),
	(1.687873, 1.717086),
	(1.774753, 1.785043),
	(1.343794, 1.342380),
	(1.213325, 1.226423),
	(1.493343, 1.505712),
	(3.932592, 3.935914),
	(1.878264, 1.886057),
	(2.006544, 2.020036),
	(2.379380, 2.605763),
	(2.490602, 2.514098),
	(1.676300, 1.993174),
	(1.758741, 1.783338),
	(1.147471, 1.241576),
	(1.436561, 1.455996),
	(2.159021, 2.266780),
	(0.867188, 0.873373),
	(1.213495, 1.235051),
	(2.406319, 2.465414),
	(2.110107, 2.128992),
	(2.555130, 2.637773),
	(2.310924, 2.335363),
	(1.183844, 1.186956),
	(1.380617, 1.442494),
	(1.467927, 1.510831),
	(1.304580, 1.312175),
	(1.697260, 1.709505),
	(1.291531, 1.299640),
	(0.976240, 0.992400),
	(3.159460, 3.202825),
	(1.902555, 1.950448),
	(2.387880, 2.390703),
	(2.247805, 2.255786),
	(1.116554, 1.119892),
	(2.572406, 2.691556),
	(1.610759, 1.614669),
	(1.362183, 1.365348),
	(1.597926, 1.610770),
	(1.575701, 1.674147),
	(1.478795, 1.702009),
	(1.460925, 1.476341),
	(2.391881, 2.564403),
	(1.361958, 1.378588),
	(1.120691, 1.121476),
	(1.467319, 1.489097),
	(4.672151, 4.683293),
	(1.528951, 1.566165),
	(1.554970, 1.555348),
	(1.894891, 1.908819),
	(2.311082, 2.378093),
	(1.157387, 1.192900),
	(1.479758, 1.494070),
	(1.175961, 1.188644),
	(1.339698, 1.371453),
	(2.457955, 2.466654),
	(1.236044, 1.244483),
	(2.000414, 2.004557),
	(1.373242, 1.376854),
	(0.760425, 0.769216),
	(1.036132, 1.061715),
	(1.341753, 1.348672),
	(1.390107, 1.428627),
	(0.769962, 0.785401),
	(1.208047, 1.212964),
	(1.257759, 1.273444),
	(0.830060, 0.832470),
	(1.299411, 1.302211),
	(1.110715, 1.134739),
	(1.780798, 1.786417),
	(3.143389, 3.161387),
	(1.089676, 1.099353),
	(1.451773, 1.470415),
	(1.375346, 1.377475),
	(1.254443, 1.277873),
	(1.633536, 1.650159),
	(0.977969, 0.979379),
	(1.153417, 1.159999),
	(0.971210, 0.982292),
	(1.213915, 1.243570),
	(2.612147, 2.648283),
	(2.784731, 2.789201),
	(0.919622, 0.937811),
	(1.261426, 1.263335),
	(1.224602, 1.226493),
	(1.034082, 1.110657),
	(0.826802, 0.839413),
	(2.589468, 3.042867),
	(1.793877, 1.811448),
	(1.374330, 1.389331),
	(2.666914, 2.759495),
	(1.318575, 1.335390),
	(1.103767, 1.113074),
	(0.976598, 0.982039),
	(0.786973, 0.790781),
	(1.042000, 1.203548),
	(1.263201, 1.265788),
	(1.702189, 1.795092),
	(0.857639, 0.895446),
	(1.024628, 1.053475),
	(2.147729, 2.278884),
	(2.281822, 2.287931),
	(1.513813, 1.589207),
	(2.186309, 2.201968),
	(2.230842, 2.267998),
	(1.651383, 1.661558),
	(4.372778, 4.431567),
	(1.423293, 1.431956),
	(1.306790, 1.308565),
	(0.870040, 0.878465),
	(1.314863, 1.355689),
	(1.852380, 1.854164),
	(1.225205, 1.262126),
	(1.118317, 1.131213),
	(1.708601, 1.721521),
	(1.589145, 1.765181),
	(2.126998, 2.380434),
	(2.914077, 3.078018),
	(1.541670, 1.548534),
	(1.110520, 1.117626),
	(2.771855, 2.523099),
} \fill \pos circle(0.03);

\draw (0,0) -- (5, 5);

\end{tikzpicture}

\end{frame}


\section{Difficulties}
\subsection{to find a combined algorithm always at least as good as both compositions}

\begin{frame}{Irregular Units}
  \begin{center}
\begin{tikzpicture}[node distance=1.3cm]
  \proofnode[color=addcolor]{luroot}{\only<4>{$\bot$}};

  \proofnode[above left of=luroot]{root}{\alt<3-4>{$\dual{a}$}{$\bot$}};
  \withchildren{root} {r2}{\alt<3-4>{$\dual{a} c$}{$c$}} {a4}{$\dual{c}$};
  \withchildren{r2} {r0}{\alt<8>{$b c$}{$b$}} {r1}{$\dual{b} c$};

  \addchildren{r0} {a0}{$\dual{a} b$} {iu}{$a$};
  \draw [proof edge] (r0) -- (a0);
  \only<1,5->{\draw [proof edge] (r0) -- (iu);}
  \only<2>{\draw [deleted edge] (r0) -- (iu);}

  \addchildren{iu} {a1}{$a c$} {a2}{$a \dual{c}$};
  \draw [proof edge] (iu) -- (a1);
  \only<1-6>{\draw [proof edge] (iu) -- (a2);}
  \only<7>{\draw [deleted edge] (iu) -- (a2);}

  \proofnode[above right of=r1]{a3}{$\dual{a} \dual{b} c$};
  \draw [proof edge] (r1) -- (a3);
  \only<1,5->{\draw [proof edge] (r1) -- (iu);}
  \only<2>{\draw [deleted edge] (r1) -- (iu);}

  \crossnode<3-4>{r0}
  \crossnode<3-4>{r1}
  \only<4>{
    \draw [proof edge,color=addcolor] (luroot) -- (root);
    \draw [proof edge,color=addcolor] (luroot) .. controls ++(1.2,1.5) .. (iu);
  }

  \pivot<6>{iu}{$c$}
  \pivot<6>{root}{$c$}
  \safelit<7>{root}{}
  \safelit<7>{r2}{c}
  \safelit<7>{r0}{b,c}
  \safelit<7>{r1}{\dual{b},c}
  \safelit<7>{iu}{a,c}
  \crossnode<8>{a2}
  \crossnode<8>{iu}

  \draw (luroot) ++(1.2,-0.5) node {};
\end{tikzpicture}
\end{center}
\end{frame}

\begin{frame}{Root and Units Safe Literals}
\begin{subpart}{In RPI.LU, after LU :}
  \item the proof is of the form $\eta \odot_{a_0} \eta_0 \odot_{a_1} \cdots \odot_{a_n} \eta_n$ ;
  \item $\set{\dual{a}_i}{i \leq n}$ is the safe literals for $\eta$ : \emph{the root's safe literals} ;
  \item $\forall i < n$, $\set{\dual{a}_j}{i < j \leq n}$ is the safe literals for $\eta_i$.
\end{subpart}
\begin{overlayarea}{\textwidth}{2em}
  \singleline<2->{\asGoodRPILU}
\end{overlayarea}
\end{frame}

\begin{frame}{Units introduced by RPI}
\begin{center}
\vspace{-3em}
\begin{tikzpicture}[node distance=1.3cm,scale=0.93]
  \rootnode;
  \addchildren{root} {n0}{\alt<3->{$\bot$}{$d$}} {a0}{$\dual{d}$};
  \draw<-2> [proof edge] (root) -- (n0);
  \draw     [proof edge] (root) -- (a0);
  \withchildren{n0} {n1}{\alt<3->{$e$}{$d e$}} {n2}{\alt<3->{$\dual{e}$}{$d \dual{e}$}};

  \withchildren{n1} {a1}{$\dual{b} e$} {u0}{\alt<3->{\alert<3>{$b$}}{$d b$}};
  \proofnode[above right of=n2]{a2}{$\dual{b} \dual{e}$};
  \drawchildren{n2} {u0} {a2};

  \withchildren{u0} {n3}{$a b d$} {a3}{$\dual{a} b$};
  \addchildren{n3} {n4}{$b d c$} {a4}{$\dual{c} a$};
  \draw    [proof edge]   (n3) -- (n4);
  \draw<-2>[proof edge]   (n3) -- (a4);
  \addchildren{n4} {a5}{$a b$}   {a6}{$\dual{a} d c$};
  \draw    [proof edge]   (n4) -- (a5);
  \draw<1> [proof edge]   (n4) -- (a6);
  \draw<2> [deleted edge] (n4) -- (a6);

  \pivot<2>{u0} {$a$}
  \pivot<2>{n4} {$a$}
  \crossnode<3->{a6}

  \crossnode<3->{n4}
  \crossnode<3->{n3}
  \crossnode<3->{a4}
  \crossnode<3->{root}
  \crossnode<3->{a0}
\end{tikzpicture}
\vspace{-2.2em}
\end{center}
\begin{overlayarea}{\textwidth}{1em}
  \singleline<4>{\asGoodLURPI}
\end{overlayarea}
\end{frame}


\section{Combined Equivalents}
\subsection{to both sequential compositions}

\begin{frame}{RPI[3]LU}
\singleline{\asGoodRPILU}
\begin{subpart}{Three traversals}
  \topdown  collect units and compute root and units safe literals ;
  \bottomup compute safe literals and mark edges to be deleted ;
  \topdown  fix the proof and reintroduce units.
\end{subpart}
\end{frame}

\begin{frame}{RPI[3]LU vs RPI.LU}
\centering
\only<+>{\charttitle{\compC} \begin{tikzpicture}

\draw (0,0) -- (5,0);
\node at (2.5,-0.6) {RPI.LU};
\node [anchor=north] at (0.714285714285714,0) {\tiny 0.1};
\draw (0.714285714285714,0) -- (0.714285714285714,0.1);
\draw [style=help lines] (0.714285714285714,0) -- (0.714285714285714,5);
\node [anchor=north] at (1.42857142857143,0) {\tiny 0.2};
\draw (1.42857142857143,0) -- (1.42857142857143,0.1);
\draw [style=help lines] (1.42857142857143,0) -- (1.42857142857143,5);
\node [anchor=north] at (2.14285714285714,0) {\tiny 0.3};
\draw (2.14285714285714,0) -- (2.14285714285714,0.1);
\draw [style=help lines] (2.14285714285714,0) -- (2.14285714285714,5);
\node [anchor=north] at (2.85714285714286,0) {\tiny 0.4};
\draw (2.85714285714286,0) -- (2.85714285714286,0.1);
\draw [style=help lines] (2.85714285714286,0) -- (2.85714285714286,5);
\node [anchor=north] at (3.57142857142857,0) {\tiny 0.5};
\draw (3.57142857142857,0) -- (3.57142857142857,0.1);
\draw [style=help lines] (3.57142857142857,0) -- (3.57142857142857,5);
\node [anchor=north] at (4.28571428571429,0) {\tiny 0.6};
\draw (4.28571428571429,0) -- (4.28571428571429,0.1);
\draw [style=help lines] (4.28571428571429,0) -- (4.28571428571429,5);
\node [anchor=north] at (5,0) {\tiny 0.7};
\draw (5,0) -- (5,0.1);
\draw [style=help lines] (5,0) -- (5,5);
\draw (0,0) -- (0,5);
\node [rotate=90] at (-2.5em,2.5) {RPI[3]LU};
\node [anchor=east] at (0,0.713285714285714) {\tiny 0.1};
\draw (0,0.713285714285714) -- (0.1,0.713285714285714);
\draw [style=help lines] (0,0.713285714285714) -- (5,0.713285714285714);
\node [anchor=east] at (0,1.42657142857143) {\tiny 0.2};
\draw (0,1.42657142857143) -- (0.1,1.42657142857143);
\draw [style=help lines] (0,1.42657142857143) -- (5,1.42657142857143);
\node [anchor=east] at (0,2.13985714285714) {\tiny 0.3};
\draw (0,2.13985714285714) -- (0.1,2.13985714285714);
\draw [style=help lines] (0,2.13985714285714) -- (5,2.13985714285714);
\node [anchor=east] at (0,2.85314285714286) {\tiny 0.4};
\draw (0,2.85314285714286) -- (0.1,2.85314285714286);
\draw [style=help lines] (0,2.85314285714286) -- (5,2.85314285714286);
\node [anchor=east] at (0,3.56642857142857) {\tiny 0.5};
\draw (0,3.56642857142857) -- (0.1,3.56642857142857);
\draw [style=help lines] (0,3.56642857142857) -- (5,3.56642857142857);
\node [anchor=east] at (0,4.27971428571429) {\tiny 0.6};
\draw (0,4.27971428571429) -- (0.1,4.27971428571429);
\draw [style=help lines] (0,4.27971428571429) -- (5,4.27971428571429);
\node [anchor=east] at (0,4.993) {\tiny 0.7};
\draw (0,4.993) -- (0.1,4.993);
\draw [style=help lines] (0,4.993) -- (5,4.993);
\foreach \pos in {
	(1.556186, 1.556186),
	(2.342277, 2.342277),
	(2.062508, 2.062508),
	(1.158038, 1.158038),
	(1.201201, 1.201201),
	(0.484526, 0.484526),
	(1.513678, 1.501520),
	(0.250054, 0.250054),
	(2.046426, 2.046426),
	(1.732451, 1.732451),
	(1.420891, 1.420891),
	(1.802097, 1.802097),
	(1.165638, 1.165638),
	(2.242647, 2.242647),
	(1.604737, 1.604737),
	(0.908736, 0.908736),
	(1.756547, 1.756547),
	(2.026342, 2.026342),
	(1.684679, 1.684679),
	(3.246073, 3.246073),
	(2.115927, 2.115927),
	(1.900922, 1.878766),
	(1.926436, 1.926436),
	(2.245221, 2.245221),
	(2.002058, 2.002058),
	(1.861827, 1.861827),
	(1.474026, 1.474026),
	(2.173218, 2.173218),
	(1.310559, 1.310559),
	(1.903352, 1.903352),
	(1.729752, 1.729752),
	(2.151067, 2.151067),
	(1.794242, 1.794242),
	(1.410488, 1.410488),
	(1.755952, 1.755952),
	(2.075078, 2.075078),
	(1.827776, 1.827776),
	(1.560806, 1.560806),
	(1.321892, 1.321892),
	(1.634847, 1.634847),
	(1.104901, 1.104901),
	(2.208315, 2.208315),
	(2.100457, 2.100457),
	(1.450054, 1.450054),
	(1.210813, 1.210813),
	(1.860465, 1.860465),
	(2.167394, 2.167394),
	(1.444692, 1.444692),
	(2.032844, 2.032844),
	(1.464543, 1.464543),
	(1.627940, 1.627940),
	(1.634645, 1.634645),
	(2.307409, 2.307409),
	(1.613466, 1.613466),
	(1.585714, 1.585714),
	(1.971680, 1.971680),
	(1.636142, 1.636142),
	(0.134983, 0.134983),
	(0.474354, 0.474354),
	(2.236025, 2.236025),
	(1.920316, 1.920316),
	(1.811800, 1.811800),
	(1.605944, 1.605944),
	(1.727484, 1.727484),
	(1.250955, 1.250955),
	(1.800115, 1.800115),
	(1.463325, 1.463325),
	(0.963880, 0.963880),
	(1.530612, 1.530612),
	(1.560178, 1.560178),
	(1.867044, 1.867044),
	(1.997579, 1.997579),
	(1.713975, 1.713975),
	(1.710076, 1.710076),
	(1.577143, 1.577143),
	(1.464286, 1.464286),
	(1.575964, 1.575964),
	(1.887304, 1.887304),
	(1.822739, 1.822739),
	(1.227863, 1.227863),
	(1.194162, 1.194162),
	(1.755562, 1.755562),
	(1.813616, 1.813616),
	(1.925841, 1.925841),
	(2.537594, 2.537594),
	(1.628264, 1.628264),
	(1.742994, 1.742994),
	(1.916376, 1.916376),
	(2.097506, 2.097506),
	(1.596841, 1.596841),
	(0.267681, 0.267681),
	(1.926164, 1.926164),
	(0.287698, 0.287698),
	(0.287698, 0.287698),
	(1.273198, 1.273198),
	(1.071429, 1.071429),
	(1.916980, 1.916980),
	(0.853528, 0.853528),
	(0.789474, 0.789474),
	(0.875576, 0.875576),
	(0.115830, 0.115830),
	(1.392857, 1.392857),
	(1.344086, 1.344086),
	(1.344086, 1.344086),
	(0.418443, 0.418443),
	(1.655629, 1.655629),
	(0.324863, 0.324863),
	(0.127767, 0.127767),
	(0.129112, 0.129112),
	(0.548326, 0.548326),
	(0.410277, 0.410277),
	(1.099314, 1.099314),
	(0.294922, 0.294922),
	(1.066836, 1.066836),
	(1.355370, 1.355370),
	(3.368314, 3.368314),
	(0.527024, 0.527024),
	(0.708075, 0.708075),
	(1.409774, 1.409774),
	(1.235231, 1.235231),
	(0.253593, 0.253593),
	(1.751152, 1.751152),
	(1.927438, 1.927438),
	(0.396825, 0.396825),
	(0.751880, 0.751880),
	(0.480769, 0.480769),
	(0.772201, 0.772201),
	(1.749271, 1.749271),
	(1.512605, 1.512605),
	(1.632653, 1.632653),
	(0.000000, 0.000000),
	(0.148810, 0.148810),
	(1.020408, 1.020408),
	(1.219512, 1.219512),
	(0.744048, 0.744048),
	(0.533049, 0.533049),
	(1.166181, 1.166181),
	(1.522248, 1.522248),
	(0.840336, 0.840336),
	(0.000000, 0.000000),
	(2.142857, 2.142857),
	(2.678571, 2.678571),
	(4.395604, 4.395604),
	(1.899454, 1.899454),
	(1.864130, 1.864130),
	(1.949833, 1.949833),
	(1.371096, 1.371096),
	(1.728143, 1.728143),
	(1.692539, 1.692539),
	(2.120792, 2.120792),
	(1.388778, 1.388778),
	(1.547748, 1.547748),
	(1.940670, 1.940670),
	(1.836503, 1.836503),
	(1.792318, 1.792318),
	(1.542395, 1.542395),
	(1.455830, 1.455830),
	(1.283614, 1.283614),
	(2.816901, 2.816901),
	(1.563488, 1.563488),
	(2.038230, 2.038230),
	(1.884921, 1.884921),
	(1.701979, 1.701979),
	(1.538110, 1.538110),
	(1.709805, 1.709805),
	(2.051342, 2.051342),
	(1.078029, 1.078029),
	(1.837379, 1.837379),
	(2.205523, 2.205523),
	(2.646262, 2.646262),
	(1.651504, 1.651504),
	(2.134208, 2.134208),
	(1.537839, 1.537839),
	(2.164359, 2.164359),
	(2.105427, 2.105427),
	(1.607786, 1.607786),
	(2.386194, 2.386194),
	(1.597045, 1.597045),
	(1.853480, 1.853480),
	(1.591711, 1.591711),
	(2.340862, 2.340862),
	(2.134493, 2.134493),
	(2.809749, 2.809749),
	(1.713848, 1.713848),
	(2.082282, 2.082282),
	(1.791912, 1.791912),
	(1.806559, 1.806559),
	(2.099460, 2.099460),
	(2.108006, 2.108006),
	(2.425876, 2.416891),
	(1.890005, 1.890005),
	(1.791444, 1.791444),
	(1.802866, 1.802866),
	(1.985603, 1.985603),
	(1.480685, 1.480685),
	(1.984127, 1.984127),
	(1.470034, 1.470034),
	(2.335578, 2.335578),
	(1.230291, 1.230291),
	(1.419248, 1.419248),
	(1.936585, 1.936585),
	(2.502839, 2.496025),
	(1.539613, 1.539613),
	(1.813564, 1.813564),
	(2.029221, 2.029221),
	(1.936119, 1.936119),
	(1.417666, 1.417666),
	(2.261445, 2.261445),
	(2.140845, 2.100604),
	(2.305835, 2.305835),
	(1.443093, 1.443093),
	(1.378106, 1.378106),
	(1.661706, 1.661706),
	(2.477596, 2.477596),
	(1.988277, 1.988277),
	(1.901705, 1.901705),
	(0.114527, 0.114527),
	(1.597744, 1.597744),
	(1.139351, 1.139351),
	(1.464714, 1.464714),
	(1.900439, 1.900439),
	(1.493321, 1.493321),
	(1.793674, 1.793674),
	(1.575092, 1.575092),
	(2.306489, 2.292528),
	(1.675705, 1.675705),
	(1.812061, 1.812061),
	(1.163693, 1.163693),
	(1.842842, 1.842842),
	(1.508977, 1.508977),
	(1.293851, 1.293851),
	(2.131964, 2.131964),
	(1.850706, 1.850706),
	(2.275112, 2.275112),
	(1.873536, 1.873536),
	(1.391543, 1.391543),
	(2.175203, 2.175203),
	(1.808905, 1.808905),
	(1.627785, 1.627785),
	(1.810678, 1.810678),
	(2.228958, 2.228958),
	(2.037667, 2.037667),
	(1.440390, 1.440390),
	(2.481447, 2.481447),
	(1.814443, 1.814443),
	(1.464624, 1.464624),
	(1.566475, 1.566475),
	(1.467283, 1.467283),
	(1.575290, 1.575290),
	(1.564395, 1.564395),
	(1.684185, 1.684185),
	(1.772847, 1.772847),
	(2.188168, 2.188168),
	(1.514716, 1.514716),
	(1.634286, 1.634286),
	(1.539212, 1.539212),
	(1.710577, 1.710577),
	(2.599366, 2.599366),
	(1.196172, 1.196172),
	(1.784863, 1.784863),
	(1.243339, 1.243339),
	(2.161272, 2.161272),
	(1.897547, 1.897547),
	(1.215596, 1.215596),
	(1.647395, 1.647395),
	(1.080923, 1.080923),
	(1.507135, 1.507135),
	(1.672037, 1.672037),
	(2.356441, 2.356441),
	(1.515152, 1.515152),
	(1.576542, 1.576542),
	(1.782247, 1.782247),
	(1.319261, 1.319261),
	(2.336977, 2.336977),
	(2.977877, 2.977877),
	(2.221519, 2.221519),
	(3.409759, 3.409759),
	(2.470830, 2.470830),
	(2.705456, 2.705456),
	(1.863602, 1.863602),
	(2.076169, 2.076169),
	(1.622987, 1.622987),
	(1.339508, 1.339508),
	(2.223823, 2.223823),
	(1.881462, 1.881462),
	(2.241101, 2.241101),
	(2.145121, 2.145121),
	(2.344689, 2.344689),
	(1.728967, 1.728967),
	(1.440073, 1.440073),
	(1.950282, 1.950282),
	(1.677489, 1.677489),
	(2.165722, 2.165722),
	(1.818885, 1.818885),
	(1.643512, 1.643512),
	(1.808967, 1.808967),
	(2.429586, 2.429586),
	(1.480135, 1.480135),
	(1.488676, 1.488676),
	(2.212590, 2.212590),
	(1.917962, 1.917962),
	(1.713996, 1.713996),
	(1.821737, 1.821737),
	(1.864078, 1.864078),
	(2.105590, 2.105590),
	(1.845992, 1.845992),
	(1.791954, 1.791954),
	(2.164018, 2.164018),
	(2.209045, 2.209045),
	(1.887657, 1.887657),
	(2.363503, 2.363503),
	(2.415675, 2.415675),
	(1.932172, 1.932172),
	(2.665723, 2.665723),
	(1.676057, 1.676057),
	(2.188768, 2.188768),
	(1.536838, 1.536838),
	(1.957185, 1.957185),
	(1.660664, 1.660664),
	(1.376147, 1.376147),
	(1.712625, 1.710821),
	(1.238858, 1.238858),
	(1.759919, 1.759919),
	(2.732655, 2.732655),
	(1.858000, 1.858000),
	(1.852295, 1.849303),
	(1.351542, 1.351542),
	(1.946406, 1.946406),
	(2.196001, 2.196001),
	(2.862425, 2.862425),
	(1.606158, 1.606158),
	(1.491578, 1.491578),
	(2.131835, 2.102264),
	(1.905561, 1.905561),
	(2.046101, 2.046101),
	(1.622336, 1.622336),
	(1.670875, 1.670875),
	(1.586752, 1.586752),
	(1.833707, 1.833707),
	(1.684428, 1.684428),
	(1.872033, 1.872033),
	(1.865691, 1.865691),
	(2.477291, 2.477291),
	(1.874278, 1.874278),
	(1.881448, 1.881448),
	(2.445129, 2.445129),
	(2.079095, 2.079095),
	(2.028380, 2.028380),
	(1.545866, 1.545866),
	(1.130524, 1.130524),
	(2.212466, 2.212466),
	(1.532533, 1.532533),
	(1.254010, 1.254010),
	(2.289628, 2.289628),
	(2.310611, 2.310611),
	(1.470293, 1.470293),
	(1.451187, 1.451187),
	(1.832159, 1.832159),
	(1.624933, 1.624933),
	(1.717108, 1.717108),
	(2.489140, 2.489140),
	(1.544402, 1.544402),
	(1.535993, 1.535993),
	(1.632923, 1.632923),
	(2.217625, 2.217625),
	(1.818248, 1.818248),
	(1.379791, 1.379791),
	(1.559731, 1.559731),
	(1.929535, 1.929535),
	(1.625194, 1.625194),
	(1.499697, 1.499697),
	(2.098422, 2.098422),
	(2.179985, 2.179985),
	(1.897464, 1.897464),
	(1.793890, 1.793890),
	(1.396966, 1.396966),
	(1.402764, 1.402764),
	(1.817193, 1.817193),
	(1.866989, 1.866989),
	(1.059639, 1.059639),
	(1.670467, 1.670467),
	(1.550016, 1.550016),
	(1.701670, 1.701670),
	(1.872202, 1.872202),
	(1.523229, 1.523229),
	(1.522930, 1.522930),
	(1.515391, 1.515391),
	(1.661519, 1.661519),
	(1.470588, 1.470588),
	(1.564801, 1.564801),
	(1.803533, 1.803533),
	(2.371403, 2.371403),
	(1.582843, 1.582843),
	(1.897444, 1.897444),
	(1.905234, 1.905234),
	(2.333186, 2.333186),
	(2.209965, 2.209965),
	(2.622768, 2.622768),
	(2.445758, 2.445758),
	(1.887060, 1.887060),
	(2.569201, 2.569201),
	(1.439037, 1.439037),
	(2.527142, 2.527142),
	(1.447333, 1.447333),
	(2.304508, 2.304508),
	(1.941144, 1.941144),
	(1.648141, 1.648141),
	(1.625157, 1.625157),
	(2.363271, 2.361441),
	(1.709126, 1.709126),
	(1.896536, 1.896536),
	(1.947713, 1.947713),
	(1.127820, 1.127820),
	(2.113921, 2.113921),
	(2.351874, 2.351874),
	(1.873678, 1.873678),
	(1.790174, 1.790174),
	(1.493174, 1.493174),
	(2.162884, 2.162884),
	(1.712924, 1.712924),
	(1.816153, 1.816153),
	(2.037065, 2.037065),
	(1.609474, 1.609474),
	(1.665304, 1.665304),
	(1.858253, 1.858253),
	(1.797122, 1.797122),
	(2.178374, 2.178374),
	(1.962605, 1.962605),
	(1.762656, 1.762656),
	(2.309133, 2.309133),
	(1.590930, 1.590930),
	(1.711998, 1.711998),
	(2.885598, 2.885598),
	(2.389674, 2.389674),
	(1.867791, 1.867791),
	(1.696880, 1.696880),
	(1.486818, 1.486818),
	(1.236996, 1.236996),
	(1.270211, 1.270211),
	(1.942054, 1.942054),
	(2.009906, 2.009906),
	(1.527108, 1.527108),
	(2.515148, 2.515148),
	(1.966366, 1.966366),
	(1.800064, 1.800064),
	(2.189574, 2.189574),
	(2.453800, 2.453800),
	(2.668923, 2.668923),
	(2.252845, 2.252845),
	(1.747366, 1.747366),
	(1.933306, 1.933306),
	(1.721575, 1.721575),
	(1.824308, 1.824308),
	(1.504327, 1.504327),
	(1.489207, 1.489207),
	(2.087097, 2.087097),
	(1.744720, 1.744720),
	(2.158680, 2.137960),
	(1.848541, 1.848541),
	(1.948437, 1.948437),
	(1.702289, 1.702289),
	(2.431524, 2.431524),
	(2.012757, 2.012757),
	(1.092772, 1.092772),
	(1.537204, 1.537204),
	(2.669490, 2.669490),
	(2.038328, 2.038328),
	(1.916784, 1.916784),
	(1.656275, 1.656275),
	(1.431813, 1.431813),
	(2.022303, 2.022303),
	(2.304147, 2.304147),
	(1.877227, 1.877227),
	(2.086301, 2.086301),
	(1.705403, 1.705403),
	(1.584144, 1.584144),
	(2.043359, 2.043359),
	(1.920720, 1.920720),
	(2.121568, 2.088533),
	(1.580776, 1.580776),
	(1.296918, 1.296918),
	(1.036455, 1.036455),
	(1.517222, 1.517222),
	(1.839086, 1.839086),
	(2.069906, 2.069906),
	(2.016994, 2.016994),
	(2.193807, 2.193807),
	(1.840426, 1.840426),
	(2.141729, 2.141729),
	(1.692849, 1.692849),
	(2.549709, 2.549709),
	(1.808566, 1.808566),
	(2.265094, 2.265094),
	(1.876868, 1.876868),
	(2.092923, 2.077319),
	(1.830481, 1.830481),
	(1.802692, 1.802692),
	(1.634327, 1.634327),
	(1.646809, 1.646809),
	(1.965484, 1.965484),
	(2.463927, 2.463927),
	(1.505792, 1.505792),
	(1.538605, 1.538605),
	(1.846438, 1.846438),
	(2.120515, 2.107295),
	(2.192524, 2.192524),
	(1.853791, 1.853791),
	(1.727733, 1.727733),
	(1.684760, 1.684760),
	(1.857197, 1.557649),
	(2.050429, 2.050429),
	(1.879586, 1.879586),
	(1.593780, 1.593780),
	(1.350013, 1.350013),
	(1.690364, 1.690364),
	(1.270616, 1.270616),
	(1.647164, 1.647164),
	(1.942480, 1.942480),
	(2.697350, 2.697350),
	(1.620315, 1.620315),
	(1.568565, 1.568565),
	(2.145743, 2.145743),
	(1.580278, 1.580278),
	(1.889535, 1.889535),
	(1.533717, 1.533717),
	(2.030430, 2.018017),
	(2.088706, 2.088706),
	(2.236386, 2.236386),
	(1.643182, 1.643182),
	(1.899551, 1.899551),
	(1.353635, 1.353635),
	(2.860858, 2.860858),
	(1.817758, 1.817758),
	(1.054300, 1.054300),
	(1.851307, 1.851307),
	(1.306003, 1.306003),
	(1.655820, 1.655820),
	(2.948944, 2.948944),
	(2.031376, 2.031376),
	(2.067543, 2.067543),
	(1.574539, 1.574539),
	(2.380525, 2.380525),
	(1.784897, 1.784897),
	(1.541409, 1.541409),
	(1.560458, 1.560458),
	(1.714617, 1.714617),
	(1.602625, 1.602625),
	(2.051117, 2.051117),
	(1.638216, 1.638216),
	(2.174653, 2.174653),
	(1.952237, 1.952237),
	(1.880735, 1.880735),
	(1.793473, 1.793473),
	(1.856492, 1.856492),
	(2.355136, 2.355136),
	(2.398061, 2.386655),
	(1.915977, 1.915977),
	(1.636651, 1.636651),
	(1.716863, 1.716863),
	(1.648855, 1.648855),
	(1.847009, 1.847009),
	(3.156938, 3.156938),
	(1.503245, 1.495426),
	(2.312902, 2.312902),
	(1.869565, 1.869565),
	(1.559755, 1.551946),
	(1.684088, 1.684088),
	(2.044604, 2.044604),
	(2.435065, 2.435065),
	(1.755080, 1.755080),
	(2.051832, 2.051832),
	(1.367319, 1.367319),
	(1.571632, 1.571632),
	(2.047952, 2.047952),
	(1.993499, 1.993499),
	(1.494848, 1.494848),
	(2.095187, 2.095187),
	(1.719716, 1.719716),
	(2.009885, 2.009885),
	(1.972334, 1.972334),
	(2.358564, 2.358564),
	(1.341118, 1.341118),
	(2.415721, 2.415721),
	(1.460222, 1.460222),
	(1.878661, 1.878661),
	(1.478132, 1.478132),
	(1.281505, 1.281505),
	(1.898290, 1.898290),
	(2.226366, 2.226366),
	(1.787500, 1.784643),
	(1.764136, 1.764136),
	(1.618669, 1.618669),
	(2.132176, 2.132176),
	(2.413085, 2.392541),
	(1.427717, 1.427717),
	(1.153696, 1.153696),
	(2.352079, 2.352079),
	(2.135549, 2.130201),
	(1.767048, 1.767048),
	(1.376756, 1.376756),
	(2.251245, 2.251245),
	(2.447279, 2.447279),
	(1.703826, 1.703826),
	(2.988029, 2.988029),
	(1.680556, 1.680556),
	(2.053514, 2.053514),
	(1.944159, 1.944159),
	(1.685526, 1.685526),
	(1.918436, 1.918436),
	(1.504724, 1.504724),
	(1.688312, 1.688312),
	(1.532333, 1.532333),
	(1.887458, 1.887458),
	(1.673678, 1.653100),
	(3.155542, 3.155542),
	(2.013225, 2.013225),
	(2.537594, 2.537594),
	(3.314720, 3.314720),
	(1.555688, 1.555688),
	(1.766122, 1.766122),
	(1.894530, 1.894530),
	(1.620312, 1.620312),
	(2.276268, 2.276268),
	(1.740139, 1.740139),
	(2.095135, 2.095135),
	(1.819516, 1.819516),
	(1.617751, 1.617751),
	(1.933016, 1.933016),
	(1.543063, 1.543063),
	(2.616819, 2.616819),
	(1.893735, 1.893735),
	(1.651761, 1.651761),
	(1.772530, 1.772530),
	(1.785714, 1.785714),
	(2.292769, 2.292769),
	(1.950051, 1.950051),
	(1.919381, 1.919381),
	(2.271259, 2.271259),
	(1.845297, 1.845297),
	(1.334180, 1.334180),
	(1.643535, 1.643535),
	(1.791999, 1.791999),
	(1.581514, 1.581514),
	(1.556890, 1.556890),
	(1.967870, 1.950194),
	(1.771966, 1.771966),
	(2.242031, 2.242031),
	(2.130858, 2.112065),
	(1.778082, 1.770681),
	(2.400053, 2.397188),
	(1.985402, 1.985402),
	(1.925922, 1.925922),
	(2.131358, 2.131358),
	(1.739130, 1.739130),
	(2.427833, 2.427833),
	(1.943564, 1.938367),
	(2.016897, 2.016897),
	(1.957603, 1.961593),
	(1.580203, 1.580203),
	(2.309487, 2.309487),
	(2.183714, 2.183714),
	(2.812284, 2.812284),
	(2.001198, 2.001198),
	(2.843097, 2.841886),
	(2.096017, 2.096017),
	(1.495536, 1.495536),
	(1.297558, 1.297558),
	(2.291298, 2.291298),
	(1.799798, 1.799798),
	(1.429266, 1.429266),
	(1.556139, 1.556139),
	(2.098538, 2.088751),
	(1.673214, 1.673214),
	(1.596900, 1.596900),
	(1.707956, 1.707956),
	(1.727113, 1.727113),
	(1.623680, 1.623680),
	(2.024726, 2.024726),
	(1.701979, 1.701979),
	(1.540978, 1.540978),
	(2.097318, 2.097318),
	(2.069246, 2.069246),
	(2.197535, 2.197535),
	(2.242256, 2.242256),
	(1.685188, 1.668129),
	(1.992326, 1.992326),
	(2.290846, 2.290846),
	(2.109229, 2.109229),
	(1.792775, 1.792775),
	(2.640777, 2.640777),
	(2.113032, 2.113032),
	(1.900155, 1.900155),
	(2.063796, 2.063796),
	(1.806450, 1.804296),
	(1.888702, 1.888702),
	(2.230756, 2.230756),
	(1.783102, 1.783102),
	(2.315120, 2.315120),
	(1.701528, 1.701528),
	(2.018823, 2.018823),
	(2.054098, 2.054098),
	(1.680233, 1.680233),
	(1.923212, 1.923212),
	(2.086148, 2.074678),
	(1.667869, 1.667869),
	(1.625376, 1.625376),
	(1.645312, 1.645312),
	(2.154679, 2.154679),
	(1.679828, 1.679828),
	(2.543193, 2.538924),
	(2.099988, 2.099988),
	(1.308916, 1.308916),
	(1.906371, 1.906371),
	(1.789229, 1.789229),
	(2.007410, 2.007410),
	(1.749949, 1.749949),
	(2.170652, 2.170652),
	(2.630592, 2.630592),
	(1.810523, 1.798372),
	(1.822853, 1.822853),
	(2.248383, 2.248383),
	(2.021065, 2.021065),
	(1.776863, 1.776863),
	(1.549656, 1.549656),
	(1.445034, 1.445034),
	(1.976297, 1.976297),
	(2.067158, 2.067158),
	(2.391713, 2.391713),
	(1.745688, 1.745688),
	(1.352592, 1.352592),
	(2.225363, 2.225363),
	(2.449870, 2.444205),
	(1.801683, 1.801683),
	(1.578561, 1.578561),
	(1.848769, 1.848769),
	(2.400100, 2.362637),
	(1.974778, 1.973788),
	(1.522413, 1.522413),
	(2.074933, 2.051672),
	(2.317088, 2.317088),
	(1.697507, 1.697507),
	(2.072365, 2.072365),
	(2.371067, 2.371067),
	(1.726809, 1.726809),
	(1.923833, 1.808097),
	(2.267327, 2.267327),
	(1.797190, 1.797190),
	(1.622793, 1.622793),
	(2.276200, 2.276200),
	(2.161100, 2.161100),
	(1.538281, 1.538281),
	(1.975052, 1.975052),
	(1.536081, 1.536081),
	(1.636425, 1.636425),
	(1.208387, 1.208387),
	(3.639874, 3.639874),
	(1.550483, 1.550483),
	(0.474155, 0.474155),
	(1.873022, 1.849322),
	(1.666088, 1.664890),
	(1.549277, 1.549277),
	(1.416229, 1.416229),
	(0.755194, 0.755194),
	(1.308831, 1.308831),
	(2.689109, 2.688172),
	(2.284598, 2.284598),
	(1.687628, 1.687628),
	(1.579833, 1.579833),
	(2.152980, 2.152980),
	(1.510873, 1.510873),
	(1.932046, 1.932046),
	(1.808629, 1.808629),
	(1.255111, 1.255111),
	(1.712349, 1.711195),
	(1.861663, 1.861663),
	(1.707801, 1.707801),
	(1.364668, 1.363673),
	(1.365248, 1.364615),
	(2.027209, 2.027209),
	(1.413438, 1.413438),
	(1.372354, 1.372354),
	(1.062902, 1.062902),
	(2.156643, 2.156643),
	(1.665612, 1.665612),
	(2.077194, 2.077194),
	(2.825160, 2.825160),
	(1.307738, 1.307738),
	(1.287174, 1.287174),
	(2.727112, 2.727112),
	(2.262869, 2.262869),
	(1.974911, 1.974911),
	(2.133613, 2.132380),
	(1.160151, 1.160151),
	(2.149875, 2.149875),
	(2.811018, 2.811018),
	(2.128644, 2.128644),
	(1.798749, 1.798749),
	(1.914527, 1.914527),
	(2.613744, 2.613744),
	(1.703841, 1.703841),
	(1.861825, 1.861825),
	(2.857525, 2.857525),
	(1.527374, 1.527374),
	(2.187902, 2.187902),
	(1.587991, 1.587991),
	(2.820611, 2.820611),
	(1.945374, 1.945374),
	(1.963491, 1.963491),
	(1.524586, 1.524586),
	(1.854338, 1.835159),
	(1.927602, 1.927602),
	(1.970945, 1.968218),
	(1.544026, 1.544026),
	(1.890894, 1.890894),
	(2.002953, 2.002953),
	(1.797967, 1.797967),
	(1.880663, 1.880663),
	(1.771516, 1.771516),
	(1.845141, 1.845141),
	(2.335310, 2.335310),
	(1.572059, 1.572059),
	(2.016006, 2.016006),
	(2.132102, 2.129571),
	(1.608863, 1.608863),
	(2.163423, 2.163423),
	(1.791297, 1.791297),
	(1.985829, 1.985829),
	(1.788124, 1.777764),
	(1.225095, 1.225095),
	(1.940907, 1.940907),
	(2.661356, 2.655411),
	(2.258527, 2.258527),
	(1.905669, 1.902892),
	(1.658077, 1.658077),
	(2.161942, 2.161942),
	(1.077195, 1.076252),
	(1.325243, 1.325243),
	(1.895109, 1.880192),
	(1.182797, 1.182797),
	(1.308206, 1.305095),
	(1.462897, 1.462897),
	(1.777269, 1.776702),
	(1.316872, 1.316872),
	(1.176763, 1.176763),
	(1.633466, 1.602243),
	(1.524880, 1.524880),
	(2.376033, 2.376033),
	(1.791758, 1.791758),
	(2.679374, 2.679374),
	(1.238410, 1.229701),
	(1.359541, 1.359541),
	(2.755216, 2.755216),
	(1.524401, 1.524043),
	(2.839572, 2.839572),
	(2.336539, 2.304999),
	(1.301596, 1.301244),
	(1.058548, 1.058548),
	(1.581604, 1.580214),
	(1.391316, 1.390967),
	(3.232853, 3.232853),
	(1.193389, 1.192935),
	(1.446871, 1.444616),
	(1.051891, 1.051891),
	(2.210339, 2.210339),
	(1.626349, 1.590956),
	(0.280112, 0.280112),
	(1.369925, 1.358570),
	(1.575139, 1.574725),
	(2.078069, 2.078069),
	(2.107155, 2.107155),
	(1.894260, 1.863521),
	(2.238143, 2.238143),
	(2.393958, 2.393958),
	(1.867601, 1.867601),
	(1.992069, 1.992069),
	(1.802870, 1.802870),
	(2.106430, 2.106430),
	(1.812243, 1.812243),
	(2.371529, 2.371529),
	(2.179697, 2.179697),
	(1.967142, 1.967142),
	(1.876348, 1.876348),
	(1.711939, 1.711939),
	(2.263083, 2.263083),
	(2.102724, 2.060418),
	(2.071222, 2.071222),
	(1.547514, 1.547514),
	(2.044422, 2.044422),
	(1.969349, 1.969349),
	(2.252188, 2.252188),
	(2.284764, 2.256473),
	(2.489517, 2.475913),
	(1.503657, 1.503657),
	(1.730161, 1.730161),
	(2.635364, 2.635364),
	(2.124936, 2.124936),
	(1.969784, 1.969784),
	(1.924919, 1.924919),
	(1.229835, 1.229835),
	(2.561979, 2.561979),
	(2.122305, 2.122305),
	(2.352855, 2.346747),
	(2.793730, 2.793730),
	(1.965967, 1.965967),
	(2.011704, 2.011704),
	(1.315237, 1.315237),
	(2.527642, 2.527642),
	(1.495667, 1.495667),
	(1.674587, 1.674587),
	(1.342955, 1.342955),
	(1.439132, 1.439132),
	(1.523548, 1.523548),
	(1.654166, 1.653446),
	(0.916913, 0.916913),
	(1.668971, 1.666928),
	(1.209515, 1.206553),
	(1.390191, 1.390191),
	(2.193040, 2.193040),
	(1.536779, 1.536426),
	(1.424210, 1.424210),
	(2.009253, 2.009253),
	(2.385389, 2.385389),
	(1.041705, 1.041705),
	(1.494771, 1.494771),
	(2.300907, 2.298375),
	(2.980770, 2.979637),
	(1.214446, 1.214446),
	(2.157710, 2.148499),
	(1.682132, 1.682132),
	(1.787293, 1.779795),
	(2.799451, 2.799451),
	(0.937365, 0.937365),
	(2.093888, 2.093888),
	(2.183958, 2.183958),
	(1.367908, 1.367908),
	(1.596560, 1.588186),
	(1.957596, 1.957596),
	(1.862027, 1.862027),
	(2.789159, 2.789159),
	(2.313731, 2.313731),
	(0.902256, 0.902256),
	(1.736453, 1.736453),
	(1.364948, 1.364948),
	(1.556410, 1.482422),
	(1.562474, 1.562474),
	(1.510074, 1.510074),
	(1.979731, 1.971864),
	(2.183313, 2.183313),
	(2.615507, 2.615507),
	(1.421625, 1.421625),
	(2.444485, 2.442189),
	(1.717086, 1.717086),
	(1.785043, 1.783701),
	(1.342380, 1.342380),
	(1.226423, 1.226423),
	(1.505712, 1.505712),
	(3.935914, 3.935914),
	(1.886057, 1.886057),
	(2.020036, 2.020036),
	(2.605763, 2.605763),
	(2.514098, 2.514098),
	(1.993174, 1.990240),
	(1.783338, 1.783338),
	(1.241576, 1.241576),
	(1.455996, 1.455996),
	(2.266780, 2.266780),
	(0.873373, 0.873373),
	(1.235051, 1.230261),
	(2.465414, 2.464989),
	(2.128992, 2.128992),
	(2.637773, 2.636727),
	(2.335363, 2.335363),
	(1.186956, 1.186956),
	(1.442494, 1.437202),
	(1.510831, 1.510831),
	(1.312175, 1.311741),
	(1.709505, 1.709505),
	(1.299640, 1.299455),
	(0.992400, 0.991925),
	(3.202825, 3.202825),
	(1.950448, 1.950448),
	(2.390703, 2.390190),
	(2.255786, 2.255786),
	(1.119892, 1.119892),
	(2.691556, 2.691556),
	(1.614669, 1.614669),
	(1.365348, 1.365348),
	(1.610770, 1.610770),
	(1.674147, 1.669608),
	(1.702009, 1.702009),
	(1.476341, 1.476341),
	(2.564403, 2.564403),
	(1.378588, 1.378588),
	(1.121476, 1.121476),
	(1.489097, 1.489097),
	(4.683293, 4.682829),
	(1.566165, 1.566165),
	(1.555348, 1.555348),
	(1.908819, 1.908819),
	(2.378093, 2.374877),
	(1.192900, 1.192900),
	(1.494070, 1.494070),
	(1.188644, 1.185051),
	(1.371453, 1.362865),
	(2.466654, 2.466654),
	(1.244483, 1.238770),
	(2.004557, 2.003601),
	(1.376854, 1.375289),
	(0.769216, 0.766286),
	(1.061715, 1.060310),
	(1.348672, 1.348434),
	(1.428627, 1.420503),
	(0.785401, 0.779628),
	(1.212964, 1.212964),
	(1.273444, 1.272287),
	(0.832470, 0.830328),
	(1.302211, 1.300885),
	(1.134739, 1.131871),
	(1.786417, 1.782781),
	(3.161387, 3.161387),
	(1.099353, 1.097164),
	(1.470415, 1.465540),
	(1.377475, 1.375346),
	(1.277873, 1.277873),
	(1.650159, 1.645076),
	(0.979379, 0.979379),
	(1.159999, 1.158282),
	(0.982292, 0.982292),
	(1.243570, 1.241887),
	(2.648283, 2.644642),
	(2.789201, 2.789201),
	(0.937811, 0.935740),
	(1.263335, 1.263335),
	(1.226493, 1.226493),
	(1.110657, 1.109020),
	(0.839413, 0.839106),
	(3.042867, 3.042867),
	(1.811448, 1.810064),
	(1.389331, 1.389331),
	(2.759495, 2.756116),
	(1.335390, 1.329374),
	(1.113074, 1.105029),
	(0.982039, 0.978049),
	(0.790781, 0.788297),
	(1.203548, 1.203401),
	(1.265788, 1.265788),
	(1.795092, 1.744170),
	(0.895446, 0.895446),
	(1.053475, 1.052551),
	(2.278884, 2.278884),
	(2.287931, 2.287931),
	(1.589207, 1.582832),
	(2.201968, 2.200010),
	(2.267998, 2.261558),
	(1.661558, 1.656795),
	(4.431567, 4.430167),
	(1.431956, 1.423696),
	(1.308565, 1.308311),
	(0.878465, 0.878465),
	(1.355689, 1.355689),
	(1.854164, 1.854164),
	(1.262126, 1.241137),
	(1.131213, 1.131213),
	(1.721521, 1.720829),
	(1.765181, 1.758493),
	(2.380434, 2.380434),
	(3.078018, 3.078018),
	(1.548534, 1.548061),
	(1.117626, 1.117626),
	(2.523099, 2.523099),
} \fill \pos circle(0.03);

\draw (0,0) -- (5, 5);

\end{tikzpicture}
}
\only<+>{\charttitle{\timeC} \begin{tikzpicture}

\draw (0,0) -- (5,0);
\node at (2.5,-0.6) {RPI.LU};
\node [anchor=north] at (0.714285714285714,0) {\tiny 10};
\draw (0.714285714285714,0) -- (0.714285714285714,0.1);
\draw [style=help lines] (0.714285714285714,0) -- (0.714285714285714,5);
\node [anchor=north] at (1.42857142857143,0) {\tiny 20};
\draw (1.42857142857143,0) -- (1.42857142857143,0.1);
\draw [style=help lines] (1.42857142857143,0) -- (1.42857142857143,5);
\node [anchor=north] at (2.14285714285714,0) {\tiny 30};
\draw (2.14285714285714,0) -- (2.14285714285714,0.1);
\draw [style=help lines] (2.14285714285714,0) -- (2.14285714285714,5);
\node [anchor=north] at (2.85714285714286,0) {\tiny 40};
\draw (2.85714285714286,0) -- (2.85714285714286,0.1);
\draw [style=help lines] (2.85714285714286,0) -- (2.85714285714286,5);
\node [anchor=north] at (3.57142857142857,0) {\tiny 50};
\draw (3.57142857142857,0) -- (3.57142857142857,0.1);
\draw [style=help lines] (3.57142857142857,0) -- (3.57142857142857,5);
\node [anchor=north] at (4.28571428571429,0) {\tiny 60};
\draw (4.28571428571429,0) -- (4.28571428571429,0.1);
\draw [style=help lines] (4.28571428571429,0) -- (4.28571428571429,5);
\node [anchor=north] at (5,0) {\tiny 70};
\draw (5,0) -- (5,0.1);
\draw [style=help lines] (5,0) -- (5,5);
\draw (0,0) -- (0,5);
\node [rotate=90] at (-2.5em,2.5) {RPI[3]LU};
\node [anchor=east] at (0,0.713285714285714) {\tiny 10};
\draw (0,0.713285714285714) -- (0.1,0.713285714285714);
\draw [style=help lines] (0,0.713285714285714) -- (5,0.713285714285714);
\node [anchor=east] at (0,1.42657142857143) {\tiny 20};
\draw (0,1.42657142857143) -- (0.1,1.42657142857143);
\draw [style=help lines] (0,1.42657142857143) -- (5,1.42657142857143);
\node [anchor=east] at (0,2.13985714285714) {\tiny 30};
\draw (0,2.13985714285714) -- (0.1,2.13985714285714);
\draw [style=help lines] (0,2.13985714285714) -- (5,2.13985714285714);
\node [anchor=east] at (0,2.85314285714286) {\tiny 40};
\draw (0,2.85314285714286) -- (0.1,2.85314285714286);
\draw [style=help lines] (0,2.85314285714286) -- (5,2.85314285714286);
\node [anchor=east] at (0,3.56642857142857) {\tiny 50};
\draw (0,3.56642857142857) -- (0.1,3.56642857142857);
\draw [style=help lines] (0,3.56642857142857) -- (5,3.56642857142857);
\node [anchor=east] at (0,4.27971428571429) {\tiny 60};
\draw (0,4.27971428571429) -- (0.1,4.27971428571429);
\draw [style=help lines] (0,4.27971428571429) -- (5,4.27971428571429);
\node [anchor=east] at (0,4.993) {\tiny 70};
\draw (0,4.993) -- (0.1,4.993);
\draw [style=help lines] (0,4.993) -- (5,4.993);
\foreach \pos in {
	(0.015662, 0.012461),
	(0.008075, 0.004779),
	(0.012127, 0.007924),
	(0.002175, 0.001186),
	(0.004734, 0.003152),
	(0.001659, 0.000516),
	(0.003952, 0.002221),
	(0.000350, 0.000324),
	(0.005059, 0.002851),
	(0.003747, 0.002243),
	(0.003165, 0.002064),
	(0.004182, 0.002464),
	(0.004170, 0.002691),
	(0.004769, 0.002593),
	(0.031448, 0.017243),
	(0.003405, 0.002585),
	(0.002368, 0.001481),
	(0.003651, 0.002011),
	(0.004125, 0.002783),
	(0.005682, 0.003128),
	(0.006618, 0.003309),
	(0.004289, 0.002660),
	(0.004149, 0.002599),
	(0.003114, 0.001487),
	(0.004452, 0.002944),
	(0.003773, 0.002219),
	(0.002414, 0.001498),
	(0.004080, 0.002267),
	(0.004098, 0.002250),
	(0.004239, 0.002785),
	(0.003593, 0.001948),
	(0.001914, 0.000930),
	(0.004474, 0.002480),
	(0.001216, 0.000913),
	(0.003731, 0.002363),
	(0.003268, 0.002330),
	(0.001678, 0.001017),
	(0.000609, 0.000479),
	(0.002015, 0.001089),
	(0.000667, 0.000505),
	(0.002330, 0.001710),
	(0.001998, 0.001246),
	(0.001847, 0.001207),
	(0.002735, 0.001600),
	(0.001804, 0.001214),
	(0.000644, 0.000609),
	(0.002471, 0.001220),
	(0.002671, 0.002262),
	(0.001732, 0.000978),
	(0.004441, 0.002792),
	(0.002867, 0.001506),
	(0.002308, 0.001270),
	(0.001526, 0.000732),
	(0.001952, 0.001094),
	(0.001974, 0.001235),
	(0.001193, 0.000746),
	(0.003062, 0.002025),
	(0.001699, 0.000581),
	(0.003560, 0.001249),
	(0.001454, 0.000787),
	(0.001161, 0.000803),
	(0.001768, 0.000795),
	(0.000870, 0.000550),
	(0.001709, 0.000977),
	(0.002009, 0.001343),
	(0.000244, 0.000233),
	(0.002449, 0.001509),
	(0.001450, 0.000882),
	(0.000551, 0.000646),
	(0.000728, 0.000491),
	(0.000442, 0.000302),
	(0.001251, 0.000665),
	(0.001284, 0.000948),
	(0.000367, 0.000363),
	(0.000709, 0.000442),
	(0.001649, 0.000739),
	(0.001034, 0.000638),
	(0.000826, 0.000455),
	(0.001271, 0.000762),
	(0.001474, 0.000838),
	(0.000101, 0.000083),
	(0.000832, 0.000502),
	(0.000427, 0.000323),
	(0.000685, 0.000411),
	(0.000663, 0.000370),
	(0.000244, 0.000242),
	(0.000948, 0.000574),
	(0.001071, 0.000489),
	(0.000621, 0.000349),
	(0.000453, 0.000316),
	(0.000771, 0.000249),
	(0.000490, 0.000282),
	(0.000425, 0.000177),
	(0.000434, 0.000175),
	(0.000305, 0.000201),
	(0.000088, 0.000088),
	(0.001158, 0.000991),
	(0.011999, 0.008034),
	(0.000046, 0.000035),
	(0.000051, 0.000046),
	(0.000124, 0.000065),
	(0.000113, 0.000075),
	(0.000106, 0.000071),
	(0.000111, 0.000071),
	(0.008175, 0.003111),
	(0.000288, 0.000201),
	(0.013441, 0.005451),
	(0.003921, 0.001359),
	(0.003943, 0.001377),
	(0.003111, 0.001137),
	(0.000791, 0.000338),
	(0.001205, 0.000623),
	(0.000871, 0.000342),
	(0.000802, 0.000407),
	(0.000981, 0.000648),
	(0.000269, 0.000119),
	(0.000459, 0.000212),
	(0.000361, 0.000218),
	(0.000208, 0.000116),
	(0.000069, 0.000037),
	(0.000200, 0.000097),
	(0.000044, 0.000039),
	(0.000117, 0.000095),
	(0.000048, 0.000020),
	(0.000038, 0.000020),
	(0.000089, 0.000040),
	(0.000058, 0.000018),
	(0.000019, 0.000010),
	(0.000037, 0.000038),
	(0.000028, 0.000012),
	(0.000001, 0.000001),
	(0.000061, 0.000024),
	(0.000046, 0.000020),
	(0.000017, 0.000016),
	(0.000020, 0.000009),
	(0.000025, 0.000015),
	(0.000019, 0.000009),
	(0.000030, 0.000013),
	(0.000006, 0.000006),
	(0.000002, 0.000001),
	(0.000004, 0.000003),
	(0.000007, 0.000004),
	(0.000011, 0.000004),
	(0.064155, 0.044265),
	(0.037181, 0.022225),
	(0.014868, 0.008213),
	(0.023422, 0.015848),
	(0.016926, 0.009848),
	(0.046884, 0.029090),
	(0.022287, 0.014194),
	(0.022394, 0.013575),
	(0.008975, 0.005098),
	(0.065844, 0.034673),
	(0.018732, 0.011084),
	(0.006712, 0.004453),
	(0.007242, 0.004893),
	(0.005591, 0.003547),
	(0.048301, 0.029733),
	(0.002851, 0.001064),
	(0.024203, 0.017529),
	(0.017537, 0.009673),
	(0.026840, 0.014196),
	(0.028242, 0.013543),
	(0.013317, 0.008967),
	(0.015096, 0.009218),
	(0.043800, 0.024348),
	(0.003809, 0.002114),
	(0.038130, 0.020967),
	(0.021980, 0.011597),
	(0.020717, 0.010805),
	(0.038947, 0.021464),
	(0.016033, 0.010136),
	(0.004983, 0.003814),
	(0.013343, 0.007450),
	(0.009006, 0.005565),
	(0.017715, 0.011971),
	(0.021649, 0.014511),
	(0.010563, 0.007130),
	(0.000690, 0.000705),
	(0.020578, 0.012898),
	(0.048917, 0.025603),
	(0.020991, 0.010923),
	(0.006346, 0.001870),
	(0.010600, 0.006826),
	(0.007795, 0.004208),
	(0.013259, 0.007542),
	(0.017238, 0.010850),
	(0.031139, 0.016865),
	(0.033392, 0.020080),
	(0.019525, 0.008732),
	(0.014280, 0.008373),
	(0.029762, 0.015198),
	(0.008867, 0.004916),
	(0.022859, 0.012974),
	(0.020955, 0.014319),
	(0.009943, 0.005594),
	(0.004790, 0.003285),
	(0.001398, 0.000859),
	(0.012349, 0.008030),
	(0.005810, 0.004037),
	(0.011723, 0.007150),
	(0.019335, 0.009718),
	(0.038829, 0.026791),
	(0.012481, 0.008852),
	(0.004753, 0.002683),
	(0.009715, 0.005982),
	(0.002944, 0.002106),
	(0.022096, 0.011870),
	(0.022420, 0.012506),
	(0.010114, 0.004879),
	(0.002409, 0.001578),
	(0.000190, 0.000136),
	(0.005534, 0.003415),
	(0.008935, 0.004933),
	(0.014153, 0.008417),
	(0.014189, 0.008555),
	(0.008345, 0.002675),
	(0.004377, 0.002772),
	(0.026761, 0.017659),
	(0.006420, 0.003871),
	(0.005355, 0.003282),
	(0.011538, 0.007514),
	(0.002755, 0.001623),
	(0.008488, 0.005142),
	(0.011845, 0.006297),
	(0.017985, 0.009905),
	(0.011598, 0.006774),
	(0.013275, 0.009337),
	(0.014736, 0.008731),
	(0.013197, 0.007871),
	(0.004341, 0.003207),
	(0.008575, 0.004753),
	(0.010767, 0.005560),
	(0.010634, 0.006014),
	(0.004103, 0.002162),
	(0.003571, 0.002292),
	(0.008383, 0.005745),
	(0.010463, 0.006015),
	(0.009929, 0.007002),
	(0.003376, 0.002023),
	(0.009228, 0.005889),
	(0.019854, 0.011758),
	(0.002140, 0.001387),
	(0.004246, 0.002454),
	(0.016805, 0.009318),
	(0.008751, 0.005874),
	(0.011502, 0.008111),
	(0.002600, 0.002175),
	(0.022869, 0.014974),
	(0.004431, 0.003020),
	(0.007707, 0.004490),
	(0.006310, 0.005134),
	(0.019096, 0.009947),
	(0.005583, 0.003144),
	(0.005111, 0.002543),
	(0.006090, 0.004038),
	(0.006734, 0.004722),
	(0.008665, 0.004151),
	(0.010852, 0.006687),
	(0.008730, 0.005075),
	(0.017856, 0.011202),
	(0.012187, 0.007214),
	(0.010589, 0.005834),
	(0.008544, 0.005511),
	(0.003053, 0.001844),
	(0.010089, 0.006820),
	(0.004447, 0.002319),
	(0.009132, 0.005630),
	(0.010687, 0.005704),
	(0.006267, 0.003827),
	(0.006570, 0.004352),
	(0.002070, 0.001423),
	(0.016466, 0.010510),
	(0.006063, 0.002922),
	(0.007492, 0.004121),
	(0.007638, 0.004303),
	(0.002771, 0.001763),
	(0.002502, 0.001412),
	(0.007706, 0.003710),
	(0.091221, 0.065880),
	(0.016478, 0.009594),
	(0.031057, 0.018156),
	(0.030181, 0.019660),
	(0.039870, 0.023853),
	(0.035506, 0.021636),
	(0.023551, 0.013155),
	(0.059101, 0.032227),
	(0.033836, 0.020824),
	(0.020126, 0.013132),
	(0.017253, 0.011000),
	(0.037015, 0.018468),
	(0.061861, 0.038449),
	(0.023125, 0.016208),
	(0.064157, 0.033683),
	(0.020844, 0.013057),
	(0.119403, 0.068476),
	(0.035372, 0.018083),
	(0.023441, 0.014637),
	(0.059672, 0.034818),
	(0.036508, 0.022938),
	(0.017546, 0.010036),
	(0.016487, 0.010461),
	(0.033814, 0.022412),
	(0.018741, 0.012022),
	(0.067019, 0.034781),
	(0.015333, 0.009422),
	(0.024615, 0.016989),
	(0.012250, 0.007290),
	(0.095776, 0.052673),
	(0.051367, 0.029581),
	(0.046873, 0.021706),
	(0.002409, 0.001650),
	(0.040773, 0.023683),
	(0.027971, 0.012758),
	(0.038097, 0.022392),
	(0.001716, 0.001171),
	(0.006262, 0.004529),
	(0.019021, 0.011935),
	(0.028572, 0.018835),
	(0.042858, 0.028394),
	(0.011366, 0.006151),
	(0.045356, 0.031208),
	(0.030339, 0.018317),
	(0.001000, 0.000525),
	(0.036179, 0.021054),
	(0.031705, 0.019225),
	(0.073650, 0.044378),
	(0.005784, 0.003462),
	(0.023228, 0.013181),
	(0.058114, 0.028032),
	(0.033116, 0.020262),
	(0.028572, 0.018434),
	(0.022751, 0.014023),
	(0.008706, 0.004325),
	(0.013958, 0.007898),
	(0.034395, 0.019512),
	(0.039703, 0.025298),
	(0.019726, 0.010739),
	(0.009507, 0.004855),
	(0.021332, 0.013115),
	(0.035605, 0.022532),
	(0.026426, 0.018917),
	(0.024551, 0.011078),
	(0.034736, 0.019407),
	(0.030552, 0.017042),
	(0.009645, 0.004709),
	(0.015504, 0.009629),
	(0.004005, 0.003014),
	(0.024631, 0.016054),
	(0.028183, 0.019641),
	(0.016620, 0.007997),
	(0.035883, 0.022193),
	(0.024681, 0.015809),
	(0.052355, 0.023868),
	(0.025799, 0.014600),
	(0.031783, 0.020491),
	(0.012594, 0.007384),
	(0.013440, 0.008152),
	(0.048546, 0.026850),
	(0.009475, 0.005730),
	(0.021113, 0.011462),
	(0.004994, 0.002591),
	(0.035341, 0.021855),
	(0.025001, 0.017311),
	(0.015651, 0.011121),
	(0.023313, 0.012163),
	(0.016055, 0.011388),
	(0.024767, 0.012960),
	(0.029632, 0.018566),
	(0.048386, 0.030845),
	(0.064001, 0.041598),
	(0.021884, 0.012621),
	(0.031667, 0.015926),
	(0.025336, 0.013904),
	(0.015700, 0.010152),
	(0.020810, 0.013679),
	(0.032154, 0.018680),
	(0.016340, 0.009418),
	(0.025466, 0.014930),
	(0.047791, 0.030574),
	(0.006779, 0.003698),
	(0.013815, 0.009575),
	(0.020012, 0.012393),
	(0.032786, 0.019694),
	(0.017608, 0.011237),
	(0.021328, 0.014472),
	(0.027401, 0.016542),
	(0.023790, 0.016493),
	(0.019468, 0.015004),
	(0.028861, 0.018258),
	(0.064572, 0.039686),
	(0.012211, 0.006878),
	(0.009518, 0.005510),
	(0.006602, 0.003829),
	(0.036364, 0.020791),
	(0.015992, 0.008367),
	(0.020664, 0.013612),
	(0.003447, 0.002015),
	(0.005734, 0.002834),
	(0.021373, 0.013656),
	(0.015986, 0.009468),
	(0.011827, 0.007170),
	(0.010523, 0.005284),
	(0.008946, 0.004981),
	(0.002062, 0.001248),
	(0.017678, 0.012495),
	(0.048004, 0.027849),
	(0.012232, 0.007346),
	(0.067730, 0.041694),
	(0.075309, 0.046791),
	(0.028876, 0.018119),
	(0.048324, 0.031998),
	(0.055688, 0.037398),
	(0.043592, 0.024796),
	(0.031628, 0.017373),
	(0.093747, 0.051578),
	(0.076622, 0.045834),
	(0.041886, 0.027340),
	(0.064113, 0.033384),
	(0.023124, 0.014132),
	(0.090713, 0.053048),
	(0.061893, 0.030808),
	(0.022868, 0.014329),
	(0.065272, 0.035016),
	(0.092031, 0.060053),
	(0.035467, 0.021149),
	(0.039954, 0.023694),
	(0.012770, 0.008001),
	(0.043570, 0.025605),
	(0.026064, 0.016220),
	(0.027830, 0.016487),
	(0.058943, 0.035509),
	(0.007846, 0.004165),
	(0.060369, 0.034496),
	(0.083694, 0.045408),
	(0.013703, 0.009113),
	(0.049186, 0.032773),
	(0.068478, 0.041593),
	(0.038755, 0.024064),
	(0.062496, 0.036834),
	(0.069945, 0.041083),
	(0.130975, 0.075795),
	(0.043494, 0.022559),
	(0.042134, 0.026688),
	(0.041582, 0.025962),
	(0.025800, 0.014220),
	(0.009030, 0.005405),
	(0.055723, 0.028743),
	(0.046922, 0.028835),
	(0.083657, 0.045191),
	(0.038603, 0.023562),
	(0.047185, 0.034037),
	(0.045429, 0.025393),
	(0.155854, 0.094987),
	(0.070973, 0.041609),
	(0.002257, 0.001629),
	(0.061267, 0.036606),
	(0.056811, 0.032878),
	(0.092735, 0.056182),
	(0.075484, 0.044774),
	(0.023714, 0.013422),
	(0.031171, 0.019067),
	(0.070050, 0.033963),
	(0.009327, 0.005206),
	(0.026085, 0.017890),
	(0.054113, 0.027512),
	(0.022780, 0.015110),
	(0.039059, 0.021186),
	(0.038486, 0.023109),
	(0.040682, 0.028610),
	(0.024348, 0.015354),
	(0.032546, 0.019643),
	(0.026365, 0.016803),
	(0.007557, 0.004226),
	(0.048418, 0.028155),
	(0.050878, 0.030169),
	(0.031705, 0.019616),
	(0.031454, 0.018825),
	(0.055541, 0.031109),
	(0.016109, 0.010332),
	(0.027561, 0.018639),
	(0.025316, 0.014611),
	(0.022412, 0.014425),
	(0.021739, 0.014496),
	(0.036357, 0.020718),
	(0.024260, 0.017607),
	(0.066113, 0.036446),
	(0.019886, 0.012366),
	(0.039684, 0.018719),
	(0.025538, 0.016081),
	(0.087450, 0.049560),
	(0.033377, 0.020521),
	(0.007559, 0.004536),
	(0.010875, 0.006890),
	(0.080909, 0.039690),
	(0.082069, 0.045276),
	(0.050615, 0.032033),
	(0.044931, 0.028283),
	(0.023341, 0.013666),
	(0.045990, 0.027499),
	(0.004455, 0.002907),
	(0.040119, 0.024955),
	(0.077547, 0.050927),
	(0.038621, 0.025705),
	(0.043617, 0.023757),
	(0.042626, 0.025165),
	(0.025197, 0.015017),
	(0.124056, 0.069768),
	(0.070131, 0.046182),
	(0.066006, 0.039825),
	(0.077347, 0.048772),
	(0.024298, 0.016479),
	(0.069634, 0.042099),
	(0.035432, 0.024258),
	(0.014785, 0.009759),
	(0.022018, 0.014552),
	(0.027346, 0.018779),
	(0.021209, 0.011747),
	(0.090750, 0.041239),
	(0.060948, 0.037060),
	(0.078288, 0.044994),
	(0.119023, 0.064554),
	(0.026566, 0.017520),
	(0.039317, 0.022682),
	(0.059125, 0.040674),
	(0.011138, 0.005591),
	(0.082325, 0.045116),
	(0.007414, 0.004142),
	(0.055317, 0.032683),
	(0.028152, 0.016933),
	(0.148003, 0.111628),
	(0.008086, 0.004200),
	(0.112180, 0.073524),
	(0.073077, 0.046584),
	(0.076474, 0.041091),
	(0.079340, 0.046118),
	(0.079852, 0.051429),
	(0.025531, 0.012172),
	(0.046943, 0.024529),
	(0.045060, 0.029439),
	(0.064646, 0.044575),
	(0.091286, 0.045585),
	(0.093925, 0.065075),
	(0.027954, 0.019222),
	(0.024487, 0.016225),
	(0.013299, 0.007322),
	(0.052992, 0.033867),
	(0.111268, 0.065356),
	(0.039293, 0.025574),
	(0.035268, 0.019513),
	(0.004944, 0.003648),
	(0.107038, 0.068681),
	(0.089715, 0.052499),
	(0.148456, 0.086393),
	(0.022813, 0.013279),
	(0.084444, 0.048861),
	(0.022746, 0.013713),
	(0.058905, 0.039481),
	(0.136142, 0.083087),
	(0.084512, 0.053124),
	(0.083007, 0.053335),
	(0.015579, 0.007013),
	(0.045820, 0.024647),
	(0.051616, 0.031476),
	(0.159029, 0.083022),
	(0.043939, 0.029004),
	(0.044912, 0.028817),
	(0.037875, 0.020607),
	(0.036772, 0.021205),
	(0.064207, 0.037913),
	(0.078505, 0.048478),
	(0.073101, 0.044942),
	(0.030727, 0.020220),
	(0.041455, 0.021423),
	(0.093757, 0.059226),
	(0.111463, 0.070219),
	(0.093403, 0.055909),
	(0.090895, 0.059932),
	(0.049736, 0.027908),
	(0.067083, 0.042378),
	(0.043847, 0.023955),
	(0.049126, 0.032411),
	(0.092086, 0.043999),
	(0.045599, 0.026858),
	(0.041716, 0.024915),
	(0.100267, 0.060603),
	(0.126775, 0.071027),
	(0.080594, 0.041220),
	(0.033269, 0.020864),
	(0.086912, 0.053216),
	(0.027315, 0.016654),
	(0.026636, 0.016890),
	(0.042119, 0.023905),
	(0.038420, 0.019661),
	(0.073058, 0.046312),
	(0.030834, 0.025059),
	(0.026863, 0.015491),
	(0.059761, 0.033246),
	(0.040363, 0.025673),
	(0.053860, 0.040821),
	(0.071337, 0.037409),
	(0.078406, 0.038678),
	(0.021833, 0.013685),
	(0.014088, 0.007750),
	(0.066182, 0.039453),
	(0.037176, 0.021652),
	(0.026984, 0.015877),
	(0.069545, 0.046033),
	(0.158004, 0.084667),
	(0.009762, 0.006676),
	(0.051004, 0.036298),
	(0.040675, 0.023020),
	(0.131142, 0.073059),
	(0.072446, 0.040555),
	(0.020781, 0.009804),
	(0.100895, 0.065859),
	(0.044491, 0.022126),
	(0.050346, 0.023965),
	(0.108897, 0.066422),
	(0.057454, 0.032350),
	(0.049924, 0.031539),
	(0.074805, 0.046283),
	(0.047619, 0.027261),
	(0.051604, 0.031333),
	(0.024408, 0.013820),
	(0.018341, 0.010889),
	(0.005631, 0.003143),
	(0.067223, 0.039080),
	(0.037870, 0.021440),
	(0.038714, 0.021102),
	(0.039173, 0.024167),
	(0.029546, 0.019385),
	(0.027010, 0.017584),
	(0.000082, 0.000082),
	(0.115609, 0.056297),
	(0.040990, 0.023750),
	(0.078940, 0.044307),
	(0.038944, 0.019708),
	(0.036124, 0.022137),
	(0.077135, 0.051411),
	(0.045953, 0.029157),
	(0.032470, 0.022015),
	(0.004010, 0.002775),
	(0.088019, 0.050451),
	(0.037174, 0.019069),
	(0.054489, 0.034381),
	(0.040186, 0.017086),
	(0.109894, 0.058152),
	(0.206081, 0.117461),
	(0.055041, 0.027972),
	(0.218284, 0.127359),
	(0.170945, 0.088317),
	(0.035145, 0.021281),
	(0.025558, 0.014490),
	(0.234615, 0.114205),
	(0.064823, 0.036178),
	(0.044739, 0.029868),
	(0.021336, 0.011170),
	(0.084753, 0.045835),
	(0.040672, 0.025368),
	(0.099146, 0.059750),
	(0.005395, 0.003179),
	(0.048662, 0.029060),
	(0.090774, 0.044847),
	(0.160336, 0.084976),
	(0.194339, 0.119934),
	(0.256912, 0.158145),
	(0.024745, 0.013012),
	(0.167706, 0.098004),
	(0.159395, 0.094300),
	(0.045993, 0.031651),
	(0.150034, 0.076102),
	(0.080053, 0.043022),
	(0.127718, 0.078859),
	(0.051117, 0.033500),
	(0.022534, 0.014458),
	(0.024553, 0.016667),
	(0.191934, 0.102649),
	(0.050297, 0.025989),
	(0.088384, 0.054934),
	(0.038192, 0.019429),
	(0.172546, 0.094138),
	(0.103130, 0.059963),
	(0.044389, 0.022996),
	(0.152553, 0.086544),
	(0.022690, 0.013295),
	(0.079510, 0.043872),
	(0.074202, 0.044676),
	(0.116189, 0.072575),
	(0.008040, 0.005134),
	(0.165220, 0.102963),
	(0.016127, 0.010142),
	(0.067553, 0.039565),
	(0.063153, 0.035755),
	(0.161368, 0.094546),
	(0.078342, 0.043664),
	(0.058132, 0.036869),
	(0.042743, 0.022773),
	(0.059835, 0.037840),
	(0.150380, 0.090189),
	(0.076864, 0.046097),
	(0.083760, 0.044104),
	(0.098948, 0.062349),
	(0.065323, 0.038535),
	(0.191803, 0.124242),
	(0.071525, 0.044494),
	(0.078085, 0.048005),
	(0.223880, 0.114893),
	(0.050904, 0.035813),
	(0.159045, 0.086025),
	(0.053888, 0.031197),
	(0.076527, 0.045437),
	(0.185491, 0.094503),
	(0.106292, 0.066456),
	(0.254852, 0.142086),
	(0.075235, 0.045169),
	(0.052605, 0.032159),
	(0.084298, 0.043209),
	(0.027528, 0.017139),
	(0.075992, 0.046387),
	(0.144780, 0.088282),
	(0.040483, 0.024638),
	(0.026478, 0.017374),
	(0.083794, 0.054912),
	(0.038489, 0.023770),
	(0.109405, 0.067633),
	(0.092651, 0.054430),
	(0.021242, 0.012541),
	(0.052510, 0.032199),
	(0.019139, 0.010715),
	(0.043291, 0.023160),
	(0.097788, 0.046358),
	(0.099846, 0.061982),
	(0.200806, 0.137894),
	(0.070693, 0.040762),
	(0.125061, 0.064972),
	(0.064222, 0.035414),
	(0.139572, 0.085974),
	(0.129419, 0.063340),
	(0.029711, 0.016852),
	(0.176435, 0.106309),
	(0.096337, 0.049994),
	(0.114776, 0.062664),
	(0.069331, 0.047751),
	(0.102987, 0.053384),
	(0.171371, 0.098394),
	(0.040013, 0.025704),
	(0.097581, 0.054697),
	(0.058136, 0.036855),
	(0.060876, 0.033382),
	(0.068441, 0.047820),
	(0.101202, 0.059313),
	(0.060172, 0.040171),
	(0.364748, 0.215432),
	(0.110468, 0.075793),
	(0.003543, 0.002342),
	(0.084235, 0.051398),
	(0.171943, 0.089283),
	(0.475468, 0.251508),
	(0.466252, 0.246303),
	(0.582788, 0.341616),
	(0.096056, 0.054369),
	(0.357701, 0.244775),
	(0.065607, 0.046080),
	(0.065287, 0.039985),
	(0.048303, 0.024355),
	(0.365981, 0.203752),
	(0.595673, 0.306342),
	(0.051204, 0.027934),
	(0.217171, 0.131265),
	(0.372085, 0.222755),
	(0.399695, 0.212718),
	(0.429567, 0.244533),
	(0.389240, 0.211866),
	(0.300133, 0.173301),
	(0.364435, 0.204019),
	(0.128018, 0.086066),
	(0.149458, 0.077897),
	(0.588874, 0.324784),
	(0.204249, 0.133298),
	(0.092643, 0.053951),
	(0.097189, 0.066113),
	(0.092250, 0.052004),
	(0.066834, 0.036541),
	(0.019703, 0.012784),
	(0.032971, 0.014558),
	(0.197491, 0.125647),
	(0.015528, 0.009426),
	(0.037932, 0.023373),
	(0.388350, 0.223539),
	(0.326815, 0.187209),
	(0.038385, 0.020118),
	(0.077137, 0.055251),
	(0.029464, 0.014358),
	(0.020559, 0.008900),
	(0.010618, 0.006323),
	(0.002811, 0.001937),
	(0.125949, 0.064661),
	(0.038438, 0.020709),
	(0.027894, 0.017288),
	(0.106374, 0.062865),
	(0.035086, 0.017719),
	(0.090429, 0.056473),
	(0.002407, 0.001415),
	(0.105651, 0.069584),
	(0.078964, 0.038647),
	(0.145078, 0.087611),
	(0.168480, 0.102608),
	(0.229841, 0.149154),
	(0.126077, 0.070378),
	(0.169505, 0.102309),
	(0.098574, 0.048286),
	(0.087888, 0.051656),
	(0.057679, 0.036411),
	(0.076439, 0.042828),
	(0.083518, 0.048648),
	(0.151137, 0.089486),
	(0.037918, 0.023548),
	(0.173587, 0.102366),
	(0.148814, 0.074165),
	(0.092024, 0.055785),
	(0.116204, 0.066716),
	(0.180806, 0.108186),
	(0.075087, 0.046551),
	(0.515985, 0.261649),
	(0.403895, 0.220281),
	(0.175070, 0.100561),
	(1.220661, 0.646248),
	(0.249202, 0.131794),
	(1.154137, 0.582480),
	(0.133718, 0.062949),
	(0.436152, 0.211991),
	(0.434891, 0.238591),
	(0.910039, 0.519309),
	(0.228351, 0.128341),
	(0.360133, 0.209276),
	(0.348018, 0.191438),
	(0.817330, 0.414784),
	(1.440674, 0.789116),
	(0.942579, 0.512445),
	(0.787338, 0.440383),
	(0.447837, 0.217723),
	(1.257564, 0.777063),
	(0.839370, 0.533312),
	(0.373844, 0.186145),
	(0.065401, 0.033381),
	(0.004286, 0.003017),
	(0.166219, 0.107469),
	(0.076851, 0.038497),
	(1.242008, 0.682334),
	(0.644406, 0.369215),
	(0.108626, 0.055742),
	(0.574722, 0.341359),
	(0.254428, 0.134100),
	(0.271943, 0.151723),
	(0.644657, 0.342582),
	(0.601184, 0.319433),
	(0.955829, 0.486719),
	(1.423368, 0.877674),
	(0.006865, 0.004602),
	(0.292640, 0.170996),
	(1.468773, 0.766858),
	(0.971811, 0.556684),
	(0.660507, 0.400817),
	(1.513678, 0.791585),
	(0.000020, 0.000016),
	(1.678164, 0.922811),
	(0.497438, 0.264072),
	(0.049719, 0.035199),
	(0.050897, 0.030772),
	(0.250835, 0.136331),
	(0.089698, 0.051240),
	(0.157687, 0.086191),
	(0.088664, 0.049212),
	(0.183521, 0.107158),
	(0.133437, 0.096187),
	(0.027991, 0.016364),
	(0.061719, 0.035478),
	(0.138916, 0.076472),
	(0.092623, 0.043768),
	(0.094408, 0.057227),
	(0.140070, 0.080881),
	(0.119411, 0.070904),
	(0.087284, 0.048005),
	(0.113063, 0.065260),
	(0.077212, 0.043317),
	(0.094361, 0.062436),
	(0.062807, 0.042496),
	(0.119352, 0.070434),
	(0.054838, 0.032631),
	(0.090582, 0.057065),
	(0.108146, 0.050690),
	(0.029617, 0.020698),
	(0.145873, 0.094753),
	(0.116617, 0.065737),
	(0.151316, 0.089904),
	(0.155115, 0.092055),
	(0.044904, 0.025200),
	(0.539541, 0.323148),
	(0.004317, 0.003524),
	(0.230133, 0.104633),
	(0.407596, 0.222324),
	(0.036203, 0.016675),
	(0.191436, 0.099775),
	(0.046280, 0.022914),
	(1.198645, 0.677052),
	(0.052689, 0.023902),
	(0.329369, 0.174046),
	(1.115820, 0.630513),
	(0.015567, 0.010561),
	(0.014776, 0.011376),
	(0.683862, 0.363207),
	(0.727758, 0.406620),
	(0.144523, 0.082915),
	(0.595162, 0.336716),
	(0.703658, 0.427003),
	(0.754461, 0.385205),
	(0.183032, 0.094416),
	(0.700991, 0.363872),
	(0.715535, 0.430338),
	(0.251388, 0.138293),
	(0.032633, 0.018721),
	(0.197001, 0.124344),
	(0.137572, 0.079403),
	(0.275697, 0.141575),
	(0.183561, 0.081739),
	(0.000325, 0.000208),
	(0.321536, 0.156070),
	(0.328033, 0.169913),
	(0.154826, 0.093875),
	(0.003686, 0.002566),
	(0.247633, 0.161511),
	(0.100454, 0.061016),
	(0.006041, 0.004014),
	(0.298080, 0.158186),
	(0.779707, 0.433199),
	(0.717206, 0.400987),
	(0.402221, 0.185619),
	(0.070696, 0.038427),
	(0.064879, 0.036180),
	(0.000043, 0.000027),
	(0.001374, 0.000260),
	(0.702361, 0.385855),
	(0.330708, 0.172093),
	(0.861561, 0.525688),
	(0.212313, 0.121018),
	(0.378623, 0.197370),
	(0.039426, 0.021354),
	(0.009161, 0.005981),
	(0.155491, 0.104395),
	(0.170840, 0.083317),
	(0.817247, 0.466849),
	(0.236192, 0.116901),
	(0.164499, 0.085595),
	(0.258000, 0.176700),
	(0.627997, 0.402864),
	(0.181852, 0.077268),
	(0.005330, 0.003508),
	(0.725848, 0.389457),
	(0.007015, 0.004502),
	(0.002114, 0.001631),
	(0.336273, 0.174589),
	(0.356980, 0.216958),
	(0.086054, 0.054583),
	(0.093894, 0.055623),
	(0.011849, 0.007742),
	(0.266462, 0.150889),
	(1.442767, 0.769733),
	(0.271390, 0.128970),
	(0.114735, 0.062075),
	(0.950124, 0.476531),
	(0.171206, 0.119729),
	(0.627022, 0.373086),
	(0.746949, 0.385126),
	(1.855058, 0.963984),
	(2.043015, 1.040228),
	(0.686984, 0.363433),
	(1.720376, 0.969630),
	(0.467282, 0.290142),
	(0.005863, 0.003037),
	(0.644044, 0.346486),
	(1.008343, 0.484079),
	(0.159052, 0.091243),
	(0.074294, 0.048093),
	(0.003728, 0.002396),
	(0.690677, 0.347941),
	(1.694203, 0.936206),
	(1.155813, 0.663049),
	(0.469419, 0.346172),
	(0.000146, 0.000101),
	(2.749053, 1.599000),
	(0.152150, 0.086178),
	(0.610397, 0.358105),
	(1.526483, 0.795053),
	(0.405777, 0.288961),
	(0.843690, 0.373897),
	(0.832481, 0.463822),
	(1.858251, 0.986780),
	(0.928563, 0.455326),
	(0.144414, 0.071755),
	(0.192782, 0.108234),
	(1.082603, 0.636337),
	(2.112347, 1.156037),
	(1.575602, 0.863710),
	(0.013562, 0.011520),
	(3.304895, 1.681595),
	(2.054999, 1.037534),
	(3.318522, 1.749381),
	(1.352099, 0.783573),
	(2.273986, 1.158809),
	(4.526040, 2.418525),
	(0.975424, 0.524336),
	(0.960825, 0.526421),
	(0.832066, 0.425844),
	(2.184104, 1.151806),
	(0.941805, 0.497491),
	(1.088886, 0.611029),
	(2.806112, 1.495103),
	(0.479820, 0.243944),
	(0.028429, 0.018828),
	(3.027360, 1.571761),
	(0.803126, 0.438501),
	(2.446648, 1.253979),
	(2.202356, 1.230309),
	(1.316013, 0.675193),
	(1.617911, 0.991174),
	(2.196312, 1.207517),
	(0.997364, 0.541491),
	(3.693079, 1.862214),
	(0.253522, 0.141616),
	(0.010848, 0.007380),
	(0.953646, 0.509296),
	(3.210441, 1.806361),
	(2.262304, 1.409803),
	(1.251781, 0.711751),
	(1.369814, 0.717231),
	(0.004434, 0.002751),
	(1.088125, 0.559910),
	(1.266871, 0.602615),
	(0.254708, 0.123257),
	(1.720766, 0.862363),
	(2.170784, 1.093023),
	(1.061846, 0.634940),
	(1.232475, 0.677668),
	(0.744844, 0.407408),
	(0.871102, 0.456804),
	(0.429040, 0.230349),
	(0.918512, 0.549533),
	(0.568603, 0.354127),
	(0.139815, 0.072420),
	(1.770638, 0.841626),
	(1.372755, 0.769782),
	(0.929621, 0.503590),
	(0.118981, 0.069979),
	(1.343856, 0.664007),
	(0.562560, 0.240582),
	(0.692528, 0.334783),
	(0.274163, 0.147473),
	(0.487001, 0.302607),
	(1.421280, 0.811553),
	(0.054493, 0.034233),
	(1.568611, 0.887736),
	(1.089679, 0.649853),
	(0.432816, 0.225343),
	(1.546146, 0.814993),
	(0.005934, 0.004363),
	(0.092923, 0.045947),
	(1.269342, 0.776762),
	(1.141472, 0.672443),
	(0.000104, 0.000068),
} \fill \pos circle(0.03);
\draw (0,0) -- (5, 5);
\end{tikzpicture}
}
\end{frame}

\begin{frame}{LUnivRPI}
\begin{subpart}{\asGoodLURPI}
  \item Collect units during fixing (top-down traversal).
\end{subpart}
\addtocounter{beamerpauses}{1}
\begin{subpart}{Problems}<+->
  \item<.-> If a unit $\{a\}$ depends on a unit $\{b\}$ it'll be seen as $\left\{a, \dual{b}\right\}$.
  \item<+-> What to do if we find $\left\{a, \dual{b}\right\}$ after $\{b\}$ ?
  \item<+-> A new algorithm extending LU is needed.
\end{subpart}
\end{frame}

\section{LowerUnivalents}
\subsection{A new algorithm extending LU}

\begin{frame}{Lowering a node}
  \begin{subpart}{The (generalized) problem}
    \item Given $\psi[\eta] \odot_{a_0} \eta_0 \odot_{a_1} \cdots \odot_{a_{n-1}} \eta_{n-1}$
    \item is $\text{Fix}\left(\psi[]\right) \odot_a \eta \odot_{a_0} \eta_0 \odot_{a_1} \cdots \odot_{a_{n-1}} \eta_{n-1}$ equivalent ?
  \end{subpart}
  \begin{subpart}{Two steps}
    \item Deleting the node : $\text{Fix}\left(\psi[]\right)$ ;
    \item Reintroducing it : $\odot_a \eta$.
  \end{subpart}
  \begin{subpart}{Beware of introduced literals}
    \item $\Delta = \set{\dual{a}_i}{i < n}$ is the safe literals of $\text{Fix}\left(\psi[]\right) \odot_a \eta$.
  \end{subpart}
\end{frame}

\begin{frame}{Conditions}
\begin{subpart}{Literals introduced by reintroducing the node}
  \item Let $\Gamma_+$ be $\eta$'s clause,
  \item $\Gamma \setminus \Delta = \{a\}$.
\end{subpart}
\begin{definition}[Active literal]
Let's consider a node $\eta$ with clause $\Gamma_+$. A literal $a$ from
$\Gamma_+$ is said to be an active literal of $\eta$ iff $a$ is the pivot of
one of $\eta$'s child.
\end{definition}
\begin{subpart}{Literals introduced by deleting the node}
  \item Let $\Gamma_-$ be the set of the duals of $\eta$'s active literals,
  \item $\Gamma_- \setminus \Delta = \{\dual{a}\}$.
\end{subpart}
\end{frame}

\begin{frame}{Partial regularization}
\begin{subpart}{Deletable node \uncover<3->{(implemented)}}
  \item If $\Gamma_- \setminus \Delta = \varnothing$ then delete $\eta$.
\end{subpart}
\begin{subpart}<2->{Partial regularization \uncover<3->{(not implemented)}}
  \item If the dual of any of $\eta$'s active literal belongs to $\Delta$ then delete the edge.
\end{subpart}
\end{frame}

\begin{frame}{Algorithm}
  \SetKw{aElseIf}{else if}
  \SetKwFor{And}{and}{then}{endif}
  \begin{algorithm}[H]
    $\Delta \leftarrow \varnothing$ \;
    \For{every node $\eta$ in a top-down traversal}{
      Fix $\eta$ \;
      Compute $\Gamma_- \setminus \Delta$ \;
      \uIf{$\Gamma_- \setminus \Delta = \varnothing$}{Delete $\eta$ \;}
    \aElseIf{$\Gamma_- \setminus \Delta = \left\{\dual{a}\right\}$}
    \And{$\Gamma_+ \setminus \Delta = \{a\}$}{
        Lower $\eta$ \;
        $\Delta \leftarrow \Delta \cup \{ a \}$ \;
      }
     }
  \end{algorithm}
\end{frame}

\begin{frame}{Comparison with LU}
\centering
\only<+>{\charttitle{\compC} \begin{tikzpicture}

\draw (0,0) -- (5,0);
\node at (2.5,-0.6) {LU};
\node [anchor=north] at (0.714285714285714,0) {\tiny 0.1};
\draw (0.714285714285714,0) -- (0.714285714285714,0.1);
\draw [style=help lines] (0.714285714285714,0) -- (0.714285714285714,5);
\node [anchor=north] at (1.42857142857143,0) {\tiny 0.2};
\draw (1.42857142857143,0) -- (1.42857142857143,0.1);
\draw [style=help lines] (1.42857142857143,0) -- (1.42857142857143,5);
\node [anchor=north] at (2.14285714285714,0) {\tiny 0.3};
\draw (2.14285714285714,0) -- (2.14285714285714,0.1);
\draw [style=help lines] (2.14285714285714,0) -- (2.14285714285714,5);
\node [anchor=north] at (2.85714285714286,0) {\tiny 0.4};
\draw (2.85714285714286,0) -- (2.85714285714286,0.1);
\draw [style=help lines] (2.85714285714286,0) -- (2.85714285714286,5);
\node [anchor=north] at (3.57142857142857,0) {\tiny 0.5};
\draw (3.57142857142857,0) -- (3.57142857142857,0.1);
\draw [style=help lines] (3.57142857142857,0) -- (3.57142857142857,5);
\node [anchor=north] at (4.28571428571429,0) {\tiny 0.6};
\draw (4.28571428571429,0) -- (4.28571428571429,0.1);
\draw [style=help lines] (4.28571428571429,0) -- (4.28571428571429,5);
\node [anchor=north] at (5,0) {\tiny 0.7};
\draw (5,0) -- (5,0.1);
\draw [style=help lines] (5,0) -- (5,5);
\draw (0,0) -- (0,5);
\node [rotate=90] at (-2.5em,2.5) {LUniv};
\node [anchor=east] at (0,0.713285714285714) {\tiny 0.1};
\draw (0,0.713285714285714) -- (0.1,0.713285714285714);
\draw [style=help lines] (0,0.713285714285714) -- (5,0.713285714285714);
\node [anchor=east] at (0,1.42657142857143) {\tiny 0.2};
\draw (0,1.42657142857143) -- (0.1,1.42657142857143);
\draw [style=help lines] (0,1.42657142857143) -- (5,1.42657142857143);
\node [anchor=east] at (0,2.13985714285714) {\tiny 0.3};
\draw (0,2.13985714285714) -- (0.1,2.13985714285714);
\draw [style=help lines] (0,2.13985714285714) -- (5,2.13985714285714);
\node [anchor=east] at (0,2.85314285714286) {\tiny 0.4};
\draw (0,2.85314285714286) -- (0.1,2.85314285714286);
\draw [style=help lines] (0,2.85314285714286) -- (5,2.85314285714286);
\node [anchor=east] at (0,3.56642857142857) {\tiny 0.5};
\draw (0,3.56642857142857) -- (0.1,3.56642857142857);
\draw [style=help lines] (0,3.56642857142857) -- (5,3.56642857142857);
\node [anchor=east] at (0,4.27971428571429) {\tiny 0.6};
\draw (0,4.27971428571429) -- (0.1,4.27971428571429);
\draw [style=help lines] (0,4.27971428571429) -- (5,4.27971428571429);
\node [anchor=east] at (0,4.993) {\tiny 0.7};
\draw (0,4.993) -- (0.1,4.993);
\draw [style=help lines] (0,4.993) -- (5,4.993);
\foreach \pos in {
	(1.272598, 1.276143),
	(1.450326, 1.627991),
	(1.284322, 1.319119),
	(1.021798, 1.089918),
	(1.110039, 1.260189),
	(0.132854, 0.132854),
	(1.179331, 1.306991),
	(0.250054, 0.250054),
	(1.470460, 2.015883),
	(1.065261, 1.272707),
	(1.107271, 1.312084),
	(1.310616, 1.380828),
	(0.894188, 0.942091),
	(1.570378, 1.706933),
	(0.432020, 0.597337),
	(0.620481, 0.640023),
	(1.140783, 1.633394),
	(1.336120, 2.159321),
	(1.425847, 1.489420),
	(1.555722, 2.681376),
	(1.060101, 1.192613),
	(1.276143, 1.581886),
	(1.378106, 1.790567),
	(1.300758, 1.611809),
	(1.356535, 1.398634),
	(1.159251, 1.592506),
	(1.318182, 1.357143),
	(1.001385, 1.198466),
	(1.043478, 1.062112),
	(1.617849, 1.728878),
	(1.254918, 1.777235),
	(1.198686, 1.839080),
	(1.132487, 1.282576),
	(1.347197, 1.374322),
	(1.500000, 1.553571),
	(1.569343, 1.569343),
	(1.475987, 1.820128),
	(1.539129, 2.146109),
	(1.097712, 1.175015),
	(1.514154, 2.007900),
	(0.760432, 0.760432),
	(1.553733, 1.899007),
	(1.480757, 1.767776),
	(1.112475, 1.388676),
	(1.163882, 1.379763),
	(1.860465, 2.868217),
	(1.540349, 1.676663),
	(1.190476, 1.345486),
	(1.594388, 1.817602),
	(1.246146, 1.303957),
	(1.075731, 1.269363),
	(1.456461, 1.495197),
	(1.774223, 2.702703),
	(1.107623, 1.203558),
	(1.185714, 1.278571),
	(1.514608, 2.141961),
	(1.111111, 1.269841),
	(0.132171, 0.132171),
	(0.391725, 0.393255),
	(1.827862, 2.111801),
	(1.444724, 1.902369),
	(1.337507, 1.716942),
	(1.354267, 1.378236),
	(1.591615, 1.659550),
	(1.059969, 1.079068),
	(1.800115, 1.872120),
	(1.079202, 1.079202),
	(0.700081, 0.700081),
	(1.530612, 1.828231),
	(1.422203, 1.528338),
	(1.796322, 2.857143),
	(1.634383, 1.634383),
	(1.481309, 1.512331),
	(1.710076, 2.157327),
	(1.360000, 1.634286),
	(0.869048, 1.011905),
	(1.507937, 1.666667),
	(1.453856, 1.706700),
	(1.617681, 1.765778),
	(0.649351, 0.649351),
	(1.194162, 1.503759),
	(1.621549, 1.688555),
	(1.702009, 1.702009),
	(1.609658, 1.911469),
	(1.691729, 1.691729),
	(1.628264, 1.628264),
	(1.389838, 1.503759),
	(1.538908, 1.713124),
	(1.941610, 2.168367),
	(1.562500, 1.562500),
	(0.169062, 0.169062),
	(1.665329, 2.126806),
	(0.168651, 0.168651),
	(0.168651, 0.168651),
	(1.230039, 1.251618),
	(1.071429, 1.071429),
	(1.852586, 1.852586),
	(0.301245, 0.301245),
	(0.789474, 0.789474),
	(0.875576, 0.921659),
	(0.115830, 0.115830),
	(1.214286, 1.892857),
	(1.152074, 1.728111),
	(1.152074, 1.728111),
	(0.027069, 0.027069),
	(1.497950, 1.592558),
	(0.135566, 0.135566),
	(0.096834, 0.096834),
	(0.096834, 0.096834),
	(0.112102, 0.112102),
	(0.316115, 0.316115),
	(0.605963, 0.745388),
	(0.172637, 0.172637),
	(0.600537, 0.600537),
	(0.406039, 0.406039),
	(0.355450, 2.775897),
	(0.414891, 0.414891),
	(0.683230, 0.683230),
	(0.869361, 1.198308),
	(1.127820, 1.127820),
	(0.190194, 0.190194),
	(1.751152, 1.751152),
	(1.549509, 1.549509),
	(0.264550, 0.264550),
	(0.601504, 0.977444),
	(0.309066, 0.309066),
	(0.257400, 0.514801),
	(1.020408, 1.020408),
	(1.512605, 1.512605),
	(0.612245, 0.612245),
	(0.000000, 0.000000),
	(0.099206, 0.297619),
	(0.480192, 0.840336),
	(1.219512, 1.219512),
	(0.446429, 0.446429),
	(0.426439, 0.426439),
	(0.728863, 0.728863),
	(0.585480, 0.702576),
	(0.840336, 0.840336),
	(0.000000, 0.000000),
	(2.142857, 2.142857),
	(0.000000, 0.000000),
	(1.098901, 4.395604),
	(1.322225, 1.575296),
	(0.889988, 1.203652),
	(1.257430, 1.352146),
	(1.139118, 1.159892),
	(1.049301, 1.049301),
	(1.145456, 1.217260),
	(1.236369, 1.272840),
	(1.209710, 1.277358),
	(1.145334, 1.385234),
	(0.990635, 1.131381),
	(1.343518, 1.459992),
	(1.369708, 1.622519),
	(1.286704, 1.443418),
	(1.034568, 1.232809),
	(1.114272, 1.154914),
	(1.501316, 2.724037),
	(1.226196, 1.247277),
	(1.314170, 1.737745),
	(1.283847, 1.356793),
	(0.934725, 1.055460),
	(1.145596, 1.186673),
	(1.379636, 1.406167),
	(0.992710, 1.066387),
	(0.938692, 1.004693),
	(1.296310, 1.330127),
	(1.223482, 1.543595),
	(1.316977, 2.086239),
	(1.204819, 1.249898),
	(1.677596, 1.699722),
	(1.335492, 1.360785),
	(1.453145, 1.776854),
	(1.247661, 1.251560),
	(1.299091, 1.367690),
	(1.419137, 2.071704),
	(1.198448, 1.238308),
	(1.853480, 2.446886),
	(1.238977, 1.287478),
	(1.006160, 1.164564),
	(1.170960, 1.318167),
	(1.513686, 2.335816),
	(1.174506, 1.267590),
	(1.396601, 1.636379),
	(1.172179, 1.221758),
	(1.131184, 1.203446),
	(1.337087, 1.552115),
	(1.001556, 1.380524),
	(1.010782, 1.415094),
	(1.357909, 1.634599),
	(1.050420, 1.291062),
	(1.189206, 1.189206),
	(1.019373, 1.041113),
	(1.033998, 1.207709),
	(1.230492, 1.413899),
	(1.269004, 1.520291),
	(2.220949, 2.815590),
	(0.967511, 0.967511),
	(1.284575, 1.326013),
	(1.528492, 1.694203),
	(1.571656, 1.873722),
	(1.263979, 1.271854),
	(1.452633, 1.564032),
	(1.408518, 1.718869),
	(1.274338, 1.602494),
	(1.183985, 1.277458),
	(0.841147, 1.241037),
	(1.440644, 1.549296),
	(1.336016, 1.855131),
	(1.352333, 1.506626),
	(1.358696, 1.436335),
	(1.309524, 1.502976),
	(1.564971, 2.026226),
	(1.307321, 1.364786),
	(1.127536, 1.267803),
	(0.080621, 0.080621),
	(1.266324, 1.340522),
	(0.893279, 0.933729),
	(1.217424, 1.293513),
	(1.401636, 1.491421),
	(1.169574, 1.236656),
	(1.310762, 1.363829),
	(1.309524, 1.414835),
	(1.044343, 1.234223),
	(1.173438, 1.333452),
	(1.247073, 1.413934),
	(0.997451, 1.030699),
	(1.081134, 1.164676),
	(1.057385, 1.123472),
	(1.227158, 1.233827),
	(1.279178, 1.425459),
	(0.872741, 1.200792),
	(1.540670, 1.747750),
	(1.245476, 1.442410),
	(1.134735, 1.576684),
	(1.569826, 1.664787),
	(1.037801, 1.107375),
	(1.391874, 1.420708),
	(0.858741, 0.911996),
	(1.671048, 1.821254),
	(1.262913, 1.455026),
	(1.329591, 1.580736),
	(1.211735, 1.362477),
	(1.197532, 1.481795),
	(1.273096, 1.284362),
	(1.245147, 1.412505),
	(1.392358, 1.461039),
	(1.148005, 1.320463),
	(0.982294, 1.026760),
	(1.134735, 1.397516),
	(1.492623, 1.675626),
	(1.079349, 1.219332),
	(1.241924, 1.407035),
	(0.960000, 0.960000),
	(1.236577, 1.285389),
	(1.470776, 1.512342),
	(1.533698, 1.922523),
	(0.895420, 1.148325),
	(1.144492, 1.243273),
	(1.027658, 1.078406),
	(1.437617, 1.738063),
	(1.360029, 1.630592),
	(0.799476, 1.035387),
	(1.448465, 1.647395),
	(1.000584, 1.051709),
	(1.314735, 1.483085),
	(1.475116, 1.521661),
	(1.400495, 1.423892),
	(1.091086, 1.289867),
	(1.224047, 1.305392),
	(1.400832, 1.560333),
	(1.068871, 1.095795),
	(1.078967, 1.347531),
	(1.564819, 2.930662),
	(1.106807, 1.154241),
	(1.081717, 3.215755),
	(1.853123, 2.313952),
	(1.588559, 1.697985),
	(1.122720, 1.329787),
	(0.419412, 0.461192),
	(0.972079, 1.190476),
	(0.876380, 0.915568),
	(1.091179, 1.250491),
	(1.096647, 1.161525),
	(1.274300, 1.542711),
	(1.471587, 1.508377),
	(1.380184, 1.893287),
	(1.451270, 1.514054),
	(1.121388, 1.142953),
	(1.238858, 1.586393),
	(1.307245, 1.341422),
	(1.413938, 1.651729),
	(0.934321, 1.063320),
	(1.242464, 1.307995),
	(1.004138, 1.070574),
	(1.304660, 1.617610),
	(0.917901, 0.955177),
	(1.024957, 1.074712),
	(1.202412, 1.901621),
	(1.205745, 1.303269),
	(1.200521, 1.231860),
	(1.337999, 1.581584),
	(1.375867, 1.442441),
	(1.344720, 1.444099),
	(1.322333, 1.403330),
	(1.235400, 1.392633),
	(1.381458, 1.553142),
	(1.177601, 1.647388),
	(1.443233, 1.468432),
	(1.379364, 1.719625),
	(2.351190, 2.951389),
	(1.418897, 1.491454),
	(1.548723, 1.706337),
	(1.087654, 1.187504),
	(2.114029, 2.142501),
	(1.495550, 2.055234),
	(1.261641, 1.394953),
	(1.429143, 1.469159),
	(0.976409, 1.107471),
	(0.402440, 0.445752),
	(1.095756, 1.112111),
	(1.217846, 1.297776),
	(1.783817, 2.952786),
	(1.244913, 1.469801),
	(1.107188, 1.119157),
	(0.609519, 1.017632),
	(1.602390, 1.950932),
	(1.238470, 1.453856),
	(1.180319, 1.239233),
	(1.339475, 1.460695),
	(1.163688, 1.176546),
	(0.986612, 1.072638),
	(0.469399, 0.485379),
	(0.947638, 1.052931),
	(1.073081, 1.129024),
	(1.294794, 1.429109),
	(1.216016, 1.591696),
	(0.449928, 0.449928),
	(1.387320, 1.424152),
	(1.251618, 1.418861),
	(1.448534, 1.584174),
	(1.094137, 1.460570),
	(1.050198, 1.354203),
	(1.177290, 1.202608),
	(1.305994, 1.338082),
	(1.478688, 1.572967),
	(1.862976, 2.498477),
	(1.262192, 1.322751),
	(0.921952, 0.964271),
	(1.106233, 1.215990),
	(1.290554, 1.457848),
	(0.676999, 0.681165),
	(1.369863, 1.805983),
	(1.225351, 1.402681),
	(1.169415, 1.233602),
	(1.086820, 1.196758),
	(1.044202, 1.262012),
	(1.111196, 1.198721),
	(0.582170, 0.599812),
	(1.071054, 1.637789),
	(0.911626, 0.975976),
	(1.287090, 1.346040),
	(1.210974, 1.335856),
	(1.125145, 1.306620),
	(1.116975, 1.145894),
	(0.961672, 1.268293),
	(0.880302, 1.075269),
	(1.222130, 1.535921),
	(1.093654, 1.208900),
	(1.248828, 1.328775),
	(1.151000, 1.397643),
	(1.396553, 1.508958),
	(1.391169, 1.425379),
	(1.268336, 1.635056),
	(1.125158, 1.153603),
	(0.795038, 0.886613),
	(1.279190, 1.385073),
	(1.148678, 1.198217),
	(0.890430, 0.895978),
	(1.199564, 1.234262),
	(1.492490, 1.508469),
	(1.427878, 1.542247),
	(0.851309, 0.956451),
	(1.368186, 1.422587),
	(1.305368, 1.354258),
	(0.951066, 1.239148),
	(1.285677, 1.397313),
	(1.126485, 1.173573),
	(1.101497, 1.211954),
	(1.075302, 1.667183),
	(1.333715, 1.352814),
	(1.186017, 1.328696),
	(1.274110, 1.352517),
	(1.165097, 1.239465),
	(1.221708, 1.568930),
	(1.092113, 1.342158),
	(1.785714, 2.515811),
	(1.322031, 2.504083),
	(1.384316, 1.396707),
	(1.256761, 2.115813),
	(1.249346, 1.373626),
	(1.343311, 1.806416),
	(0.964889, 1.172608),
	(1.711601, 2.589775),
	(1.250345, 1.550843),
	(0.919893, 1.305919),
	(1.213972, 1.331454),
	(1.265776, 1.882202),
	(1.112544, 1.203913),
	(1.404503, 1.532462),
	(1.189235, 1.252545),
	(0.879947, 0.950178),
	(1.150326, 1.261510),
	(1.313769, 1.385303),
	(1.273963, 1.298229),
	(1.180477, 1.384790),
	(1.177779, 1.199110),
	(1.319537, 1.544281),
	(1.270513, 1.332905),
	(1.201299, 1.335227),
	(1.406813, 1.650660),
	(1.141365, 1.480791),
	(1.020084, 1.106416),
	(1.431743, 1.479309),
	(1.260091, 1.323271),
	(1.337806, 1.687056),
	(1.376705, 1.700070),
	(1.204872, 1.254648),
	(1.331772, 1.344262),
	(1.217692, 1.245684),
	(1.086529, 1.172308),
	(2.396761, 3.108128),
	(0.948456, 1.665795),
	(1.007098, 1.053682),
	(1.157223, 1.225630),
	(1.122777, 1.151517),
	(0.922989, 1.035587),
	(0.715053, 0.729124),
	(0.988239, 1.031268),
	(1.334724, 1.629301),
	(1.185706, 1.238858),
	(1.449717, 1.594079),
	(1.478059, 1.486818),
	(1.363824, 1.400560),
	(1.362207, 1.488823),
	(1.499118, 2.273599),
	(1.140007, 1.543760),
	(1.147914, 1.443742),
	(1.044862, 1.067094),
	(1.221287, 1.335782),
	(1.330744, 1.441040),
	(1.407743, 1.450625),
	(0.999292, 1.177902),
	(1.001698, 1.186030),
	(1.881503, 2.214815),
	(1.098649, 1.141283),
	(1.312914, 1.708484),
	(1.271715, 1.560128),
	(1.153373, 1.280583),
	(1.127820, 1.138380),
	(1.450167, 1.838336),
	(1.220749, 1.339022),
	(0.956175, 0.961867),
	(1.120603, 1.189158),
	(1.316063, 1.637813),
	(1.273361, 1.385809),
	(1.145395, 1.606225),
	(1.141264, 1.192078),
	(1.056300, 1.080614),
	(1.419530, 1.597348),
	(1.249209, 1.303425),
	(1.201895, 1.323259),
	(1.629661, 1.642229),
	(1.172869, 1.265577),
	(1.204092, 1.282233),
	(1.181039, 1.580650),
	(1.373037, 1.472616),
	(1.020408, 1.846278),
	(1.238193, 1.292028),
	(1.044073, 1.059331),
	(0.832227, 0.852650),
	(1.065209, 1.205368),
	(1.396677, 1.442975),
	(1.506835, 1.510376),
	(1.250695, 1.326134),
	(1.392461, 1.582905),
	(1.328904, 1.479196),
	(1.166780, 1.410517),
	(1.398777, 1.402647),
	(0.968186, 0.997247),
	(1.194829, 1.612693),
	(2.054206, 2.249473),
	(1.443515, 1.563060),
	(1.293204, 1.560428),
	(1.414865, 1.569616),
	(1.263530, 1.393176),
	(1.116682, 1.367366),
	(1.125558, 1.334632),
	(1.252942, 1.385862),
	(2.328917, 3.025061),
	(1.222092, 1.243915),
	(1.058374, 1.214565),
	(1.410086, 1.425132),
	(1.183205, 1.500489),
	(1.130571, 1.656646),
	(1.309175, 1.495078),
	(1.231967, 1.374527),
	(1.346681, 1.419932),
	(1.018463, 1.511356),
	(1.454697, 1.628864),
	(1.264566, 1.370306),
	(1.346939, 1.444121),
	(1.019937, 1.079350),
	(1.379142, 1.449320),
	(1.041407, 1.073795),
	(1.163309, 1.269689),
	(1.324847, 1.364922),
	(1.072111, 1.240554),
	(1.067665, 1.113224),
	(1.037226, 1.100859),
	(1.334776, 1.487734),
	(1.172675, 1.196652),
	(1.406769, 1.505399),
	(0.991431, 1.053525),
	(0.461058, 0.524897),
	(1.214544, 1.418257),
	(1.727259, 2.270327),
	(1.070887, 1.079558),
	(1.274184, 1.440297),
	(1.195791, 1.269133),
	(1.696699, 2.538857),
	(1.414297, 1.481298),
	(0.612794, 0.612794),
	(1.341278, 1.630621),
	(0.767537, 0.803237),
	(1.369791, 1.381600),
	(0.402414, 1.892606),
	(1.186645, 1.347546),
	(1.471505, 1.508683),
	(1.337641, 1.459679),
	(1.231262, 1.356823),
	(1.203008, 1.374632),
	(1.036832, 1.128380),
	(1.138659, 1.215890),
	(1.209343, 1.242476),
	(1.320462, 1.417158),
	(1.350025, 1.604688),
	(1.227187, 1.290120),
	(1.278207, 1.374976),
	(1.651684, 1.716668),
	(1.430994, 1.459766),
	(1.212889, 1.250388),
	(1.282135, 1.394403),
	(1.323664, 1.512005),
	(1.350157, 1.532649),
	(1.273424, 1.383242),
	(1.066764, 1.164111),
	(1.256241, 1.401192),
	(1.256270, 1.376227),
	(1.301487, 1.538493),
	(0.577731, 1.899273),
	(1.106420, 1.272578),
	(1.207794, 1.314149),
	(0.960248, 0.960248),
	(1.218131, 1.255222),
	(1.319525, 1.392115),
	(1.401730, 1.635351),
	(0.996241, 1.372180),
	(1.093391, 1.378049),
	(1.438294, 1.475174),
	(1.118935, 1.171988),
	(1.163858, 1.184246),
	(1.156843, 1.202797),
	(1.388330, 1.527627),
	(1.078806, 1.315853),
	(1.220590, 1.299400),
	(1.407452, 1.431588),
	(1.155566, 1.179101),
	(1.376212, 1.579636),
	(1.333916, 1.455750),
	(1.118820, 1.243404),
	(1.274616, 1.370577),
	(0.855992, 0.913538),
	(1.311117, 1.333880),
	(0.666584, 0.728534),
	(0.951138, 0.993766),
	(1.243493, 1.317855),
	(1.448652, 1.544272),
	(1.204527, 1.340268),
	(1.246263, 1.256789),
	(1.254510, 1.346385),
	(1.144192, 1.397864),
	(1.333755, 1.401707),
	(0.927588, 0.972472),
	(0.879937, 1.047545),
	(1.280983, 1.610808),
	(0.914471, 1.103426),
	(1.187837, 1.398253),
	(0.894998, 0.974221),
	(1.250866, 1.288676),
	(1.444552, 1.554226),
	(1.235893, 1.343242),
	(1.800270, 2.808701),
	(1.148810, 1.176587),
	(1.356094, 1.490564),
	(1.415296, 1.423861),
	(1.093824, 1.150401),
	(1.281639, 1.295045),
	(1.254276, 1.260384),
	(1.335065, 1.464935),
	(1.154345, 1.200667),
	(1.264180, 1.313850),
	(0.953448, 1.019755),
	(0.380796, 1.050114),
	(1.525468, 1.607307),
	(1.295739, 1.333333),
	(1.341628, 2.855644),
	(1.037597, 1.112621),
	(1.217545, 1.248211),
	(1.094450, 1.159865),
	(1.283186, 1.430678),
	(1.064566, 1.131334),
	(1.141863, 1.367252),
	(1.132925, 1.324979),
	(1.367535, 1.510469),
	(1.290665, 1.754774),
	(1.312797, 1.385156),
	(1.004279, 1.022106),
	(1.411485, 1.528490),
	(1.470092, 1.681069),
	(1.212634, 1.335509),
	(1.186570, 1.321341),
	(1.785714, 1.785714),
	(1.260378, 1.484062),
	(1.239942, 1.354263),
	(1.295379, 1.465930),
	(1.419285, 1.544456),
	(1.358124, 1.380906),
	(1.111817, 1.141314),
	(1.100908, 1.270479),
	(1.041442, 1.479566),
	(1.362727, 1.362727),
	(1.260086, 1.290300),
	(1.366904, 1.667387),
	(1.011243, 1.076928),
	(1.431659, 1.705642),
	(1.013387, 1.055310),
	(0.806705, 0.836309),
	(1.195729, 1.449774),
	(1.256583, 1.434690),
	(1.071899, 1.168733),
	(1.389421, 1.657723),
	(1.592724, 2.165040),
	(1.031961, 1.169087),
	(1.197838, 1.535623),
	(1.441266, 1.603231),
	(0.332473, 0.335133),
	(1.011585, 1.049919),
	(1.190213, 1.283224),
	(1.499782, 1.591998),
	(2.408558, 3.043478),
	(1.399263, 1.619867),
	(1.111569, 1.208437),
	(1.215587, 1.326282),
	(1.071429, 1.141582),
	(0.919191, 0.957971),
	(1.184802, 1.399292),
	(1.350590, 1.365415),
	(0.955160, 1.094092),
	(1.194026, 1.255562),
	(1.343567, 1.736432),
	(0.969643, 1.125000),
	(1.068714, 1.182249),
	(1.321620, 1.351086),
	(1.065604, 1.111622),
	(1.278453, 1.358635),
	(1.076536, 1.271675),
	(1.146336, 1.173599),
	(1.029078, 1.129347),
	(1.058901, 1.263748),
	(1.402679, 1.531071),
	(1.313204, 1.335168),
	(1.262866, 1.354113),
	(1.291612, 1.492665),
	(1.119761, 1.259962),
	(1.585265, 1.763951),
	(1.169055, 1.475404),
	(1.043739, 1.161620),
	(2.241331, 3.015257),
	(1.278940, 1.386361),
	(1.649680, 2.133356),
	(1.329327, 1.391545),
	(1.169830, 1.222612),
	(1.120392, 1.160304),
	(1.310804, 1.350060),
	(1.389566, 1.614776),
	(1.210592, 1.320313),
	(1.157098, 1.306928),
	(1.230175, 1.416081),
	(1.310342, 1.452428),
	(0.806512, 1.501008),
	(1.503389, 1.623003),
	(1.299860, 1.350835),
	(1.156688, 1.175344),
	(1.073811, 1.104962),
	(1.363098, 1.446351),
	(0.905501, 1.052965),
	(1.290380, 1.346016),
	(1.413201, 1.447357),
	(1.271590, 1.397921),
	(0.704568, 0.789602),
	(1.174389, 1.224949),
	(0.950858, 1.065101),
	(1.255945, 1.363672),
	(1.144476, 1.277420),
	(1.345628, 1.870643),
	(1.301587, 1.822511),
	(1.350804, 1.628255),
	(1.092314, 1.319515),
	(1.531180, 1.735338),
	(1.449388, 1.527061),
	(1.444946, 1.471499),
	(1.280271, 1.319151),
	(0.849415, 0.860847),
	(1.360977, 1.542845),
	(1.238354, 1.612966),
	(1.182651, 1.229604),
	(1.344026, 1.358070),
	(0.893379, 0.960174),
	(1.220452, 1.328937),
	(1.452929, 1.530815),
	(1.273072, 1.365579),
	(1.292217, 1.294315),
	(1.208906, 1.237482),
	(1.208791, 1.321179),
	(0.941360, 1.075982),
	(0.970733, 1.069924),
	(1.156876, 1.299547),
	(1.594619, 1.742306),
	(1.021902, 1.087241),
	(1.387914, 1.424476),
	(1.372227, 1.787399),
	(1.287845, 1.449229),
	(0.978294, 1.054723),
	(1.304102, 1.922207),
	(1.154542, 1.165670),
	(0.877483, 0.888497),
	(1.313248, 1.340637),
	(1.218877, 1.395293),
	(1.219154, 1.255919),
	(1.225126, 1.335264),
	(1.207037, 1.224955),
	(1.037626, 1.087481),
	(0.522629, 0.522629),
	(2.197262, 4.082134),
	(0.805395, 0.904611),
	(0.024001, 0.024001),
	(1.167589, 1.432475),
	(0.662363, 0.712070),
	(0.799083, 0.837857),
	(0.369485, 0.369485),
	(0.312702, 0.312702),
	(0.836591, 0.883461),
	(1.731857, 2.010041),
	(0.378215, 2.085585),
	(0.942572, 1.003708),
	(0.773254, 0.805449),
	(1.471532, 1.696213),
	(1.007431, 1.012356),
	(1.343115, 1.496525),
	(1.087933, 1.131666),
	(0.718243, 0.722200),
	(0.922477, 0.927666),
	(1.151096, 1.280394),
	(0.886501, 0.902652),
	(0.626095, 0.626095),
	(0.691489, 0.696555),
	(1.013897, 1.023831),
	(0.843658, 0.861164),
	(0.739418, 0.757275),
	(0.551859, 0.595471),
	(1.263274, 1.362917),
	(0.823770, 0.829794),
	(1.523062, 1.595131),
	(1.718058, 2.091192),
	(0.625778, 0.699246),
	(0.952893, 1.001563),
	(1.863674, 2.126971),
	(1.217556, 1.236384),
	(1.244532, 1.370801),
	(0.351288, 0.351288),
	(0.718541, 0.738895),
	(1.286522, 1.324799),
	(0.435442, 0.448344),
	(1.761751, 2.198056),
	(1.254200, 1.454061),
	(1.139928, 1.245046),
	(1.605964, 2.392739),
	(1.272349, 1.741459),
	(1.246910, 1.324277),
	(0.351460, 2.227189),
	(0.860362, 0.865913),
	(1.604730, 2.151705),
	(1.211153, 1.265430),
	(1.278600, 1.920784),
	(1.244660, 1.364602),
	(1.345517, 1.391537),
	(1.164532, 1.190674),
	(1.311342, 1.379666),
	(1.169712, 1.254499),
	(1.157294, 1.263659),
	(0.936616, 1.080354),
	(1.370077, 1.484250),
	(1.347001, 1.566055),
	(1.004316, 1.054188),
	(1.273222, 1.364739),
	(0.761249, 1.032110),
	(1.334658, 1.359013),
	(1.348787, 1.366038),
	(1.049275, 1.097475),
	(1.307417, 1.465807),
	(1.477920, 1.690497),
	(1.179304, 1.249868),
	(0.818235, 0.878845),
	(1.062885, 1.092995),
	(1.567088, 1.644964),
	(0.924900, 0.944896),
	(0.515864, 0.552385),
	(0.978154, 0.983729),
	(1.475834, 1.536770),
	(1.177054, 1.238119),
	(0.877169, 1.078759),
	(1.020529, 1.028166),
	(1.266087, 1.320318),
	(0.377709, 0.405362),
	(0.503579, 0.513377),
	(0.737101, 0.812566),
	(0.789990, 0.796683),
	(0.730031, 0.739367),
	(0.881746, 0.895645),
	(1.214652, 1.262057),
	(0.834569, 0.870126),
	(0.693846, 0.700050),
	(1.102682, 1.176874),
	(1.052893, 1.209585),
	(2.263872, 2.500000),
	(1.089275, 1.158955),
	(1.697471, 2.814145),
	(0.813934, 0.829932),
	(0.794097, 0.840061),
	(1.839465, 2.060838),
	(1.019488, 1.051336),
	(1.816464, 2.170540),
	(1.409433, 1.652956),
	(0.810617, 0.821348),
	(0.365600, 0.403057),
	(0.971142, 0.983136),
	(0.760802, 0.771267),
	(2.427760, 3.172003),
	(0.505454, 0.524065),
	(1.032406, 1.039733),
	(0.269904, 0.271800),
	(1.062952, 1.081279),
	(0.981196, 1.032362),
	(0.280112, 0.280112),
	(0.689558, 0.743267),
	(0.996953, 1.008147),
	(1.256342, 1.295761),
	(1.359338, 1.422058),
	(1.132201, 1.284613),
	(1.194076, 1.227679),
	(1.078535, 1.525835),
	(1.095660, 1.271884),
	(1.300378, 1.323434),
	(1.297567, 1.474319),
	(1.201412, 1.418616),
	(1.369863, 1.477357),
	(1.330779, 1.756921),
	(1.242617, 1.368341),
	(1.351059, 1.469952),
	(1.029133, 1.108397),
	(1.154523, 1.242536),
	(1.366554, 1.564574),
	(1.143406, 1.228018),
	(1.161490, 1.294058),
	(0.938141, 0.990493),
	(1.314872, 1.396601),
	(1.232369, 1.672849),
	(1.277599, 1.732998),
	(1.103341, 1.657840),
	(1.122723, 1.578054),
	(1.174914, 1.202148),
	(1.379614, 1.460418),
	(1.070617, 1.181970),
	(1.267793, 1.344598),
	(1.433113, 1.550249),
	(1.201193, 1.223083),
	(0.513145, 0.538506),
	(2.038381, 3.292582),
	(1.418688, 1.521449),
	(1.742041, 2.007134),
	(1.849183, 2.007716),
	(0.277130, 0.277130),
	(1.564949, 1.933326),
	(0.654663, 0.655648),
	(1.571869, 2.038066),
	(1.030892, 1.052907),
	(0.915497, 1.028125),
	(1.076531, 1.100916),
	(1.145947, 1.151464),
	(0.975362, 0.990528),
	(0.843279, 0.896546),
	(0.268619, 0.269421),
	(0.783054, 0.862756),
	(0.785494, 0.799321),
	(0.607323, 0.619680),
	(1.316475, 1.565759),
	(0.640472, 0.698858),
	(0.892807, 0.925918),
	(1.206322, 1.269448),
	(1.815665, 2.093871),
	(0.560743, 0.565301),
	(1.135058, 1.169820),
	(0.364212, 0.374341),
	(1.301431, 1.874242),
	(1.102589, 1.102589),
	(1.174037, 1.267300),
	(1.042215, 1.068910),
	(1.469219, 1.527230),
	(2.340758, 2.721426),
	(0.484574, 0.533860),
	(1.565514, 1.722914),
	(1.575952, 2.113430),
	(0.895087, 0.901035),
	(0.726889, 0.774623),
	(1.097183, 1.174915),
	(0.773433, 0.921697),
	(1.406961, 1.535275),
	(1.586301, 1.786164),
	(0.902256, 1.203008),
	(0.775862, 0.775862),
	(0.807937, 0.811669),
	(0.849332, 0.901581),
	(0.835656, 0.884303),
	(0.828078, 0.861757),
	(1.131010, 1.193947),
	(1.679268, 1.853031),
	(2.064525, 2.688529),
	(0.819582, 0.892705),
	(1.069534, 1.224539),
	(0.677004, 1.032199),
	(1.036130, 1.048210),
	(0.799676, 0.807691),
	(0.795058, 0.809029),
	(0.907476, 0.959578),
	(0.287070, 2.810228),
	(1.451563, 1.962045),
	(1.142199, 1.224839),
	(1.905391, 2.339292),
	(2.431861, 2.588503),
	(0.968714, 1.338399),
	(1.279643, 1.299769),
	(0.826453, 0.857568),
	(0.600322, 0.634333),
	(1.826124, 2.141703),
	(0.405141, 0.405141),
	(0.891161, 0.911918),
	(1.672094, 1.770303),
	(1.677583, 1.756810),
	(1.444682, 2.124132),
	(1.361688, 1.467587),
	(0.543568, 0.548014),
	(0.894358, 0.974960),
	(0.742252, 0.755513),
	(0.658366, 0.666829),
	(0.997303, 1.008633),
	(0.771454, 0.782696),
	(0.490734, 0.518301),
	(1.920456, 2.521373),
	(1.552516, 1.705196),
	(0.182696, 0.584525),
	(1.618516, 1.668396),
	(0.799446, 1.185817),
	(2.483808, 2.645729),
	(1.010634, 1.044843),
	(0.880889, 0.884845),
	(1.008277, 1.039199),
	(0.763168, 0.866721),
	(1.590402, 1.869420),
	(0.727560, 0.764598),
	(1.679938, 1.967213),
	(0.606740, 0.642612),
	(0.724749, 0.733779),
	(1.025399, 1.081660),
	(1.038978, 1.202856),
	(0.545722, 0.604480),
	(0.830151, 0.830530),
	(0.999623, 1.009694),
	(1.623279, 1.998006),
	(0.587149, 0.621986),
	(0.644886, 0.688103),
	(0.587452, 0.614087),
	(0.973617, 0.992591),
	(2.034603, 2.034603),
	(0.851379, 0.873838),
	(1.170294, 1.179057),
	(0.926008, 0.929500),
	(0.289838, 0.319008),
	(0.495083, 0.535707),
	(0.929613, 0.939872),
	(0.510434, 0.536488),
	(0.260592, 0.262069),
	(0.762438, 0.772954),
	(0.936730, 0.953829),
	(0.146040, 0.147245),
	(0.351559, 0.355242),
	(0.258765, 0.309561),
	(1.328327, 1.414426),
	(2.251315, 2.565106),
	(0.615168, 0.630720),
	(0.870003, 0.877890),
	(0.688174, 0.698694),
	(0.898465, 0.925756),
	(0.990755, 1.123185),
	(0.230359, 0.231455),
	(0.163261, 0.163261),
	(0.320546, 0.342207),
	(0.865954, 0.872947),
	(1.728919, 1.863378),
	(2.455078, 3.619480),
	(0.230682, 0.236153),
	(0.603315, 0.605632),
	(0.883166, 0.892351),
	(0.462719, 0.478754),
	(0.414324, 0.428165),
	(2.407623, 3.016196),
	(0.775050, 0.784181),
	(0.705016, 0.716116),
	(1.636370, 1.820516),
	(0.743906, 0.793737),
	(0.598656, 0.610172),
	(0.412039, 0.422014),
	(0.178489, 0.178489),
	(0.395074, 0.438760),
	(0.718482, 0.719632),
	(1.344375, 1.505498),
	(0.317987, 0.321124),
	(0.501237, 0.535853),
	(1.547093, 1.726427),
	(0.921194, 0.990579),
	(0.803176, 0.819222),
	(0.988765, 1.093481),
	(1.585320, 1.741375),
	(0.804908, 0.828073),
	(1.332549, 3.052126),
	(0.930127, 0.946445),
	(0.270991, 0.327268),
	(0.399790, 0.404640),
	(0.740056, 0.753881),
	(1.546920, 1.569217),
	(0.611212, 0.726272),
	(0.575082, 0.592979),
	(1.017409, 1.043709),
	(0.697458, 0.889814),
	(2.027489, 2.455066),
	(1.946954, 2.129958),
	(1.018548, 1.024703),
	(0.659997, 0.678417),
	(2.167733, 2.380952),
} \fill \pos circle(0.03);
\draw (0,0) -- (5, 5);
\end{tikzpicture}
}
\only<+>{\charttitle{\axioC} \begin{tikzpicture}

\draw (0,0) -- (5,0);
\node at (2.5,-0.6) {LU};
\node [anchor=north] at (1,0) {\tiny 0.1};
\draw (1,0) -- (1,0.1);
\draw [style=help lines] (1,0) -- (1,5);
\node [anchor=north] at (2,0) {\tiny 0.2};
\draw (2,0) -- (2,0.1);
\draw [style=help lines] (2,0) -- (2,5);
\node [anchor=north] at (3,0) {\tiny 0.3};
\draw (3,0) -- (3,0.1);
\draw [style=help lines] (3,0) -- (3,5);
\node [anchor=north] at (4,0) {\tiny 0.4};
\draw (4,0) -- (4,0.1);
\draw [style=help lines] (4,0) -- (4,5);
\node [anchor=north] at (5,0) {\tiny 0.5};
\draw (5,0) -- (5,0.1);
\draw [style=help lines] (5,0) -- (5,5);
\draw (0,0) -- (0,5);
\node [rotate=90] at (-2.5em,2.5) {LUniv};
\node [anchor=east] at (0,0.998) {\tiny 0.1};
\draw (0,0.998) -- (0.1,0.998);
\draw [style=help lines] (0,0.998) -- (5,0.998);
\node [anchor=east] at (0,1.996) {\tiny 0.2};
\draw (0,1.996) -- (0.1,1.996);
\draw [style=help lines] (0,1.996) -- (5,1.996);
\node [anchor=east] at (0,2.994) {\tiny 0.3};
\draw (0,2.994) -- (0.1,2.994);
\draw [style=help lines] (0,2.994) -- (5,2.994);
\node [anchor=east] at (0,3.992) {\tiny 0.4};
\draw (0,3.992) -- (0.1,3.992);
\draw [style=help lines] (0,3.992) -- (5,3.992);
\node [anchor=east] at (0,4.99) {\tiny 0.5};
\draw (0,4.99) -- (0.1,4.99);
\draw [style=help lines] (0,4.99) -- (5,4.99);
\foreach \pos in {
	(0.000000, 0.000000),
	(0.000000, 0.085837),
	(0.000000, 0.037665),
	(0.000000, 0.068027),
	(0.000000, 0.042283),
	(0.000000, 0.000000),
	(0.000000, 0.049140),
	(0.000000, 0.000000),
	(0.000000, 0.350515),
	(0.000000, 0.305882),
	(0.000000, 0.147783),
	(0.000000, 0.021834),
	(0.000000, 0.021692),
	(0.000000, 0.177215),
	(0.000000, 0.058565),
	(0.000000, 0.049383),
	(0.000000, 0.224719),
	(0.000000, 0.432099),
	(0.000000, 0.023810),
	(0.000000, 0.755208),
	(0.000000, 0.182768),
	(0.000000, 0.024814),
	(0.000000, 0.264317),
	(0.000000, 0.190736),
	(0.000000, 0.053908),
	(0.000000, 0.652819),
	(0.000000, 0.000000),
	(0.000000, 0.151134),
	(0.000000, 0.026042),
	(0.000000, 0.133333),
	(0.000000, 0.409357),
	(0.000000, 0.433333),
	(0.000000, 0.177515),
	(0.000000, 0.000000),
	(0.000000, 0.025707),
	(0.000000, 0.000000),
	(0.000000, 0.408805),
	(0.000000, 0.000000),
	(0.000000, 0.138122),
	(0.000000, 0.000000),
	(0.000000, 0.000000),
	(0.000000, 0.319489),
	(0.000000, 0.222841),
	(0.000000, 0.231884),
	(0.000000, 0.068966),
	(0.000000, 0.970874),
	(0.000000, 0.142857),
	(0.000000, 0.087977),
	(0.000000, 0.103093),
	(0.000000, 0.057971),
	(0.000000, 0.120120),
	(0.000000, 0.029762),
	(0.000000, 0.937500),
	(0.000000, 0.141844),
	(0.000000, 0.031250),
	(0.000000, 0.368852),
	(0.000000, 0.159236),
	(0.000000, 0.000000),
	(0.000000, 0.000000),
	(0.000000, 0.441176),
	(0.000000, 0.156863),
	(0.000000, 0.596491),
	(0.000000, 0.044444),
	(0.000000, 0.038168),
	(0.000000, 0.000000),
	(0.000000, 0.000000),
	(0.000000, 0.000000),
	(0.000000, 0.000000),
	(0.000000, 0.234375),
	(0.000000, 0.077519),
	(0.000000, 0.740741),
	(0.000000, 0.000000),
	(0.000000, 0.000000),
	(0.000000, 0.105820),
	(0.000000, 0.375000),
	(0.000000, 0.087336),
	(0.000000, 0.000000),
	(0.000000, 0.000000),
	(0.000000, 0.049261),
	(0.000000, 0.000000),
	(0.000000, 0.000000),
	(0.000000, 0.099010),
	(0.000000, 0.000000),
	(0.000000, 0.439560),
	(0.000000, 0.000000),
	(0.000000, 0.000000),
	(0.000000, 0.153061),
	(0.000000, 0.285714),
	(0.000000, 0.388889),
	(0.000000, 0.000000),
	(0.000000, 0.000000),
	(0.000000, 0.240000),
	(0.000000, 0.000000),
	(0.000000, 0.000000),
	(0.000000, 0.000000),
	(0.000000, 0.000000),
	(0.000000, 0.000000),
	(0.000000, 0.000000),
	(0.000000, 0.000000),
	(0.000000, 0.000000),
	(0.000000, 0.000000),
	(0.000000, 0.000000),
	(0.000000, 0.000000),
	(0.000000, 0.000000),
	(0.000000, 0.000000),
	(0.000000, 0.000000),
	(0.000000, 0.000000),
	(0.000000, 0.000000),
	(0.000000, 0.000000),
	(0.000000, 0.000000),
	(0.000000, 0.000000),
	(0.000000, 0.039370),
	(0.000000, 0.000000),
	(0.000000, 0.000000),
	(0.000000, 0.000000),
	(0.000000, 1.750000),
	(0.000000, 0.000000),
	(0.000000, 0.000000),
	(0.000000, 0.500000),
	(0.000000, 0.000000),
	(0.000000, 0.000000),
	(0.000000, 0.000000),
	(0.000000, 0.000000),
	(0.000000, 0.000000),
	(0.000000, 0.000000),
	(0.000000, 0.000000),
	(0.000000, 0.212766),
	(0.000000, 0.000000),
	(0.000000, 0.000000),
	(0.000000, 0.000000),
	(0.000000, 0.000000),
	(0.000000, 0.000000),
	(0.000000, 0.208333),
	(0.000000, 0.000000),
	(0.000000, 0.000000),
	(0.000000, 0.000000),
	(0.000000, 0.000000),
	(0.000000, 0.000000),
	(0.000000, 0.000000),
	(0.000000, 0.000000),
	(0.000000, 0.000000),
	(0.000000, 0.000000),
	(0.000000, 5.000000),
	(0.000000, 0.070547),
	(0.000000, 0.272315),
	(0.000000, 0.035651),
	(0.000000, 0.017123),
	(0.000000, 0.000000),
	(0.000000, 0.122592),
	(0.000000, 0.060060),
	(0.000000, 0.070547),
	(0.000000, 0.206490),
	(0.000000, 0.097879),
	(0.000000, 0.096463),
	(0.000000, 0.326531),
	(0.000000, 0.052724),
	(0.000000, 0.180180),
	(0.000000, 0.000000),
	(0.000000, 1.842900),
	(0.000000, 0.061602),
	(0.000000, 0.425532),
	(0.000000, 0.081301),
	(0.000000, 0.138648),
	(0.000000, 0.039062),
	(0.000000, 0.033613),
	(0.000000, 0.055970),
	(0.000000, 0.098039),
	(0.000000, 0.054845),
	(0.000000, 0.274443),
	(0.000000, 0.323102),
	(0.000000, 0.000000),
	(0.000000, 0.052174),
	(0.000000, 0.023202),
	(0.000000, 0.294659),
	(0.000000, 0.000000),
	(0.000000, 0.059289),
	(0.000000, 0.719844),
	(0.000000, 0.031201),
	(0.000000, 0.000000),
	(0.000000, 0.044444),
	(0.000000, 0.362595),
	(0.000000, 0.074074),
	(0.000000, 1.247002),
	(0.000000, 0.016313),
	(0.000000, 0.283019),
	(0.000000, 0.078125),
	(0.000000, 0.032000),
	(0.000000, 0.109649),
	(0.000000, 0.128676),
	(0.000000, 0.769231),
	(0.000000, 0.356394),
	(0.000000, 0.414414),
	(0.000000, 0.000000),
	(0.000000, 0.033058),
	(0.000000, 0.116279),
	(0.000000, 0.035587),
	(0.000000, 0.239808),
	(0.000000, 0.260586),
	(0.000000, 0.000000),
	(0.000000, 0.047281),
	(0.000000, 0.000000),
	(0.000000, 0.112782),
	(0.000000, 0.000000),
	(0.000000, 0.114155),
	(0.000000, 0.123457),
	(0.000000, 0.245184),
	(0.000000, 0.117994),
	(0.000000, 0.349265),
	(0.000000, 0.108932),
	(0.000000, 0.552381),
	(0.000000, 0.000000),
	(0.000000, 0.000000),
	(0.000000, 0.040984),
	(0.000000, 0.351852),
	(0.000000, 0.035651),
	(0.000000, 0.161580),
	(0.000000, 0.000000),
	(0.000000, 0.064103),
	(0.000000, 0.074074),
	(0.000000, 0.061224),
	(0.000000, 0.164319),
	(0.000000, 0.048232),
	(0.000000, 0.047847),
	(0.000000, 0.200893),
	(0.000000, 0.124777),
	(0.000000, 0.176817),
	(0.000000, 0.142602),
	(0.000000, 0.020704),
	(0.000000, 0.082781),
	(0.000000, 0.021786),
	(0.000000, 0.000000),
	(0.000000, 0.121528),
	(0.000000, 0.528846),
	(0.000000, 0.231317),
	(0.000000, 0.284553),
	(0.000000, 0.553097),
	(0.000000, 0.067912),
	(0.000000, 0.052770),
	(0.000000, 0.018832),
	(0.000000, 0.097087),
	(0.000000, 0.127660),
	(0.000000, 0.074349),
	(0.000000, 0.056022),
	(0.000000, 0.196078),
	(0.000000, 0.434783),
	(0.000000, 0.000000),
	(0.000000, 0.091743),
	(0.000000, 0.024510),
	(0.000000, 0.201465),
	(0.000000, 0.000000),
	(0.000000, 0.197183),
	(0.000000, 0.000000),
	(0.000000, 0.080321),
	(0.000000, 0.173410),
	(0.000000, 0.000000),
	(0.000000, 0.019084),
	(0.000000, 0.038462),
	(0.000000, 0.335968),
	(0.000000, 0.242131),
	(0.000000, 0.061224),
	(0.000000, 0.025000),
	(0.000000, 0.233813),
	(0.000000, 0.041152),
	(0.000000, 0.218579),
	(0.000000, 0.130208),
	(0.000000, 0.000000),
	(0.000000, 0.122324),
	(0.000000, 0.056926),
	(0.000000, 0.000000),
	(0.000000, 0.255403),
	(0.000000, 0.020921),
	(0.000000, 0.000000),
	(0.000000, 0.000000),
	(0.000000, 0.087336),
	(0.000000, 0.989247),
	(0.000000, 0.000000),
	(0.000000, 2.351421),
	(0.000000, 0.213777),
	(0.000000, 0.136986),
	(0.000000, 0.014347),
	(0.000000, 0.033445),
	(0.000000, 0.288462),
	(0.000000, 0.022523),
	(0.000000, 0.122888),
	(0.000000, 0.028090),
	(0.000000, 0.066007),
	(0.000000, 0.000000),
	(0.000000, 0.056338),
	(0.000000, 0.000000),
	(0.000000, 0.028777),
	(0.000000, 0.276680),
	(0.000000, 0.059701),
	(0.000000, 0.105820),
	(0.000000, 0.028289),
	(0.000000, 0.169731),
	(0.000000, 0.030257),
	(0.000000, 0.118519),
	(0.000000, 0.068966),
	(0.000000, 0.067295),
	(0.000000, 0.179718),
	(0.000000, 0.107198),
	(0.000000, 0.059084),
	(0.000000, 0.084890),
	(0.000000, 0.080257),
	(0.000000, 0.149750),
	(0.000000, 0.013158),
	(0.000000, 0.086831),
	(0.000000, 0.088300),
	(0.000000, 0.068966),
	(0.000000, 0.000000),
	(0.000000, 0.094044),
	(0.000000, 0.230769),
	(0.000000, 0.000000),
	(0.000000, 0.080645),
	(0.000000, 0.092450),
	(0.000000, 0.091463),
	(0.000000, 0.251938),
	(0.000000, 0.200286),
	(0.000000, 0.033223),
	(0.000000, 0.073638),
	(0.000000, 0.000000),
	(0.000000, 0.020619),
	(0.000000, 0.132890),
	(0.000000, 0.742188),
	(0.000000, 0.156986),
	(0.000000, 0.017483),
	(0.000000, 0.222222),
	(0.000000, 0.169851),
	(0.000000, 0.157819),
	(0.000000, 0.105740),
	(0.000000, 0.000000),
	(0.000000, 0.028450),
	(0.000000, 0.136426),
	(0.000000, 0.000000),
	(0.000000, 0.173913),
	(0.000000, 0.045524),
	(0.000000, 0.059435),
	(0.000000, 0.160000),
	(0.000000, 0.000000),
	(0.000000, 0.028531),
	(0.000000, 0.126783),
	(0.000000, 0.105263),
	(0.000000, 0.092807),
	(0.000000, 0.295567),
	(0.000000, 0.014451),
	(0.000000, 0.029455),
	(0.000000, 0.094637),
	(0.000000, 0.183727),
	(0.000000, 0.080386),
	(0.000000, 0.045249),
	(0.000000, 0.119863),
	(0.000000, 0.151515),
	(0.000000, 0.013175),
	(0.000000, 0.259319),
	(0.000000, 0.099256),
	(0.000000, 0.056657),
	(0.000000, 0.050251),
	(0.000000, 0.397351),
	(0.000000, 0.051370),
	(0.000000, 0.023895),
	(0.000000, 0.073314),
	(0.000000, 0.081967),
	(0.000000, 0.035336),
	(0.000000, 0.162722),
	(0.000000, 0.102916),
	(0.000000, 0.071813),
	(0.000000, 0.601719),
	(0.000000, 0.000000),
	(0.000000, 0.519263),
	(0.000000, 0.147300),
	(0.000000, 0.052724),
	(0.000000, 0.315789),
	(0.000000, 0.018622),
	(0.000000, 0.014472),
	(0.000000, 0.175953),
	(0.000000, 0.031299),
	(0.000000, 0.125000),
	(0.000000, 0.103245),
	(0.000000, 0.051903),
	(0.000000, 0.015408),
	(0.000000, 0.064103),
	(0.000000, 0.000000),
	(0.000000, 0.106762),
	(0.000000, 0.064620),
	(0.000000, 0.092025),
	(0.000000, 0.092593),
	(0.000000, 0.243056),
	(0.000000, 0.058997),
	(0.000000, 0.109489),
	(0.000000, 0.124777),
	(0.000000, 0.083056),
	(0.000000, 0.034072),
	(0.000000, 0.062630),
	(0.000000, 0.059406),
	(0.000000, 0.137931),
	(0.000000, 0.236769),
	(0.000000, 0.064935),
	(0.000000, 0.249307),
	(0.000000, 0.878735),
	(0.000000, 0.000000),
	(0.000000, 0.831826),
	(0.000000, 0.051151),
	(0.000000, 0.284757),
	(0.000000, 0.231959),
	(0.000000, 0.477327),
	(0.000000, 0.167015),
	(0.000000, 0.157729),
	(0.000000, 0.125261),
	(0.000000, 0.136612),
	(0.000000, 0.037975),
	(0.000000, 0.084541),
	(0.000000, 0.000000),
	(0.000000, 0.069324),
	(0.000000, 0.093567),
	(0.000000, 0.155317),
	(0.000000, 0.013699),
	(0.000000, 0.141011),
	(0.000000, 0.078227),
	(0.000000, 0.122616),
	(0.000000, 0.040053),
	(0.000000, 0.105820),
	(0.000000, 0.581717),
	(0.000000, 0.141026),
	(0.000000, 0.142450),
	(0.000000, 0.055866),
	(0.000000, 0.169713),
	(0.000000, 0.192044),
	(0.000000, 0.147929),
	(0.000000, 0.055866),
	(0.000000, 0.000000),
	(0.000000, 0.012642),
	(0.000000, 0.024907),
	(0.000000, 0.291060),
	(0.000000, 0.275103),
	(0.000000, 0.043796),
	(0.000000, 0.015408),
	(0.000000, 0.029586),
	(0.000000, 0.135970),
	(0.000000, 0.054171),
	(0.000000, 0.040161),
	(0.000000, 0.211864),
	(0.000000, 0.026212),
	(0.000000, 0.193741),
	(0.000000, 0.000000),
	(0.000000, 0.030120),
	(0.000000, 0.101652),
	(0.000000, 0.907504),
	(0.000000, 0.485893),
	(0.000000, 0.336134),
	(0.000000, 0.013055),
	(0.000000, 0.198830),
	(0.000000, 0.042373),
	(0.000000, 0.023364),
	(0.000000, 0.144928),
	(0.000000, 0.366412),
	(0.000000, 0.000000),
	(0.000000, 0.065189),
	(0.000000, 0.283505),
	(0.000000, 0.185449),
	(0.000000, 0.144300),
	(0.000000, 0.000000),
	(0.000000, 0.240202),
	(0.000000, 0.153417),
	(0.000000, 0.000000),
	(0.000000, 0.053981),
	(0.000000, 0.219479),
	(0.000000, 0.047619),
	(0.000000, 0.560897),
	(0.000000, 0.113636),
	(0.000000, 0.044843),
	(0.000000, 0.216667),
	(0.000000, 0.014245),
	(0.000000, 0.013850),
	(0.000000, 0.018692),
	(0.000000, 0.077220),
	(0.000000, 0.070547),
	(0.000000, 0.063211),
	(0.000000, 0.127841),
	(0.000000, 0.334507),
	(0.000000, 0.081744),
	(0.000000, 0.024510),
	(0.000000, 0.043478),
	(0.000000, 0.025641),
	(0.000000, 0.029806),
	(0.000000, 0.000000),
	(0.000000, 0.013280),
	(0.000000, 0.120664),
	(0.000000, 0.295567),
	(0.000000, 0.145310),
	(0.000000, 0.000000),
	(0.000000, 0.078125),
	(0.000000, 0.305085),
	(0.000000, 0.138340),
	(0.000000, 0.000000),
	(0.000000, 0.213675),
	(0.000000, 0.045113),
	(0.000000, 0.060976),
	(0.000000, 0.175439),
	(0.000000, 0.174326),
	(0.000000, 0.331325),
	(0.000000, 0.387097),
	(0.000000, 0.041494),
	(0.000000, 0.312925),
	(0.000000, 0.000000),
	(0.000000, 0.174564),
	(0.000000, 0.369181),
	(0.000000, 0.268562),
	(0.000000, 0.083682),
	(0.000000, 0.098453),
	(0.000000, 0.435511),
	(0.000000, 0.042017),
	(0.000000, 0.113314),
	(0.000000, 0.014144),
	(0.000000, 0.029762),
	(0.000000, 0.084034),
	(0.000000, 0.041609),
	(0.000000, 0.024010),
	(0.000000, 0.115702),
	(0.000000, 0.441826),
	(0.000000, 0.116580),
	(0.000000, 0.080128),
	(0.000000, 0.160099),
	(0.000000, 0.037594),
	(0.000000, 0.045732),
	(0.000000, 0.087209),
	(0.000000, 0.130641),
	(0.000000, 0.189504),
	(0.000000, 0.286225),
	(0.000000, 0.013280),
	(0.000000, 0.354505),
	(0.000000, 0.070423),
	(0.000000, 0.589060),
	(0.000000, 0.139860),
	(0.000000, 0.000000),
	(0.000000, 0.072029),
	(0.000000, 0.062305),
	(0.000000, 0.025873),
	(0.000000, 1.504339),
	(0.000000, 0.307018),
	(0.000000, 0.051125),
	(0.000000, 0.051948),
	(0.000000, 0.211020),
	(0.000000, 0.168675),
	(0.000000, 0.118765),
	(0.000000, 0.053079),
	(0.000000, 0.038660),
	(0.000000, 0.077434),
	(0.000000, 0.095352),
	(0.000000, 0.115830),
	(0.000000, 0.167464),
	(0.000000, 0.063291),
	(0.000000, 0.038217),
	(0.000000, 0.011641),
	(0.000000, 0.135287),
	(0.000000, 0.142672),
	(0.000000, 0.179487),
	(0.000000, 0.041841),
	(0.000000, 0.012755),
	(0.000000, 0.074906),
	(0.000000, 0.119617),
	(0.000000, 0.209699),
	(0.000000, 1.638362),
	(0.000000, 0.050891),
	(0.000000, 0.104167),
	(0.000000, 0.000000),
	(0.000000, 0.046893),
	(0.000000, 0.076726),
	(0.000000, 0.094899),
	(0.000000, 0.176678),
	(0.000000, 0.390244),
	(0.000000, 0.037360),
	(0.000000, 0.036276),
	(0.000000, 0.048485),
	(0.000000, 0.000000),
	(0.000000, 0.335121),
	(0.000000, 0.256739),
	(0.000000, 0.064020),
	(0.000000, 0.012469),
	(0.000000, 0.013210),
	(0.000000, 0.114943),
	(0.000000, 0.142518),
	(0.000000, 0.207254),
	(0.000000, 0.101695),
	(0.000000, 0.048780),
	(0.000000, 0.053476),
	(0.000000, 0.090090),
	(0.000000, 0.039526),
	(0.000000, 0.071736),
	(0.000000, 0.082742),
	(0.000000, 0.093333),
	(0.000000, 0.025253),
	(0.000000, 0.146396),
	(0.000000, 0.214286),
	(0.000000, 0.111386),
	(0.000000, 0.046893),
	(0.000000, 0.099379),
	(0.000000, 0.064185),
	(0.000000, 0.026212),
	(0.000000, 0.284653),
	(0.000000, 0.072464),
	(0.000000, 0.024752),
	(0.000000, 0.026247),
	(0.000000, 0.028818),
	(0.000000, 0.919378),
	(0.000000, 0.070755),
	(0.000000, 0.164975),
	(0.000000, 0.013986),
	(0.000000, 0.028090),
	(0.000000, 0.014205),
	(0.000000, 0.000000),
	(0.000000, 0.062992),
	(0.000000, 0.089989),
	(0.000000, 0.012903),
	(0.000000, 0.052910),
	(0.000000, 0.135699),
	(0.000000, 0.151515),
	(0.000000, 0.071259),
	(0.000000, 1.576087),
	(0.000000, 0.104651),
	(0.000000, 0.049875),
	(0.000000, 0.090772),
	(0.000000, 0.106525),
	(0.000000, 0.097297),
	(0.000000, 0.296209),
	(0.000000, 0.137845),
	(0.000000, 0.237288),
	(0.000000, 0.231214),
	(0.000000, 0.038710),
	(0.000000, 0.000000),
	(0.000000, 0.130890),
	(0.000000, 0.443114),
	(0.000000, 0.075094),
	(0.000000, 0.015175),
	(0.000000, 0.000000),
	(0.000000, 0.310559),
	(0.000000, 0.103226),
	(0.000000, 0.055249),
	(0.000000, 0.092838),
	(0.000000, 0.014903),
	(0.000000, 0.042254),
	(0.000000, 0.101302),
	(0.000000, 0.088183),
	(0.000000, 0.000000),
	(0.000000, 0.000000),
	(0.000000, 0.095628),
	(0.000000, 0.061665),
	(0.000000, 0.050761),
	(0.000000, 0.000000),
	(0.000000, 0.043956),
	(0.000000, 0.041754),
	(0.000000, 0.042735),
	(0.000000, 0.033937),
	(0.000000, 0.655556),
	(0.000000, 0.340909),
	(0.000000, 0.119904),
	(0.000000, 0.181113),
	(0.000000, 0.170843),
	(0.000000, 0.009980),
	(0.000000, 0.033076),
	(0.000000, 0.045819),
	(0.000000, 0.230150),
	(0.000000, 0.085106),
	(0.000000, 0.129260),
	(0.000000, 0.070211),
	(0.000000, 0.130719),
	(0.000000, 0.101466),
	(0.000000, 0.000000),
	(0.000000, 0.186125),
	(0.000000, 0.036585),
	(0.000000, 0.089629),
	(0.000000, 0.023148),
	(0.000000, 0.241597),
	(0.000000, 0.107643),
	(0.000000, 0.263447),
	(0.000000, 0.010132),
	(0.000000, 0.000000),
	(0.000000, 0.036188),
	(0.000000, 0.307329),
	(0.000000, 0.033822),
	(0.000000, 0.069541),
	(0.000000, 0.134078),
	(0.000000, 0.211111),
	(0.000000, 0.044150),
	(0.000000, 0.057803),
	(0.000000, 0.036014),
	(0.000000, 0.113507),
	(0.000000, 0.198830),
	(0.000000, 0.308642),
	(0.000000, 0.158395),
	(0.000000, 0.328152),
	(0.000000, 0.300334),
	(0.000000, 0.305927),
	(0.000000, 0.068027),
	(0.000000, 0.069849),
	(0.000000, 0.054825),
	(0.000000, 0.072569),
	(0.000000, 0.448276),
	(0.000000, 0.024155),
	(0.000000, 0.135983),
	(0.000000, 0.099668),
	(0.000000, 0.140693),
	(0.000000, 0.854701),
	(0.000000, 0.087500),
	(0.000000, 0.107843),
	(0.000000, 0.010695),
	(0.000000, 0.047337),
	(0.000000, 0.010638),
	(0.000000, 0.171123),
	(0.000000, 0.066741),
	(0.000000, 0.080092),
	(0.000000, 0.075377),
	(0.000000, 0.087719),
	(0.000000, 0.069045),
	(0.000000, 0.092486),
	(0.000000, 0.048193),
	(0.000000, 0.144357),
	(0.000000, 0.083218),
	(0.000000, 0.444444),
	(0.000000, 0.187500),
	(0.000000, 0.319270),
	(0.000000, 0.142518),
	(0.000000, 0.073171),
	(0.000000, 0.027739),
	(0.000000, 0.044593),
	(0.000000, 0.029326),
	(0.000000, 0.206659),
	(0.000000, 0.128370),
	(0.000000, 0.058309),
	(0.000000, 0.023256),
	(0.000000, 0.098361),
	(0.000000, 0.106383),
	(0.000000, 0.114358),
	(0.000000, 0.099010),
	(0.000000, 0.000000),
	(0.000000, 0.084986),
	(0.000000, 0.324450),
	(0.000000, 0.080552),
	(0.000000, 0.067797),
	(0.000000, 0.203160),
	(0.000000, 0.215827),
	(0.000000, 0.086957),
	(0.000000, 0.011655),
	(0.000000, 0.241657),
	(0.000000, 0.182025),
	(0.000000, 0.076336),
	(0.000000, 0.399088),
	(0.000000, 0.013717),
	(0.000000, 0.065789),
	(0.000000, 0.011655),
	(0.000000, 0.125156),
	(0.000000, 0.044994),
	(0.000000, 0.215054),
	(0.000000, 0.022321),
	(0.000000, 0.025773),
	(0.000000, 0.000000),
	(0.000000, 0.828877),
	(0.000000, 0.030746),
	(0.000000, 0.000000),
	(0.000000, 0.281690),
	(0.000000, 0.103751),
	(0.000000, 0.067682),
	(0.000000, 0.000000),
	(0.000000, 0.000000),
	(0.000000, 0.008319),
	(0.000000, 0.094044),
	(0.000000, 1.784160),
	(0.000000, 0.149961),
	(0.000000, 0.026362),
	(0.000000, 0.157005),
	(0.000000, 0.009662),
	(0.000000, 0.094162),
	(0.000000, 0.000000),
	(0.000000, 0.008143),
	(0.000000, 0.000000),
	(0.000000, 0.098039),
	(0.000000, 0.035026),
	(0.000000, 0.000000),
	(0.000000, 0.008197),
	(0.000000, 0.016234),
	(0.000000, 0.048077),
	(0.000000, 0.026339),
	(0.000000, 0.055762),
	(0.000000, 0.095902),
	(0.000000, 0.009921),
	(0.000000, 0.038217),
	(0.000000, 0.027510),
	(0.000000, 0.045496),
	(0.000000, 0.072254),
	(0.000000, 0.086207),
	(0.000000, 0.060484),
	(0.000000, 0.080483),
	(0.000000, 0.000000),
	(0.000000, 0.020704),
	(0.000000, 0.020877),
	(0.000000, 0.039024),
	(0.000000, 0.302548),
	(0.000000, 0.023529),
	(0.000000, 0.098912),
	(0.000000, 0.484375),
	(0.000000, 0.245098),
	(0.000000, 0.049702),
	(0.000000, 1.469534),
	(0.000000, 0.030364),
	(0.000000, 0.253700),
	(0.000000, 0.098765),
	(0.000000, 0.373711),
	(0.000000, 0.069238),
	(0.000000, 0.010834),
	(0.000000, 0.032223),
	(0.000000, 0.010917),
	(0.000000, 0.084135),
	(0.000000, 0.093567),
	(0.000000, 0.048721),
	(0.000000, 0.148221),
	(0.000000, 0.136319),
	(0.000000, 0.092688),
	(0.000000, 0.042872),
	(0.000000, 0.161137),
	(0.000000, 0.000000),
	(0.000000, 0.045610),
	(0.000000, 0.034091),
	(0.000000, 0.317460),
	(0.000000, 0.175439),
	(0.000000, 0.039526),
	(0.000000, 0.065554),
	(0.000000, 0.077410),
	(0.000000, 0.023881),
	(0.000000, 0.013986),
	(0.000000, 0.054795),
	(0.000000, 0.021536),
	(0.000000, 0.057618),
	(0.000000, 0.046030),
	(0.000000, 0.044053),
	(0.000000, 0.006940),
	(0.000000, 0.041987),
	(0.000000, 0.069052),
	(0.000000, 0.010887),
	(0.000000, 0.065062),
	(0.000000, 0.019280),
	(0.000000, 0.017291),
	(0.000000, 0.038735),
	(0.000000, 0.099786),
	(0.000000, 0.014599),
	(0.000000, 0.024213),
	(0.000000, 0.116195),
	(0.000000, 0.126183),
	(0.000000, 0.000000),
	(0.000000, 0.098765),
	(0.000000, 1.064774),
	(0.000000, 0.024405),
	(0.000000, 0.049322),
	(0.000000, 0.085299),
	(0.000000, 0.035253),
	(0.000000, 0.178469),
	(0.000000, 0.091130),
	(0.000000, 0.027762),
	(0.000000, 0.000000),
	(0.000000, 0.052386),
	(0.000000, 0.035440),
	(0.000000, 0.138408),
	(0.000000, 0.023002),
	(0.000000, 0.011105),
	(0.000000, 0.018083),
	(0.000000, 0.030960),
	(0.000000, 0.005945),
	(0.000000, 0.000000),
	(0.000000, 0.083983),
	(0.000000, 0.038251),
	(0.000000, 0.061100),
	(0.000000, 0.077864),
	(0.000000, 0.168955),
	(0.000000, 0.049801),
	(0.000000, 0.365591),
	(0.000000, 0.234467),
	(0.000000, 0.032293),
	(0.000000, 0.084746),
	(0.000000, 0.180941),
	(0.000000, 0.035294),
	(0.000000, 0.228509),
	(0.000000, 0.169661),
	(0.000000, 0.044004),
	(0.000000, 0.063559),
	(0.000000, 0.129260),
	(0.000000, 0.184758),
	(0.000000, 0.125392),
	(0.000000, 0.061983),
	(0.000000, 0.049322),
	(0.000000, 0.069721),
	(0.000000, 0.436635),
	(0.000000, 0.342541),
	(0.000000, 0.080564),
	(0.000000, 0.205405),
	(0.000000, 0.031088),
	(0.000000, 0.033975),
	(0.000000, 0.198675),
	(0.000000, 0.174482),
	(0.000000, 0.059102),
	(0.000000, 0.000000),
	(0.000000, 0.024272),
	(0.000000, 0.123267),
	(0.000000, 0.035587),
	(0.000000, 0.258014),
	(0.000000, 0.125945),
	(0.000000, 0.000000),
	(0.000000, 0.361727),
	(0.000000, 0.006892),
	(0.000000, 0.235294),
	(0.000000, 0.045103),
	(0.000000, 0.065062),
	(0.000000, 0.039370),
	(0.000000, 0.000000),
	(0.000000, 0.022779),
	(0.000000, 0.089352),
	(0.000000, 0.007194),
	(0.000000, 0.176630),
	(0.000000, 0.000000),
	(0.000000, 0.048176),
	(0.000000, 0.025147),
	(0.000000, 0.144928),
	(0.000000, 0.019789),
	(0.000000, 0.041899),
	(0.000000, 0.214777),
	(0.000000, 0.000000),
	(0.000000, 0.033031),
	(0.000000, 0.052493),
	(0.000000, 0.296530),
	(0.000000, 0.000000),
	(0.000000, 0.091938),
	(0.000000, 0.022222),
	(0.000000, 0.014993),
	(0.000000, 0.035211),
	(0.000000, 0.000000),
	(0.000000, 0.067698),
	(0.000000, 0.042614),
	(0.000000, 0.000000),
	(0.000000, 0.035461),
	(0.000000, 0.042313),
	(0.000000, 0.033445),
	(0.000000, 0.186005),
	(0.000000, 0.120603),
	(0.000000, 0.000000),
	(0.000000, 0.000000),
	(0.000000, 0.007123),
	(0.000000, 0.151515),
	(0.000000, 0.064148),
	(0.000000, 0.029176),
	(0.000000, 0.128205),
	(0.000000, 0.078466),
	(0.000000, 0.265363),
	(0.000000, 0.078927),
	(0.000000, 0.210444),
	(0.000000, 0.073892),
	(0.000000, 0.030234),
	(0.000000, 0.000000),
	(0.000000, 0.042017),
	(0.000000, 0.032868),
	(0.000000, 1.295408),
	(0.000000, 0.000000),
	(0.000000, 0.033333),
	(0.000000, 0.358127),
	(0.000000, 0.020534),
	(0.000000, 0.084098),
	(0.000000, 0.053908),
	(0.000000, 0.018657),
	(0.000000, 0.039714),
	(0.000000, 0.219058),
	(0.000000, 0.000000),
	(0.000000, 0.055556),
	(0.000000, 0.056848),
	(0.000000, 0.150820),
	(0.000000, 0.466258),
	(0.000000, 0.032206),
	(0.000000, 0.009506),
	(0.000000, 0.055835),
	(0.000000, 0.026709),
	(0.000000, 0.045501),
	(0.000000, 0.006200),
	(0.000000, 0.006618),
	(0.000000, 0.085919),
	(0.000000, 0.083056),
	(0.000000, 0.101744),
	(0.000000, 0.114823),
	(0.000000, 0.055586),
	(0.000000, 0.000000),
	(0.000000, 0.060484),
	(0.000000, 0.036630),
	(0.000000, 0.035336),
	(0.000000, 0.056243),
	(0.000000, 0.062470),
	(0.000000, 0.000000),
	(0.000000, 0.136674),
	(0.000000, 0.170562),
	(0.000000, 0.089333),
	(0.000000, 0.032362),
	(0.000000, 0.048368),
	(0.000000, 0.168734),
	(0.000000, 0.024010),
	(0.000000, 0.000000),
	(0.000000, 0.058366),
	(0.000000, 0.217510),
	(0.000000, 0.023810),
	(0.000000, 0.036439),
	(0.000000, 0.033512),
	(0.000000, 0.060080),
	(0.000000, 0.000000),
	(0.000000, 0.099850),
	(0.000000, 0.021796),
	(0.000000, 0.021019),
	(0.000000, 0.057054),
	(0.000000, 0.051643),
	(0.000000, 0.038369),
	(0.000000, 0.016639),
	(0.000000, 0.009355),
	(0.000000, 0.028329),
	(0.000000, 0.010010),
	(0.000000, 0.004560),
	(0.000000, 0.017452),
	(0.000000, 0.137348),
	(0.000000, 0.085511),
	(0.000000, 0.176751),
	(0.000000, 0.044910),
	(0.000000, 0.037861),
	(0.000000, 0.061767),
	(0.000000, 0.014903),
	(0.000000, 0.148699),
	(0.000000, 0.011783),
	(0.000000, 0.000000),
	(0.000000, 0.035642),
	(0.000000, 0.037234),
	(0.000000, 0.186916),
	(0.000000, 0.000000),
	(0.000000, 0.013239),
	(0.000000, 0.000000),
	(0.000000, 0.024522),
	(0.000000, 0.013774),
	(0.000000, 0.030196),
	(0.000000, 1.549925),
	(0.000000, 0.022472),
	(0.000000, 0.027042),
	(0.000000, 0.120874),
	(0.000000, 0.054735),
	(0.000000, 0.044204),
	(0.000000, 0.020704),
	(0.000000, 0.000000),
	(0.000000, 0.063796),
	(0.000000, 0.004914),
	(0.000000, 0.075717),
	(0.000000, 0.000000),
	(0.000000, 0.054585),
	(0.000000, 0.075594),
	(0.000000, 0.140755),
	(0.000000, 0.013316),
	(0.000000, 0.115668),
	(0.000000, 0.058617),
	(0.000000, 0.042373),
	(0.000000, 1.369863),
	(0.000000, 0.013928),
	(0.000000, 0.000000),
	(0.000000, 0.010428),
	(0.000000, 0.024155),
	(0.000000, 0.111732),
	(0.000000, 0.144840),
	(0.000000, 0.028952),
	(0.000000, 0.078431),
	(0.000000, 0.129630),
	(0.000000, 0.254630),
	(0.000000, 0.174419),
	(0.000000, 0.031427),
	(0.000000, 0.046620),
	(0.000000, 0.322581),
} \fill \pos circle(0.03);
\draw (0,0) -- (5, 5);
\end{tikzpicture}
}
\only<+>{\charttitle{\timeC} \begin{tikzpicture}

\draw (0,0) -- (5,0);
\node at (2.5,-0.6) {LU};
\node [anchor=north] at (0.833333333333333,0) {\tiny 10};
\draw (0.833333333333333,0) -- (0.833333333333333,0.1);
\draw [style=help lines] (0.833333333333333,0) -- (0.833333333333333,5);
\node [anchor=north] at (1.66666666666667,0) {\tiny 20};
\draw (1.66666666666667,0) -- (1.66666666666667,0.1);
\draw [style=help lines] (1.66666666666667,0) -- (1.66666666666667,5);
\node [anchor=north] at (2.5,0) {\tiny 30};
\draw (2.5,0) -- (2.5,0.1);
\draw [style=help lines] (2.5,0) -- (2.5,5);
\node [anchor=north] at (3.33333333333333,0) {\tiny 40};
\draw (3.33333333333333,0) -- (3.33333333333333,0.1);
\draw [style=help lines] (3.33333333333333,0) -- (3.33333333333333,5);
\node [anchor=north] at (4.16666666666667,0) {\tiny 50};
\draw (4.16666666666667,0) -- (4.16666666666667,0.1);
\draw [style=help lines] (4.16666666666667,0) -- (4.16666666666667,5);
\node [anchor=north] at (5,0) {\tiny 60};
\draw (5,0) -- (5,0.1);
\draw [style=help lines] (5,0) -- (5,5);
\draw (0,0) -- (0,5);
\node [rotate=90] at (-2.5em,2.5) {LUniv};
\node [anchor=east] at (0,0.832333333333333) {\tiny 10};
\draw (0,0.832333333333333) -- (0.1,0.832333333333333);
\draw [style=help lines] (0,0.832333333333333) -- (5,0.832333333333333);
\node [anchor=east] at (0,1.66466666666667) {\tiny 20};
\draw (0,1.66466666666667) -- (0.1,1.66466666666667);
\draw [style=help lines] (0,1.66466666666667) -- (5,1.66466666666667);
\node [anchor=east] at (0,2.497) {\tiny 30};
\draw (0,2.497) -- (0.1,2.497);
\draw [style=help lines] (0,2.497) -- (5,2.497);
\node [anchor=east] at (0,3.32933333333333) {\tiny 40};
\draw (0,3.32933333333333) -- (0.1,3.32933333333333);
\draw [style=help lines] (0,3.32933333333333) -- (5,3.32933333333333);
\node [anchor=east] at (0,4.16166666666667) {\tiny 50};
\draw (0,4.16166666666667) -- (0.1,4.16166666666667);
\draw [style=help lines] (0,4.16166666666667) -- (5,4.16166666666667);
\node [anchor=east] at (0,4.994) {\tiny 60};
\draw (0,4.994) -- (0.1,4.994);
\draw [style=help lines] (0,4.994) -- (5,4.994);
\foreach \pos in {
	(0.030616, 0.027014),
	(0.005489, 0.011754),
	(0.007831, 0.012924),
	(0.001396, 0.003488),
	(0.002915, 0.006156),
	(0.000971, 0.000544),
	(0.002533, 0.005951),
	(0.000286, 0.000439),
	(0.003311, 0.006042),
	(0.002379, 0.004358),
	(0.002219, 0.005047),
	(0.002829, 0.006344),
	(0.002392, 0.004250),
	(0.003187, 0.006347),
	(0.017565, 0.029127),
	(0.002162, 0.003640),
	(0.001585, 0.003154),
	(0.002329, 0.004227),
	(0.002591, 0.005588),
	(0.003876, 0.006938),
	(0.004582, 0.011088),
	(0.002544, 0.004915),
	(0.002909, 0.005476),
	(0.002364, 0.004880),
	(0.002624, 0.004733),
	(0.002341, 0.003935),
	(0.001612, 0.003768),
	(0.002573, 0.003998),
	(0.002253, 0.004557),
	(0.002599, 0.004439),
	(0.002275, 0.004635),
	(0.001347, 0.002749),
	(0.002793, 0.005374),
	(0.000943, 0.002447),
	(0.002280, 0.004794),
	(0.002074, 0.003762),
	(0.001112, 0.002858),
	(0.000402, 0.002088),
	(0.001375, 0.003141),
	(0.000435, 0.002025),
	(0.001509, 0.002277),
	(0.001328, 0.003168),
	(0.001350, 0.003126),
	(0.001844, 0.003904),
	(0.001259, 0.002509),
	(0.000599, 0.001448),
	(0.001667, 0.003995),
	(0.002103, 0.004133),
	(0.001186, 0.002853),
	(0.002609, 0.005418),
	(0.001872, 0.003682),
	(0.001443, 0.003497),
	(0.001111, 0.002187),
	(0.001272, 0.003265),
	(0.001338, 0.003187),
	(0.000872, 0.001953),
	(0.001968, 0.003666),
	(0.002488, 0.000591),
	(0.002859, 0.001451),
	(0.001013, 0.002165),
	(0.000865, 0.001791),
	(0.001310, 0.003260),
	(0.000655, 0.001411),
	(0.001068, 0.002071),
	(0.001230, 0.002900),
	(0.000245, 0.000751),
	(0.001523, 0.002630),
	(0.000839, 0.001707),
	(0.000582, 0.002090),
	(0.000532, 0.001472),
	(0.000293, 0.000646),
	(0.000886, 0.001886),
	(0.000828, 0.001501),
	(0.000358, 0.001044),
	(0.000497, 0.001281),
	(0.001112, 0.002443),
	(0.000614, 0.001608),
	(0.000507, 0.000766),
	(0.000765, 0.001859),
	(0.001062, 0.001905),
	(0.000101, 0.000278),
	(0.000567, 0.001524),
	(0.000328, 0.000880),
	(0.000486, 0.001052),
	(0.000453, 0.000972),
	(0.000247, 0.000761),
	(0.000618, 0.001406),
	(0.000687, 0.001304),
	(0.000450, 0.001079),
	(0.000343, 0.000799),
	(0.000506, 0.000228),
	(0.000302, 0.000574),
	(0.000310, 0.000157),
	(0.000311, 0.000156),
	(0.000202, 0.000488),
	(0.000086, 0.000206),
	(0.001041, 0.001476),
	(0.006299, 0.007518),
	(0.000045, 0.000118),
	(0.000051, 0.000116),
	(0.000114, 0.000131),
	(0.000069, 0.000177),
	(0.000066, 0.000172),
	(0.000065, 0.000172),
	(0.006673, 0.002255),
	(0.000209, 0.000327),
	(0.009808, 0.003622),
	(0.003121, 0.001366),
	(0.003129, 0.001349),
	(0.002410, 0.000928),
	(0.000634, 0.000398),
	(0.000924, 0.000832),
	(0.000648, 0.000383),
	(0.000595, 0.000593),
	(0.000704, 0.001000),
	(0.000184, 0.000173),
	(0.000335, 0.000242),
	(0.000295, 0.000310),
	(0.000130, 0.000150),
	(0.000047, 0.000053),
	(0.000154, 0.000128),
	(0.000046, 0.000072),
	(0.000100, 0.000133),
	(0.000035, 0.000024),
	(0.000033, 0.000024),
	(0.000067, 0.000054),
	(0.000041, 0.000029),
	(0.000015, 0.000013),
	(0.000052, 0.000055),
	(0.000022, 0.000016),
	(0.000001, 0.000002),
	(0.000046, 0.000034),
	(0.000059, 0.000028),
	(0.000015, 0.000021),
	(0.000015, 0.000012),
	(0.000038, 0.000016),
	(0.000028, 0.000013),
	(0.000019, 0.000019),
	(0.000006, 0.000008),
	(0.000001, 0.000002),
	(0.000003, 0.000004),
	(0.000005, 0.000004),
	(0.000011, 0.000004),
	(0.058247, 0.079494),
	(0.020564, 0.026627),
	(0.009432, 0.017349),
	(0.015166, 0.029001),
	(0.010798, 0.016090),
	(0.032749, 0.058966),
	(0.013225, 0.023505),
	(0.014026, 0.027454),
	(0.005889, 0.018625),
	(0.041075, 0.064270),
	(0.010616, 0.019775),
	(0.004088, 0.016361),
	(0.004882, 0.010657),
	(0.003397, 0.010454),
	(0.026801, 0.047691),
	(0.002497, 0.003343),
	(0.015112, 0.024689),
	(0.010596, 0.017012),
	(0.017399, 0.032042),
	(0.018614, 0.040321),
	(0.008851, 0.015978),
	(0.008395, 0.015715),
	(0.027121, 0.040980),
	(0.002515, 0.012289),
	(0.020963, 0.046524),
	(0.013947, 0.017877),
	(0.013264, 0.021187),
	(0.025591, 0.039841),
	(0.009363, 0.016493),
	(0.003842, 0.011185),
	(0.007747, 0.014130),
	(0.005894, 0.018787),
	(0.010471, 0.021384),
	(0.014933, 0.025403),
	(0.006073, 0.011691),
	(0.000643, 0.002725),
	(0.013652, 0.020301),
	(0.031233, 0.040032),
	(0.013559, 0.025402),
	(0.005491, 0.008814),
	(0.006568, 0.012414),
	(0.005289, 0.011492),
	(0.008669, 0.015789),
	(0.010362, 0.017955),
	(0.017877, 0.032620),
	(0.021480, 0.027311),
	(0.014946, 0.024350),
	(0.007781, 0.016847),
	(0.022583, 0.039254),
	(0.005167, 0.010536),
	(0.013597, 0.021909),
	(0.011441, 0.021342),
	(0.006358, 0.011261),
	(0.003626, 0.014324),
	(0.000944, 0.003131),
	(0.007043, 0.019380),
	(0.004040, 0.008711),
	(0.006936, 0.012248),
	(0.012142, 0.021666),
	(0.024100, 0.043912),
	(0.006250, 0.015880),
	(0.003161, 0.007499),
	(0.005842, 0.008667),
	(0.001651, 0.007579),
	(0.017227, 0.018474),
	(0.011923, 0.018026),
	(0.006732, 0.012269),
	(0.001421, 0.002451),
	(0.000145, 0.000379),
	(0.003402, 0.012645),
	(0.005521, 0.010923),
	(0.008332, 0.015748),
	(0.007818, 0.018155),
	(0.008380, 0.002941),
	(0.002421, 0.005622),
	(0.014818, 0.020324),
	(0.003694, 0.007094),
	(0.003154, 0.007085),
	(0.006721, 0.013027),
	(0.001797, 0.004728),
	(0.004501, 0.008486),
	(0.007062, 0.012929),
	(0.010848, 0.013096),
	(0.006474, 0.011186),
	(0.008129, 0.016139),
	(0.008515, 0.014552),
	(0.008204, 0.011753),
	(0.002143, 0.005473),
	(0.005835, 0.010689),
	(0.006158, 0.010329),
	(0.006213, 0.013138),
	(0.002932, 0.006898),
	(0.002175, 0.008903),
	(0.005158, 0.011074),
	(0.006251, 0.013112),
	(0.005298, 0.011183),
	(0.002030, 0.004331),
	(0.004908, 0.010884),
	(0.011676, 0.018908),
	(0.001355, 0.003259),
	(0.002983, 0.015889),
	(0.009396, 0.016612),
	(0.005345, 0.009886),
	(0.006808, 0.011125),
	(0.002188, 0.005382),
	(0.014112, 0.024637),
	(0.002795, 0.004858),
	(0.005142, 0.016926),
	(0.004242, 0.007309),
	(0.012337, 0.020310),
	(0.003474, 0.012672),
	(0.003249, 0.006443),
	(0.003440, 0.005805),
	(0.003852, 0.007614),
	(0.005514, 0.010546),
	(0.007097, 0.014325),
	(0.004870, 0.008369),
	(0.011069, 0.020894),
	(0.007818, 0.013994),
	(0.006103, 0.010951),
	(0.004487, 0.007423),
	(0.001778, 0.004107),
	(0.006165, 0.012128),
	(0.003174, 0.006356),
	(0.004971, 0.009516),
	(0.006810, 0.012526),
	(0.003967, 0.007373),
	(0.003823, 0.007392),
	(0.001512, 0.003321),
	(0.009198, 0.015055),
	(0.004136, 0.007774),
	(0.004880, 0.008251),
	(0.004704, 0.008934),
	(0.002080, 0.003839),
	(0.001687, 0.004310),
	(0.005294, 0.011493),
	(0.077417, 0.111438),
	(0.009336, 0.014425),
	(0.019741, 0.040254),
	(0.017246, 0.026227),
	(0.026514, 0.040575),
	(0.023521, 0.040644),
	(0.014690, 0.059238),
	(0.033624, 0.069422),
	(0.022277, 0.033602),
	(0.011944, 0.019917),
	(0.009895, 0.016988),
	(0.023995, 0.041972),
	(0.036544, 0.067075),
	(0.014425, 0.024915),
	(0.040077, 0.051418),
	(0.011237, 0.020110),
	(0.069021, 0.100946),
	(0.023573, 0.038216),
	(0.012831, 0.020605),
	(0.034918, 0.047549),
	(0.024755, 0.033991),
	(0.010110, 0.019306),
	(0.010339, 0.018767),
	(0.021949, 0.046577),
	(0.011199, 0.027178),
	(0.042201, 0.072396),
	(0.009227, 0.014649),
	(0.014379, 0.026205),
	(0.007363, 0.013956),
	(0.053188, 0.095886),
	(0.028559, 0.046873),
	(0.029060, 0.046568),
	(0.001580, 0.005068),
	(0.025264, 0.050682),
	(0.019176, 0.031914),
	(0.028119, 0.046695),
	(0.001178, 0.001634),
	(0.004655, 0.009437),
	(0.010872, 0.019008),
	(0.015116, 0.032311),
	(0.023127, 0.036035),
	(0.007164, 0.010411),
	(0.026031, 0.045120),
	(0.017157, 0.031131),
	(0.000640, 0.001127),
	(0.021114, 0.036773),
	(0.021814, 0.031541),
	(0.038619, 0.048080),
	(0.003312, 0.008268),
	(0.013822, 0.024826),
	(0.037455, 0.059220),
	(0.019178, 0.029367),
	(0.015835, 0.027630),
	(0.013493, 0.040951),
	(0.005402, 0.007659),
	(0.009404, 0.014758),
	(0.020812, 0.035002),
	(0.025331, 0.040769),
	(0.012622, 0.029276),
	(0.005702, 0.007952),
	(0.011914, 0.023460),
	(0.020861, 0.034196),
	(0.017813, 0.033086),
	(0.015268, 0.024931),
	(0.024038, 0.029576),
	(0.019888, 0.034759),
	(0.006815, 0.013186),
	(0.008692, 0.018417),
	(0.002686, 0.005868),
	(0.013821, 0.024759),
	(0.017218, 0.022901),
	(0.011881, 0.020072),
	(0.020462, 0.034712),
	(0.013079, 0.018236),
	(0.036634, 0.047324),
	(0.015561, 0.026884),
	(0.016193, 0.027903),
	(0.006820, 0.012036),
	(0.007791, 0.014439),
	(0.032647, 0.050868),
	(0.004933, 0.008319),
	(0.013393, 0.020944),
	(0.003562, 0.007870),
	(0.022509, 0.037802),
	(0.013339, 0.021585),
	(0.009738, 0.015225),
	(0.014484, 0.026385),
	(0.008396, 0.022573),
	(0.016096, 0.020049),
	(0.018421, 0.026130),
	(0.030831, 0.040368),
	(0.040264, 0.064302),
	(0.013714, 0.025021),
	(0.023175, 0.048925),
	(0.017702, 0.034417),
	(0.008305, 0.017432),
	(0.012578, 0.020330),
	(0.017208, 0.021482),
	(0.009722, 0.017388),
	(0.015969, 0.022904),
	(0.024292, 0.040964),
	(0.004289, 0.008879),
	(0.008161, 0.017067),
	(0.012542, 0.022032),
	(0.018594, 0.028556),
	(0.009709, 0.018131),
	(0.011319, 0.021393),
	(0.015851, 0.031198),
	(0.012816, 0.022950),
	(0.011640, 0.019047),
	(0.014849, 0.028463),
	(0.035400, 0.064637),
	(0.007588, 0.015838),
	(0.005403, 0.009849),
	(0.004349, 0.008152),
	(0.020970, 0.038003),
	(0.010373, 0.016931),
	(0.014302, 0.021525),
	(0.002258, 0.004739),
	(0.004191, 0.007450),
	(0.012535, 0.022484),
	(0.010716, 0.026072),
	(0.008454, 0.015464),
	(0.007335, 0.012770),
	(0.005304, 0.013092),
	(0.001525, 0.004522),
	(0.010351, 0.017098),
	(0.027853, 0.046189),
	(0.007481, 0.011774),
	(0.067866, 0.099739),
	(0.044893, 0.063502),
	(0.015154, 0.027908),
	(0.030461, 0.052212),
	(0.033815, 0.054143),
	(0.026384, 0.044070),
	(0.019659, 0.036426),
	(0.055516, 0.090915),
	(0.046171, 0.068430),
	(0.022331, 0.038853),
	(0.039712, 0.064930),
	(0.012014, 0.023508),
	(0.055534, 0.087142),
	(0.039812, 0.065928),
	(0.013188, 0.023490),
	(0.039122, 0.074343),
	(0.060285, 0.094174),
	(0.019608, 0.034845),
	(0.024123, 0.042189),
	(0.007594, 0.015478),
	(0.025197, 0.041710),
	(0.016372, 0.031644),
	(0.015961, 0.025230),
	(0.034897, 0.050259),
	(0.004730, 0.011438),
	(0.035835, 0.056560),
	(0.049279, 0.070873),
	(0.008256, 0.013307),
	(0.031953, 0.048124),
	(0.035805, 0.049768),
	(0.021141, 0.032147),
	(0.040859, 0.049190),
	(0.043598, 0.066205),
	(0.076438, 0.121276),
	(0.028009, 0.055739),
	(0.023425, 0.050388),
	(0.022324, 0.047233),
	(0.015830, 0.028185),
	(0.005800, 0.011300),
	(0.036200, 0.056701),
	(0.031116, 0.049403),
	(0.050484, 0.072915),
	(0.022939, 0.040299),
	(0.026311, 0.043885),
	(0.026215, 0.044513),
	(0.106671, 0.148229),
	(0.041115, 0.077860),
	(0.001646, 0.002580),
	(0.038176, 0.059417),
	(0.030616, 0.048243),
	(0.051490, 0.076404),
	(0.047557, 0.089585),
	(0.014246, 0.025676),
	(0.018540, 0.034263),
	(0.042535, 0.068375),
	(0.005587, 0.015506),
	(0.017565, 0.027296),
	(0.032449, 0.054514),
	(0.014559, 0.026307),
	(0.023783, 0.050717),
	(0.021634, 0.038870),
	(0.026702, 0.037452),
	(0.013714, 0.024447),
	(0.021707, 0.032931),
	(0.014488, 0.026480),
	(0.004679, 0.007816),
	(0.029077, 0.044113),
	(0.030905, 0.065813),
	(0.019585, 0.032982),
	(0.019011, 0.033574),
	(0.031003, 0.048603),
	(0.008756, 0.016616),
	(0.015000, 0.025310),
	(0.015073, 0.033043),
	(0.013811, 0.026080),
	(0.015884, 0.040977),
	(0.021094, 0.038410),
	(0.019272, 0.028002),
	(0.040289, 0.064885),
	(0.011920, 0.022710),
	(0.026197, 0.042932),
	(0.013700, 0.028389),
	(0.056190, 0.074636),
	(0.020228, 0.043366),
	(0.004215, 0.007518),
	(0.006519, 0.018546),
	(0.048705, 0.080392),
	(0.046893, 0.083460),
	(0.029904, 0.044040),
	(0.027445, 0.053946),
	(0.014231, 0.023584),
	(0.027851, 0.039530),
	(0.003050, 0.007635),
	(0.024025, 0.033829),
	(0.042332, 0.073551),
	(0.026752, 0.046067),
	(0.025274, 0.046619),
	(0.027679, 0.039976),
	(0.016551, 0.025347),
	(0.072063, 0.105283),
	(0.042528, 0.061865),
	(0.035342, 0.042963),
	(0.051031, 0.091616),
	(0.014567, 0.028492),
	(0.040743, 0.059652),
	(0.023709, 0.033211),
	(0.008728, 0.016811),
	(0.012085, 0.020348),
	(0.014842, 0.024539),
	(0.012768, 0.018596),
	(0.066995, 0.093537),
	(0.036367, 0.056783),
	(0.048682, 0.067908),
	(0.069912, 0.103195),
	(0.016569, 0.026827),
	(0.022096, 0.055012),
	(0.030518, 0.044929),
	(0.006956, 0.010538),
	(0.050561, 0.073203),
	(0.004799, 0.012435),
	(0.030529, 0.045209),
	(0.015269, 0.028107),
	(0.115273, 0.161027),
	(0.005389, 0.014379),
	(0.070876, 0.125894),
	(0.041017, 0.060849),
	(0.044099, 0.088459),
	(0.042773, 0.056795),
	(0.045011, 0.070369),
	(0.016335, 0.020687),
	(0.029330, 0.047873),
	(0.025790, 0.053272),
	(0.033200, 0.083806),
	(0.058465, 0.087573),
	(0.055160, 0.087468),
	(0.017485, 0.027827),
	(0.014474, 0.028455),
	(0.008428, 0.015675),
	(0.031030, 0.048515),
	(0.069635, 0.131965),
	(0.023288, 0.037909),
	(0.021820, 0.037677),
	(0.003302, 0.005889),
	(0.061935, 0.108530),
	(0.054730, 0.104479),
	(0.085761, 0.133711),
	(0.013692, 0.023771),
	(0.049378, 0.084899),
	(0.014424, 0.028145),
	(0.034360, 0.062908),
	(0.079963, 0.125684),
	(0.048571, 0.098976),
	(0.044574, 0.086256),
	(0.011325, 0.016616),
	(0.026589, 0.044126),
	(0.032364, 0.052959),
	(0.097679, 0.153949),
	(0.030023, 0.043228),
	(0.025443, 0.041303),
	(0.022994, 0.037841),
	(0.025490, 0.036474),
	(0.038984, 0.058959),
	(0.048242, 0.079979),
	(0.040292, 0.080281),
	(0.017136, 0.032121),
	(0.027454, 0.049239),
	(0.051327, 0.093572),
	(0.060408, 0.088159),
	(0.056109, 0.083860),
	(0.054834, 0.100686),
	(0.030586, 0.045343),
	(0.045755, 0.076604),
	(0.026328, 0.047029),
	(0.025872, 0.054697),
	(0.060665, 0.114992),
	(0.025260, 0.036019),
	(0.024635, 0.044982),
	(0.055906, 0.083154),
	(0.070436, 0.116822),
	(0.048606, 0.075664),
	(0.020777, 0.038885),
	(0.057421, 0.085248),
	(0.019275, 0.032030),
	(0.014438, 0.024852),
	(0.025925, 0.041535),
	(0.025019, 0.044794),
	(0.035700, 0.068340),
	(0.015846, 0.020774),
	(0.016968, 0.027254),
	(0.036665, 0.046100),
	(0.024161, 0.038988),
	(0.030038, 0.043364),
	(0.045974, 0.071449),
	(0.045555, 0.081050),
	(0.012139, 0.022796),
	(0.009097, 0.016665),
	(0.036450, 0.074958),
	(0.023697, 0.042336),
	(0.017931, 0.030057),
	(0.045035, 0.073933),
	(0.101152, 0.167494),
	(0.005756, 0.009873),
	(0.028524, 0.042829),
	(0.022629, 0.041620),
	(0.075505, 0.127736),
	(0.040561, 0.051220),
	(0.012996, 0.018705),
	(0.058083, 0.115607),
	(0.028251, 0.047533),
	(0.031964, 0.048124),
	(0.063062, 0.083122),
	(0.031159, 0.052061),
	(0.029118, 0.042483),
	(0.039461, 0.075472),
	(0.027640, 0.045943),
	(0.026453, 0.044645),
	(0.016089, 0.025946),
	(0.009978, 0.022709),
	(0.003213, 0.007179),
	(0.040209, 0.061792),
	(0.025852, 0.033053),
	(0.024619, 0.039055),
	(0.021429, 0.037016),
	(0.017247, 0.029839),
	(0.020407, 0.035234),
	(0.000086, 0.000163),
	(0.074596, 0.128968),
	(0.027216, 0.046220),
	(0.051996, 0.076792),
	(0.024941, 0.047351),
	(0.019397, 0.043917),
	(0.046249, 0.083970),
	(0.026163, 0.036721),
	(0.021465, 0.027253),
	(0.002707, 0.003794),
	(0.050415, 0.080841),
	(0.021411, 0.040570),
	(0.049283, 0.065216),
	(0.030984, 0.058319),
	(0.069056, 0.116469),
	(0.128812, 0.176450),
	(0.032935, 0.051546),
	(0.130570, 0.201012),
	(0.105753, 0.179801),
	(0.020261, 0.035980),
	(0.013726, 0.039955),
	(0.147680, 0.237003),
	(0.040601, 0.068774),
	(0.026626, 0.053464),
	(0.012735, 0.018553),
	(0.052575, 0.077197),
	(0.025528, 0.040975),
	(0.056778, 0.106303),
	(0.003107, 0.008242),
	(0.027383, 0.045222),
	(0.062943, 0.100600),
	(0.095654, 0.140715),
	(0.105477, 0.189027),
	(0.145923, 0.222292),
	(0.017108, 0.028255),
	(0.093307, 0.169873),
	(0.085456, 0.163779),
	(0.025793, 0.045397),
	(0.091707, 0.147890),
	(0.050441, 0.065146),
	(0.075903, 0.111370),
	(0.026690, 0.047785),
	(0.012110, 0.020634),
	(0.013187, 0.027757),
	(0.118674, 0.175919),
	(0.029853, 0.051213),
	(0.051984, 0.079893),
	(0.023315, 0.040105),
	(0.099571, 0.173044),
	(0.061604, 0.115258),
	(0.026637, 0.042977),
	(0.089486, 0.148757),
	(0.012996, 0.027858),
	(0.051187, 0.105898),
	(0.040997, 0.074966),
	(0.069997, 0.090483),
	(0.005173, 0.010291),
	(0.102477, 0.146069),
	(0.010068, 0.022020),
	(0.040084, 0.071643),
	(0.031457, 0.057938),
	(0.094488, 0.142729),
	(0.045367, 0.079092),
	(0.030184, 0.053963),
	(0.027014, 0.036477),
	(0.032833, 0.048893),
	(0.091532, 0.138142),
	(0.043074, 0.076800),
	(0.051010, 0.074350),
	(0.057439, 0.103881),
	(0.039257, 0.063000),
	(0.114955, 0.186273),
	(0.041118, 0.058041),
	(0.043880, 0.083901),
	(0.138433, 0.203213),
	(0.031623, 0.047832),
	(0.099935, 0.165304),
	(0.034978, 0.059683),
	(0.041685, 0.074983),
	(0.119330, 0.207499),
	(0.064491, 0.093357),
	(0.151694, 0.209926),
	(0.042274, 0.067691),
	(0.031992, 0.051049),
	(0.054026, 0.070848),
	(0.015152, 0.036392),
	(0.041482, 0.057579),
	(0.084267, 0.146591),
	(0.022202, 0.044291),
	(0.014874, 0.026627),
	(0.048322, 0.079154),
	(0.021569, 0.034175),
	(0.063856, 0.098846),
	(0.055429, 0.090376),
	(0.013576, 0.049786),
	(0.028621, 0.048402),
	(0.011026, 0.019457),
	(0.026704, 0.050309),
	(0.062661, 0.096166),
	(0.057014, 0.093363),
	(0.117261, 0.190538),
	(0.038508, 0.059624),
	(0.073800, 0.113329),
	(0.040861, 0.063720),
	(0.085082, 0.113467),
	(0.073637, 0.118187),
	(0.016892, 0.029908),
	(0.097475, 0.163323),
	(0.056215, 0.098189),
	(0.070150, 0.106149),
	(0.040600, 0.057885),
	(0.062949, 0.112057),
	(0.110489, 0.157426),
	(0.026035, 0.039259),
	(0.062811, 0.093795),
	(0.036895, 0.051848),
	(0.035488, 0.052247),
	(0.043032, 0.056369),
	(0.061380, 0.097032),
	(0.035849, 0.048333),
	(0.209669, 0.362725),
	(0.078976, 0.096962),
	(0.002331, 0.005463),
	(0.049582, 0.075244),
	(0.150400, 0.032945),
	(0.265093, 0.404965),
	(0.304412, 0.436535),
	(0.363038, 0.513812),
	(0.055800, 0.074986),
	(0.222468, 0.253536),
	(0.037097, 0.061760),
	(0.041251, 0.087586),
	(0.031201, 0.039858),
	(0.208644, 0.334245),
	(0.388780, 0.524589),
	(0.029802, 0.044420),
	(0.131827, 0.256556),
	(0.232417, 0.392169),
	(0.235623, 0.396478),
	(0.263875, 0.373531),
	(0.243279, 0.398231),
	(0.184130, 0.329010),
	(0.228874, 0.363103),
	(0.082471, 0.106086),
	(0.090081, 0.137957),
	(0.358909, 0.533487),
	(0.137654, 0.171458),
	(0.054116, 0.082455),
	(0.062954, 0.091153),
	(0.059266, 0.091678),
	(0.040083, 0.063359),
	(0.011435, 0.021994),
	(0.022761, 0.048044),
	(0.104097, 0.127581),
	(0.008837, 0.015641),
	(0.022902, 0.041450),
	(0.239982, 0.375324),
	(0.197487, 0.296414),
	(0.024157, 0.029398),
	(0.045492, 0.074633),
	(0.020116, 0.031356),
	(0.014246, 0.018928),
	(0.006852, 0.015767),
	(0.001821, 0.002733),
	(0.074710, 0.117812),
	(0.027696, 0.048060),
	(0.016270, 0.025964),
	(0.064343, 0.097820),
	(0.023410, 0.028293),
	(0.047781, 0.103930),
	(0.001478, 0.002582),
	(0.061276, 0.093125),
	(0.056034, 0.089624),
	(0.083370, 0.176880),
	(0.101563, 0.175113),
	(0.136435, 0.206993),
	(0.078265, 0.135566),
	(0.099196, 0.158517),
	(0.060149, 0.092966),
	(0.049609, 0.088472),
	(0.032249, 0.064543),
	(0.044116, 0.068394),
	(0.050652, 0.076078),
	(0.088766, 0.164794),
	(0.022382, 0.033225),
	(0.091693, 0.175394),
	(0.092899, 0.154726),
	(0.057208, 0.089354),
	(0.070296, 0.115036),
	(0.112209, 0.174825),
	(0.045716, 0.076555),
	(0.347671, 0.574052),
	(0.263006, 0.406203),
	(0.094839, 0.172418),
	(0.753755, 1.166154),
	(0.150322, 0.214381),
	(0.747419, 1.138938),
	(0.086464, 0.155595),
	(0.263394, 0.408335),
	(0.295940, 0.393014),
	(0.540434, 0.867943),
	(0.148506, 0.189155),
	(0.235827, 0.305773),
	(0.227017, 0.285701),
	(0.533750, 0.672718),
	(0.863374, 1.338244),
	(0.560989, 0.832355),
	(0.476079, 0.750169),
	(0.270405, 0.422794),
	(0.850857, 1.182957),
	(0.528766, 0.763152),
	(0.221845, 0.329790),
	(0.040759, 0.074370),
	(0.002671, 0.005899),
	(0.078030, 0.094814),
	(0.050432, 0.090173),
	(0.740871, 1.183773),
	(0.399344, 0.586820),
	(0.065696, 0.124621),
	(0.368713, 0.547153),
	(0.159485, 0.334892),
	(0.185439, 0.296272),
	(0.391068, 0.603750),
	(0.391443, 0.483631),
	(0.595743, 0.943367),
	(0.943383, 1.453713),
	(0.004504, 0.015246),
	(0.187445, 0.267718),
	(0.856434, 1.459693),
	(0.622691, 0.755504),
	(0.341921, 0.429615),
	(0.958224, 1.494887),
	(0.000022, 0.000023),
	(1.046370, 1.329923),
	(0.300926, 0.461467),
	(0.051445, 0.069421),
	(0.028032, 0.048422),
	(0.138497, 0.206174),
	(0.049733, 0.095847),
	(0.089792, 0.144198),
	(0.052779, 0.085658),
	(0.107833, 0.197980),
	(0.080314, 0.110717),
	(0.015500, 0.024497),
	(0.035056, 0.057168),
	(0.091910, 0.122205),
	(0.060713, 0.083353),
	(0.055936, 0.094526),
	(0.092392, 0.134034),
	(0.068337, 0.113116),
	(0.055511, 0.083844),
	(0.072487, 0.107337),
	(0.046653, 0.070463),
	(0.058123, 0.084707),
	(0.036313, 0.061186),
	(0.072430, 0.114536),
	(0.036457, 0.056814),
	(0.057337, 0.076842),
	(0.073581, 0.103713),
	(0.018470, 0.032486),
	(0.082554, 0.130572),
	(0.077317, 0.106208),
	(0.087426, 0.124866),
	(0.092131, 0.139062),
	(0.029528, 0.041966),
	(0.386976, 0.453560),
	(0.003243, 0.007060),
	(0.122708, 0.218417),
	(0.226114, 0.470842),
	(0.025180, 0.052546),
	(0.111597, 0.141795),
	(0.026740, 0.107244),
	(0.727351, 1.038481),
	(0.037807, 0.077530),
	(0.192156, 0.280712),
	(0.713588, 1.032731),
	(0.008821, 0.013695),
	(0.009546, 0.014068),
	(0.399487, 0.680031),
	(0.437659, 0.671263),
	(0.093731, 0.125249),
	(0.372451, 0.492529),
	(0.431415, 0.546427),
	(0.449883, 0.676649),
	(0.113461, 0.160063),
	(0.462166, 0.571840),
	(0.398977, 0.634697),
	(0.158990, 0.256925),
	(0.019975, 0.035499),
	(0.109047, 0.150349),
	(0.082930, 0.112258),
	(0.186204, 0.230418),
	(0.121023, 0.195311),
	(0.000236, 0.000296),
	(0.182452, 0.325575),
	(0.193576, 0.299992),
	(0.092257, 0.158723),
	(0.002325, 0.004829),
	(0.140472, 0.168538),
	(0.060731, 0.107383),
	(0.003267, 0.008637),
	(0.164106, 0.210767),
	(0.474527, 0.671613),
	(0.430193, 0.653534),
	(0.245408, 0.410586),
	(0.038660, 0.078543),
	(0.040296, 0.099411),
	(0.000034, 0.000057),
	(0.000402, 0.000390),
	(0.407399, 0.615969),
	(0.202409, 0.307828),
	(0.498165, 0.737897),
	(0.139452, 0.201789),
	(0.239365, 0.383001),
	(0.023864, 0.052418),
	(0.004744, 0.011620),
	(0.095268, 0.144468),
	(0.110942, 0.166020),
	(0.515763, 0.665069),
	(0.152107, 0.229650),
	(0.097784, 0.149785),
	(0.149832, 0.297773),
	(0.399177, 0.621028),
	(0.125363, 0.130781),
	(0.003381, 0.005116),
	(0.462628, 0.703974),
	(0.004095, 0.010708),
	(0.001260, 0.003303),
	(0.212206, 0.340127),
	(0.182784, 0.356449),
	(0.051452, 0.080480),
	(0.056846, 0.085694),
	(0.007120, 0.020714),
	(0.172399, 0.193990),
	(0.883008, 1.360845),
	(0.155753, 0.336930),
	(0.067613, 0.151926),
	(0.580673, 0.946877),
	(0.100373, 0.197225),
	(0.408839, 0.509503),
	(0.453156, 0.676580),
	(1.111010, 1.710455),
	(1.157454, 1.707398),
	(0.432478, 0.641690),
	(1.068299, 1.627406),
	(0.311787, 0.386883),
	(0.003929, 0.008446),
	(0.342078, 0.623023),
	(0.635440, 0.745417),
	(0.090011, 0.167797),
	(0.043376, 0.074097),
	(0.002216, 0.004787),
	(0.413999, 0.615619),
	(1.079444, 1.634215),
	(0.682610, 1.153213),
	(0.247973, 0.548172),
	(0.000108, 0.000170),
	(1.761591, 2.614114),
	(0.088845, 0.184010),
	(0.393914, 0.556443),
	(0.867313, 1.338837),
	(0.242457, 0.391938),
	(0.546749, 0.871650),
	(0.552118, 0.729145),
	(1.232528, 1.886630),
	(0.593817, 0.873545),
	(0.086452, 0.160230),
	(0.116969, 0.163544),
	(0.686489, 1.024105),
	(1.380011, 1.885527),
	(0.969538, 1.471725),
	(0.019568, 0.033048),
	(1.903531, 2.843917),
	(1.327674, 2.002561),
	(2.090880, 3.302461),
	(0.924142, 1.101620),
	(1.384052, 1.761113),
	(2.764128, 4.205609),
	(0.650668, 0.827668),
	(0.630840, 0.866307),
	(0.511390, 0.779068),
	(1.305395, 2.108846),
	(0.569966, 0.763978),
	(0.744012, 0.936754),
	(1.831106, 2.087262),
	(0.301254, 0.510494),
	(0.014877, 0.042407),
	(1.815696, 2.713520),
	(0.507525, 0.775077),
	(1.546060, 2.348033),
	(1.352121, 2.247830),
	(0.815056, 1.324963),
	(1.121136, 1.298913),
	(1.462091, 1.687617),
	(0.645973, 0.788185),
	(2.151935, 3.221429),
	(0.155781, 0.297224),
	(0.005943, 0.012849),
	(0.611020, 0.803802),
	(2.236514, 2.872524),
	(1.342512, 2.136099),
	(0.801470, 1.050610),
	(0.825684, 0.986245),
	(0.002713, 0.007650),
	(0.681746, 0.969923),
	(0.690218, 1.072724),
	(0.164260, 0.320576),
	(1.049600, 1.626599),
	(1.263153, 1.945710),
	(0.726782, 0.863341),
	(0.814868, 0.972412),
	(0.466652, 0.661684),
	(0.550480, 0.806777),
	(0.257991, 0.432030),
	(0.633304, 0.772577),
	(0.417096, 0.519948),
	(0.091185, 0.160634),
	(1.144745, 1.740808),
	(0.928229, 1.322086),
	(0.575324, 0.729082),
	(0.066059, 0.129971),
	(0.804115, 1.337431),
	(0.351733, 0.512606),
	(0.455050, 0.638505),
	(0.186763, 0.254927),
	(0.324412, 0.397876),
	(0.918508, 1.307403),
	(0.034307, 0.062928),
	(1.007585, 1.300234),
	(0.707991, 0.865249),
	(0.262837, 0.403946),
	(0.900057, 1.256092),
	(0.003483, 0.006775),
	(0.060388, 0.118606),
	(0.799604, 1.206859),
	(0.742016, 1.067167),
	(0.000073, 0.000122),
} \fill \pos circle(0.03);
\draw (0,0) -- (5, 5);
\end{tikzpicture}
}
\end{frame}

\appendix

\begin{frame}{Conclusion}
\charttitle*{Compression ratio comparison}
\renewcommand{\chartscale}{0.85}
\begin{columns}

  \column{0.45\textwidth}
  \centering
  \begin{tikzpicture}

\draw (0,0) -- (5,0);
\node at (2.5,-0.6) {LU.RPI};
\node [anchor=north] at (0.714285714285714,0) {\small 0.1};
\draw (0.714285714285714,0) -- (0.714285714285714,0.1);
\draw [style=help lines] (0.714285714285714,0) -- (0.714285714285714,5);
\node [anchor=north] at (1.42857142857143,0) {\small 0.2};
\draw (1.42857142857143,0) -- (1.42857142857143,0.1);
\draw [style=help lines] (1.42857142857143,0) -- (1.42857142857143,5);
\node [anchor=north] at (2.14285714285714,0) {\small 0.3};
\draw (2.14285714285714,0) -- (2.14285714285714,0.1);
\draw [style=help lines] (2.14285714285714,0) -- (2.14285714285714,5);
\node [anchor=north] at (2.85714285714286,0) {\small 0.4};
\draw (2.85714285714286,0) -- (2.85714285714286,0.1);
\draw [style=help lines] (2.85714285714286,0) -- (2.85714285714286,5);
\node [anchor=north] at (3.57142857142857,0) {\small 0.5};
\draw (3.57142857142857,0) -- (3.57142857142857,0.1);
\draw [style=help lines] (3.57142857142857,0) -- (3.57142857142857,5);
\node [anchor=north] at (4.28571428571429,0) {\small 0.6};
\draw (4.28571428571429,0) -- (4.28571428571429,0.1);
\draw [style=help lines] (4.28571428571429,0) -- (4.28571428571429,5);
\node [anchor=north] at (5,0) {\small 0.7};
\draw (5,0) -- (5,0.1);
\draw [style=help lines] (5,0) -- (5,5);
\draw (0,0) -- (0,5);
\node [rotate=90] at (-2.5em,2.5) {RPI.LU};
\node [anchor=east] at (0,0.713285714285714) {\small 0.1};
\draw (0,0.713285714285714) -- (0.1,0.713285714285714);
\draw [style=help lines] (0,0.713285714285714) -- (5,0.713285714285714);
\node [anchor=east] at (0,1.42657142857143) {\small 0.2};
\draw (0,1.42657142857143) -- (0.1,1.42657142857143);
\draw [style=help lines] (0,1.42657142857143) -- (5,1.42657142857143);
\node [anchor=east] at (0,2.13985714285714) {\small 0.3};
\draw (0,2.13985714285714) -- (0.1,2.13985714285714);
\draw [style=help lines] (0,2.13985714285714) -- (5,2.13985714285714);
\node [anchor=east] at (0,2.85314285714286) {\small 0.4};
\draw (0,2.85314285714286) -- (0.1,2.85314285714286);
\draw [style=help lines] (0,2.85314285714286) -- (5,2.85314285714286);
\node [anchor=east] at (0,3.56642857142857) {\small 0.5};
\draw (0,3.56642857142857) -- (0.1,3.56642857142857);
\draw [style=help lines] (0,3.56642857142857) -- (5,3.56642857142857);
\node [anchor=east] at (0,4.27971428571429) {\small 0.6};
\draw (0,4.27971428571429) -- (0.1,4.27971428571429);
\draw [style=help lines] (0,4.27971428571429) -- (5,4.27971428571429);
\node [anchor=east] at (0,4.993) {\small 0.7};
\draw (0,4.993) -- (0.1,4.993);
\draw [style=help lines] (0,4.993) -- (5,4.993);

\foreach \pos in {
	(1.556186, 1.556186),
	(2.287890, 2.342277),
	(2.034038, 2.062508),
	(1.070455, 1.158038),
	(1.190476, 1.201201),
	(0.480619, 0.484526),
	(1.349544, 1.513678),
	(0.250054, 0.250054),
	(1.972249, 2.046426),
	(1.440906, 1.732451),
	(1.363287, 1.420891),
	(1.745928, 1.802097),
	(1.032574, 1.165638),
	(2.179622, 2.242647),
	(1.552430, 1.604737),
	(0.898964, 0.908736),
	(1.510241, 1.756547),
	(1.728723, 2.026342),
	(1.602943, 1.684679),
	(2.587883, 3.246073),
	(2.077456, 2.115927),
	(1.750266, 1.900922),
	(1.620730, 1.926436),
	(1.996380, 2.245221),
	(1.712040, 2.002058),
	(1.686183, 1.861827),
	(1.246753, 1.474026),
	(2.119953, 2.173218),
	(1.521739, 1.310559),
	(1.839907, 1.903352),
	(1.668702, 1.729752),
	(1.642036, 2.151067),
	(1.760131, 1.794242),
	(0.994575, 1.410488),
	(1.714286, 1.755952),
	(2.059437, 2.075078),
	(1.361273, 1.827776),
	(1.560806, 1.560806),
	(1.321892, 1.321892),
	(1.634847, 1.634847),
	(1.104901, 1.104901),
	(1.719177, 2.208315),
	(1.754729, 2.100457),
	(1.396348, 1.450054),
	(1.004318, 1.210813),
	(1.517165, 1.860465),
	(2.024264, 2.167394),
	(1.326885, 1.444692),
	(1.729911, 2.032844),
	(1.400308, 1.464543),
	(1.448652, 1.627940),
	(1.495197, 1.634645),
	(1.783416, 2.307409),
	(1.308216, 1.613466),
	(1.342857, 1.585714),
	(1.631117, 1.971680),
	(1.489621, 1.636142),
	(0.129359, 0.134983),
	(0.469764, 0.474354),
	(1.801242, 2.236025),
	(1.866475, 1.920316),
	(1.403908, 1.811800),
	(1.522052, 1.605944),
	(1.630435, 1.727484),
	(1.250955, 1.250955),
	(1.785714, 1.800115),
	(1.618804, 1.463325),
	(0.933442, 0.963880),
	(1.243622, 1.530612),
	(1.538951, 1.560178),
	(1.838755, 1.867044),
	(1.957224, 1.997579),
	(1.535598, 1.713975),
	(1.710076, 1.710076),
	(1.394286, 1.577143),
	(1.392857, 1.464286),
	(1.553288, 1.575964),
	(1.815062, 1.887304),
	(1.617681, 1.822739),
	(1.251476, 1.227863),
	(1.194162, 1.194162),
	(1.661753, 1.755562),
	(1.813616, 1.813616),
	(1.624030, 1.925841),
	(2.376477, 2.537594),
	(1.551459, 1.628264),
	(1.720210, 1.742994),
	(1.655052, 1.916376),
	(1.941610, 2.097506),
	(1.562500, 1.596841),
	(0.267681, 0.267681),
	(1.926164, 1.926164),
	(0.267857, 0.287698),
	(0.267857, 0.287698),
	(1.251618, 1.273198),
	(1.071429, 1.071429),
	(1.020408, 1.916980),
	(0.850181, 0.853528),
	(0.676692, 0.789474),
	(0.875576, 0.875576),
	(0.115830, 0.115830),
	(1.392857, 1.392857),
	(1.344086, 1.344086),
	(1.344086, 1.344086),
	(0.418443, 0.418443),
	(1.561022, 1.655629),
	(0.315770, 0.324863),
	(0.127767, 0.127767),
	(0.127767, 0.129112),
	(0.545889, 0.548326),
	(0.369922, 0.410277),
	(1.034964, 1.099314),
	(0.294922, 0.294922),
	(1.059771, 1.066836),
	(1.378245, 1.355370),
	(0.473934, 3.368314),
	(0.527024, 0.527024),
	(0.683230, 0.708075),
	(1.104323, 1.409774),
	(1.235231, 1.235231),
	(0.253593, 0.253593),
	(1.751152, 1.751152),
	(1.095994, 1.927438),
	(0.992063, 0.396825),
	(0.601504, 0.751880),
	(0.480769, 0.480769),
	(0.514801, 0.772201),
	(1.749271, 1.749271),
	(0.588235, 1.512605),
	(1.632653, 1.632653),
	(0.000000, 0.000000),
	(0.148810, 0.148810),
	(1.020408, 1.020408),
	(0.696864, 1.219512),
	(0.744048, 0.744048),
	(0.533049, 0.533049),
	(1.166181, 1.166181),
	(1.288056, 1.522248),
	(0.420168, 0.840336),
	(0.000000, 0.000000),
	(1.428571, 2.142857),
	(3.125000, 2.678571),
	(4.395604, 4.395604),
	(1.586670, 1.899454),
	(2.175244, 1.864130),
	(1.744072, 1.949833),
	(1.187591, 1.371096),
	(1.531900, 1.728143),
	(1.572865, 1.692539),
	(2.033262, 2.120792),
	(1.142061, 1.388778),
	(1.338802, 1.547748),
	(1.843231, 1.940670),
	(1.676689, 1.836503),
	(1.694212, 1.792318),
	(1.365061, 1.542395),
	(1.412464, 1.455830),
	(1.198943, 1.283614),
	(2.406748, 2.816901),
	(1.285925, 1.563488),
	(1.802911, 2.038230),
	(1.820728, 1.884921),
	(1.651348, 1.701979),
	(1.515290, 1.538110),
	(1.600731, 1.709805),
	(2.004808, 2.051342),
	(1.034028, 1.078029),
	(1.837379, 1.837379),
	(1.882698, 2.205523),
	(2.475999, 2.646262),
	(1.381034, 1.651504),
	(2.128173, 2.134208),
	(1.537839, 1.537839),
	(1.987146, 2.164359),
	(2.105427, 2.105427),
	(1.496313, 1.607786),
	(1.839767, 2.386194),
	(1.464179, 1.597045),
	(1.369963, 1.853480),
	(1.587302, 1.591711),
	(2.073922, 2.340862),
	(1.983941, 2.134493),
	(2.447045, 2.809749),
	(1.678257, 1.713848),
	(1.998149, 2.082282),
	(1.675048, 1.791912),
	(1.698166, 1.806559),
	(1.962624, 2.099460),
	(1.515869, 2.108006),
	(2.403414, 2.425876),
	(1.434531, 1.890005),
	(1.657754, 1.791444),
	(1.776185, 1.802866),
	(1.966279, 1.985603),
	(1.306973, 1.480685),
	(1.954115, 1.984127),
	(1.344390, 1.470034),
	(2.321249, 2.335578),
	(1.254181, 1.230291),
	(2.496633, 1.419248),
	(1.817867, 1.936585),
	(2.262094, 2.502839),
	(1.531737, 1.539613),
	(1.336779, 1.813564),
	(1.675898, 2.029221),
	(1.692737, 1.936119),
	(1.324194, 1.417666),
	(1.982212, 2.261445),
	(2.040241, 2.140845),
	(1.766600, 2.305835),
	(1.443093, 1.443093),
	(1.378106, 1.378106),
	(1.453373, 1.661706),
	(2.125066, 2.477596),
	(1.979658, 1.988277),
	(1.890915, 1.901705),
	(0.113773, 0.114527),
	(1.503759, 1.597744),
	(1.051709, 1.139351),
	(1.331558, 1.464714),
	(1.735834, 1.900439),
	(1.394155, 1.493321),
	(1.761834, 1.793674),
	(1.387363, 1.575092),
	(2.060762, 2.306489),
	(1.640146, 1.675705),
	(1.797424, 1.812061),
	(1.141527, 1.163693),
	(2.555408, 1.842842),
	(1.376804, 1.508977),
	(1.220488, 1.293851),
	(2.119514, 2.131964),
	(1.355534, 1.850706),
	(2.098404, 2.275112),
	(1.591441, 1.873536),
	(1.325848, 1.391543),
	(2.142560, 2.175203),
	(1.675557, 1.808905),
	(1.501966, 1.627785),
	(1.764079, 1.810678),
	(2.097527, 2.228958),
	(1.968380, 2.037667),
	(1.300044, 1.440390),
	(2.382885, 2.481447),
	(1.808395, 1.814443),
	(1.498423, 1.464624),
	(1.566475, 1.566475),
	(1.361139, 1.467283),
	(1.575290, 1.575290),
	(1.503759, 1.564395),
	(1.487100, 1.684185),
	(1.749971, 1.772847),
	(2.077654, 2.188168),
	(1.442929, 1.514716),
	(1.594286, 1.634286),
	(1.604295, 1.539212),
	(1.675406, 1.710577),
	(2.318548, 2.599366),
	(1.401230, 1.196172),
	(1.791675, 1.784863),
	(1.217965, 1.243339),
	(2.038509, 2.161272),
	(1.897547, 1.897547),
	(0.825688, 1.215596),
	(1.504414, 1.647395),
	(1.037102, 1.080923),
	(1.362835, 1.507135),
	(1.657716, 1.672037),
	(2.333044, 2.356441),
	(1.515152, 1.515152),
	(2.114967, 1.576542),
	(1.782247, 1.782247),
	(1.319261, 1.319261),
	(2.327554, 2.336977),
	(1.888574, 2.977877),
	(2.023875, 2.221519),
	(2.492651, 3.409759),
	(2.294343, 2.470830),
	(2.592257, 2.705456),
	(1.762918, 1.863602),
	(2.055279, 2.076169),
	(1.451696, 1.622987),
	(1.189882, 1.339508),
	(2.147440, 2.223823),
	(1.845883, 1.881462),
	(1.967478, 2.241101),
	(1.729115, 2.145121),
	(1.612379, 2.344689),
	(1.632377, 1.728967),
	(1.420904, 1.440073),
	(1.578216, 1.950282),
	(1.495215, 1.677489),
	(1.869398, 2.165722),
	(1.772814, 1.818885),
	(1.454784, 1.643512),
	(1.788087, 1.808967),
	(2.412670, 2.429586),
	(1.472370, 1.480135),
	(1.466783, 1.488676),
	(2.495288, 2.212590),
	(1.894320, 1.917962),
	(1.689890, 1.713996),
	(1.626184, 1.821737),
	(1.783634, 1.864078),
	(2.031056, 2.105590),
	(1.663276, 1.845992),
	(1.762005, 1.791954),
	(2.076180, 2.164018),
	(2.183990, 2.209045),
	(1.851003, 1.887657),
	(2.065121, 2.363503),
	(2.375992, 2.415675),
	(1.808556, 1.932172),
	(2.382475, 2.665723),
	(1.611868, 1.676057),
	(2.135383, 2.188768),
	(1.633177, 1.536838),
	(1.887631, 1.957185),
	(1.629223, 1.660664),
	(1.117300, 1.376147),
	(1.710821, 1.712625),
	(1.226593, 1.238858),
	(1.700334, 1.759919),
	(2.072264, 2.732655),
	(1.839259, 1.858000),
	(1.789455, 1.852295),
	(1.378043, 1.351542),
	(1.756292, 1.946406),
	(2.057873, 2.196001),
	(2.836015, 2.862425),
	(1.572823, 1.606158),
	(1.489970, 1.491578),
	(2.032367, 2.131835),
	(1.847635, 1.905561),
	(2.011952, 2.046101),
	(1.591822, 1.622336),
	(1.531188, 1.670875),
	(1.384083, 1.586752),
	(1.765717, 1.833707),
	(1.519914, 1.684428),
	(1.718278, 1.872033),
	(1.893842, 1.865691),
	(2.379232, 2.477291),
	(1.808954, 1.874278),
	(1.875119, 1.881448),
	(2.374535, 2.445129),
	(1.893018, 2.079095),
	(1.862976, 2.028380),
	(1.153822, 1.545866),
	(1.203071, 1.130524),
	(2.146034, 2.212466),
	(1.383163, 1.532533),
	(1.254010, 1.254010),
	(2.077160, 2.289628),
	(2.205987, 2.310611),
	(1.442211, 1.470293),
	(1.397789, 1.451187),
	(1.668802, 1.832159),
	(1.499353, 1.624933),
	(1.626941, 1.717108),
	(2.449194, 2.489140),
	(1.383526, 1.544402),
	(1.391891, 1.535993),
	(1.646168, 1.632923),
	(1.880081, 2.217625),
	(1.771255, 1.818248),
	(1.052265, 1.379791),
	(1.341132, 1.559731),
	(1.654280, 1.929535),
	(1.611082, 1.625194),
	(1.474886, 1.499697),
	(2.043613, 2.098422),
	(2.159548, 2.179985),
	(1.849571, 1.897464),
	(1.539288, 1.793890),
	(1.238938, 1.396966),
	(1.323676, 1.402764),
	(1.505266, 1.817193),
	(1.795777, 1.866989),
	(1.045770, 1.059639),
	(1.630812, 1.670467),
	(1.537232, 1.550016),
	(1.580370, 1.701670),
	(1.794193, 1.872202),
	(1.705473, 1.523229),
	(1.469150, 1.522930),
	(1.353591, 1.515391),
	(1.579652, 1.661519),
	(1.427123, 1.470588),
	(1.531051, 1.564801),
	(1.729160, 1.803533),
	(2.365037, 2.371403),
	(1.422329, 1.582843),
	(1.893524, 1.897444),
	(1.809618, 1.905234),
	(2.219283, 2.333186),
	(2.200772, 2.209965),
	(2.506510, 2.622768),
	(2.193017, 2.445758),
	(1.802089, 1.887060),
	(2.171492, 2.569201),
	(1.412873, 1.439037),
	(2.511808, 2.527142),
	(1.045296, 1.447333),
	(2.203826, 2.304508),
	(1.647555, 1.941144),
	(1.700159, 1.648141),
	(1.456767, 1.625157),
	(2.092555, 2.363271),
	(1.682253, 1.709126),
	(1.786857, 1.896536),
	(1.785094, 1.947713),
	(1.036933, 1.127820),
	(2.099666, 2.113921),
	(2.311745, 2.351874),
	(1.826880, 1.873678),
	(1.637749, 1.790174),
	(1.400232, 1.493174),
	(2.098353, 2.162884),
	(1.669440, 1.712924),
	(1.828328, 1.816153),
	(1.768833, 2.037065),
	(1.747482, 1.609474),
	(1.560796, 1.665304),
	(1.840812, 1.858253),
	(1.751492, 1.797122),
	(1.807419, 2.178374),
	(1.856951, 1.962605),
	(1.627968, 1.762656),
	(2.299766, 2.309133),
	(1.541943, 1.590930),
	(1.628007, 1.711998),
	(2.560922, 2.885598),
	(2.243590, 2.389674),
	(1.767968, 1.867791),
	(1.645574, 1.696880),
	(1.452330, 1.486818),
	(1.192591, 1.236996),
	(1.266373, 1.270211),
	(1.923408, 1.942054),
	(1.577164, 2.009906),
	(1.302232, 1.527108),
	(2.507015, 2.515148),
	(1.898485, 1.966366),
	(2.132984, 1.800064),
	(2.036762, 2.189574),
	(2.285101, 2.453800),
	(2.113763, 2.668923),
	(1.964602, 2.252845),
	(1.738473, 1.747366),
	(1.897526, 1.933306),
	(1.702393, 1.721575),
	(1.736094, 1.824308),
	(1.364210, 1.504327),
	(1.234538, 1.489207),
	(2.087097, 2.087097),
	(1.698806, 1.744720),
	(1.977848, 2.158680),
	(1.814809, 1.848541),
	(1.899673, 1.948437),
	(1.693841, 1.702289),
	(1.998251, 2.431524),
	(1.943060, 2.012757),
	(1.070006, 1.092772),
	(1.355271, 1.537204),
	(2.621746, 2.669490),
	(1.995565, 2.038328),
	(1.469311, 1.916784),
	(1.627434, 1.656275),
	(1.391290, 1.431813),
	(1.898734, 2.022303),
	(2.286076, 2.304147),
	(1.791097, 1.877227),
	(2.086301, 2.086301),
	(1.625631, 1.705403),
	(1.431413, 1.584144),
	(1.613008, 2.043359),
	(1.836461, 1.920720),
	(1.299369, 2.121568),
	(1.546518, 1.580776),
	(1.259863, 1.296918),
	(1.832942, 1.036455),
	(1.438383, 1.517222),
	(1.677041, 1.839086),
	(1.930023, 2.069906),
	(1.987215, 2.016994),
	(2.062723, 2.193807),
	(1.832516, 1.840426),
	(2.083051, 2.141729),
	(1.572899, 1.692849),
	(2.519119, 2.549709),
	(1.335205, 1.808566),
	(2.249473, 2.265094),
	(1.826061, 1.876868),
	(1.796442, 2.092923),
	(1.719945, 1.830481),
	(1.775939, 1.802692),
	(1.546425, 1.634327),
	(1.414824, 1.646809),
	(1.887039, 1.965484),
	(2.387984, 2.463927),
	(1.482290, 1.505792),
	(1.394069, 1.538605),
	(1.801298, 1.846438),
	(1.984347, 2.120515),
	(2.048752, 2.192524),
	(1.589338, 1.853791),
	(1.610707, 1.727733),
	(1.632170, 1.684760),
	(1.549480, 1.857197),
	(1.957408, 2.050429),
	(1.858006, 1.879586),
	(1.430515, 1.593780),
	(1.340111, 1.350013),
	(1.641545, 1.690364),
	(1.263142, 1.270616),
	(1.612848, 1.647164),
	(1.909477, 1.942480),
	(2.635892, 2.697350),
	(1.602488, 1.620315),
	(1.412663, 1.568565),
	(2.073593, 2.145743),
	(1.517067, 1.580278),
	(1.863580, 1.889535),
	(1.510949, 1.533717),
	(1.973684, 2.030430),
	(2.073234, 2.088706),
	(2.236386, 2.236386),
	(1.619337, 1.643182),
	(1.801837, 1.899551),
	(1.288265, 1.353635),
	(2.105394, 2.860858),
	(1.682300, 1.817758),
	(1.053384, 1.054300),
	(1.510470, 1.851307),
	(1.288154, 1.306003),
	(1.546919, 1.655820),
	(2.843310, 2.948944),
	(1.884841, 2.031376),
	(2.009978, 2.067543),
	(1.478823, 1.574539),
	(2.304933, 2.380525),
	(1.614907, 1.784897),
	(1.464765, 1.541409),
	(1.483227, 1.560458),
	(1.645591, 1.714617),
	(1.566166, 1.602625),
	(1.917649, 2.051117),
	(1.602816, 1.638216),
	(2.142751, 2.174653),
	(1.941406, 1.952237),
	(1.871650, 1.880735),
	(1.917606, 1.793473),
	(1.750732, 1.856492),
	(2.246004, 2.355136),
	(2.322498, 2.398061),
	(1.775784, 1.915977),
	(1.504827, 1.636651),
	(1.594460, 1.716863),
	(1.555071, 1.648855),
	(1.481285, 1.847009),
	(3.137066, 3.156938),
	(1.358589, 1.503245),
	(2.208992, 2.312902),
	(1.868323, 1.869565),
	(1.514856, 1.559755),
	(1.609885, 1.684088),
	(1.913709, 2.044604),
	(2.226589, 2.435065),
	(1.711722, 1.755080),
	(1.835585, 2.051832),
	(1.326324, 1.367319),
	(1.484980, 1.571632),
	(2.045954, 2.047952),
	(1.886705, 1.993499),
	(1.417445, 1.494848),
	(2.052899, 2.095187),
	(1.700106, 1.719716),
	(2.000471, 2.009885),
	(1.931649, 1.972334),
	(2.221111, 2.358564),
	(1.272718, 1.341118),
	(2.388601, 2.415721),
	(1.442239, 1.460222),
	(1.817961, 1.878661),
	(1.421137, 1.478132),
	(1.225555, 1.281505),
	(1.902421, 1.898290),
	(2.156244, 2.226366),
	(1.588889, 1.787500),
	(1.743085, 1.764136),
	(1.563544, 1.618669),
	(1.914553, 2.132176),
	(2.378319, 2.413085),
	(1.389245, 1.427717),
	(0.946980, 1.153696),
	(2.223814, 2.352079),
	(2.126636, 2.135549),
	(1.511381, 1.767048),
	(1.310380, 1.376756),
	(2.211860, 2.251245),
	(2.122959, 2.447279),
	(1.670795, 1.703826),
	(2.443057, 2.988029),
	(1.476190, 1.680556),
	(2.012490, 2.053514),
	(1.937736, 1.944159),
	(1.624234, 1.685526),
	(1.902349, 1.918436),
	(1.441603, 1.504724),
	(1.651948, 1.688312),
	(1.448953, 1.532333),
	(1.849003, 1.887458),
	(1.614231, 1.673678),
	(3.158141, 3.155542),
	(1.985400, 2.013225),
	(2.536341, 2.537594),
	(2.523984, 3.314720),
	(1.487741, 1.555688),
	(1.638916, 1.766122),
	(1.791375, 1.894530),
	(1.533923, 1.620312),
	(2.242884, 2.276268),
	(1.532980, 1.740139),
	(1.870102, 2.095135),
	(1.328904, 1.819516),
	(1.418847, 1.617751),
	(1.912342, 1.933016),
	(1.511370, 1.543063),
	(2.566674, 2.616819),
	(1.848164, 1.893735),
	(1.448312, 1.651761),
	(1.754951, 1.772530),
	(1.785714, 1.785714),
	(2.211038, 2.292769),
	(1.917073, 1.950051),
	(1.912613, 1.919381),
	(2.218768, 2.271259),
	(1.783962, 1.845297),
	(1.291069, 1.334180),
	(1.573098, 1.643535),
	(1.748905, 1.791999),
	(1.519003, 1.581514),
	(1.414709, 1.556890),
	(1.769512, 1.967870),
	(1.730722, 1.771966),
	(2.222737, 2.242031),
	(2.094718, 2.130858),
	(1.748478, 1.778082),
	(2.393368, 2.400053),
	(1.845539, 1.985402),
	(1.897679, 1.925922),
	(2.050594, 2.131358),
	(1.601597, 1.739130),
	(2.371576, 2.427833),
	(1.408304, 1.943564),
	(1.945764, 2.016897),
	(1.958933, 1.957603),
	(1.484368, 1.580203),
	(1.694675, 2.309487),
	(2.108148, 2.183714),
	(2.746722, 2.812284),
	(1.829441, 2.001198),
	(2.730487, 2.843097),
	(1.990447, 2.096017),
	(1.462054, 1.495536),
	(1.276596, 1.297558),
	(2.287893, 2.291298),
	(1.789421, 1.799798),
	(1.342433, 1.429266),
	(1.513538, 1.556139),
	(2.045410, 2.098538),
	(1.532143, 1.673214),
	(1.566048, 1.596900),
	(1.700316, 1.707956),
	(1.681095, 1.727113),
	(1.554635, 1.623680),
	(1.913406, 2.024726),
	(1.698084, 1.701979),
	(1.456541, 1.540978),
	(1.942895, 2.097318),
	(2.001840, 2.069246),
	(2.180196, 2.197535),
	(2.169258, 2.242256),
	(1.596237, 1.685188),
	(2.093787, 1.992326),
	(2.093833, 2.290846),
	(1.903236, 2.109229),
	(1.762077, 1.792775),
	(2.535368, 2.640777),
	(2.037206, 2.113032),
	(1.261012, 1.900155),
	(2.041033, 2.063796),
	(1.722429, 1.806450),
	(1.845939, 1.888702),
	(2.196621, 2.230756),
	(1.593880, 1.783102),
	(2.156024, 2.315120),
	(1.587302, 1.701528),
	(1.970894, 2.018823),
	(1.978670, 2.054098),
	(1.624225, 1.680233),
	(1.896240, 1.923212),
	(2.007136, 2.086148),
	(1.644238, 1.667869),
	(1.568570, 1.625376),
	(1.674945, 1.645312),
	(2.064494, 2.154679),
	(1.541637, 1.679828),
	(2.516153, 2.543193),
	(2.060639, 2.099988),
	(1.181365, 1.308916),
	(1.852363, 1.906371),
	(1.724199, 1.789229),
	(1.776190, 2.007410),
	(1.567874, 1.749949),
	(1.597106, 2.170652),
	(2.144300, 2.630592),
	(1.727490, 1.810523),
	(1.717991, 1.822853),
	(1.779086, 2.248383),
	(1.963587, 2.021065),
	(1.719331, 1.776863),
	(1.539936, 1.549656),
	(1.429029, 1.445034),
	(1.933861, 1.976297),
	(1.791537, 2.067158),
	(2.200963, 2.391713),
	(1.720409, 1.745688),
	(1.302496, 1.352592),
	(2.149709, 2.225363),
	(2.403138, 2.449870),
	(1.561605, 1.801683),
	(1.575414, 1.578561),
	(1.843799, 1.848769),
	(2.343906, 2.400100),
	(1.943103, 1.974778),
	(1.410962, 1.522413),
	(2.011352, 2.074933),
	(2.017722, 2.317088),
	(1.349903, 1.697507),
	(2.015327, 2.072365),
	(2.314453, 2.371067),
	(1.594474, 1.726809),
	(1.781893, 1.923833),
	(1.854314, 2.267327),
	(1.791626, 1.797190),
	(1.616367, 1.622793),
	(2.227188, 2.276200),
	(1.902490, 2.161100),
	(1.457396, 1.538281),
	(1.940402, 1.975052),
	(1.519791, 1.536081),
	(1.587667, 1.636425),
	(1.207614, 1.208387),
	(3.201123, 3.639874),
	(1.543350, 1.550483),
	(0.473271, 0.474155),
	(1.760097, 1.873022),
	(1.647522, 1.666088),
	(1.530171, 1.549277),
	(1.416229, 1.416229),
	(0.754647, 0.755194),
	(1.307947, 1.308831),
	(2.563598, 2.689109),
	(1.982926, 2.284598),
	(1.649334, 1.687628),
	(1.493414, 1.579833),
	(1.913486, 2.152980),
	(1.507043, 1.510873),
	(1.910425, 1.932046),
	(1.771486, 1.808629),
	(1.232687, 1.255111),
	(1.704853, 1.712349),
	(1.786143, 1.861663),
	(1.656956, 1.707801),
	(1.356705, 1.364668),
	(1.339286, 1.365248),
	(1.990393, 2.027209),
	(1.408920, 1.413438),
	(1.339947, 1.372354),
	(1.060665, 1.062902),
	(2.131391, 2.156643),
	(1.659589, 1.665612),
	(2.032351, 2.077194),
	(2.794407, 2.825160),
	(1.033744, 1.307738),
	(1.235944, 1.287174),
	(2.635914, 2.727112),
	(2.220332, 2.262869),
	(1.921268, 1.974911),
	(2.452853, 2.133613),
	(1.156451, 1.160151),
	(2.149875, 2.149875),
	(2.811018, 2.811018),
	(1.956766, 2.128644),
	(1.871162, 1.798749),
	(1.849905, 1.914527),
	(1.897690, 2.613744),
	(1.442733, 1.703841),
	(1.840725, 1.861825),
	(1.956908, 2.857525),
	(1.515348, 1.527374),
	(2.071268, 2.187902),
	(1.541468, 1.587991),
	(2.370698, 2.820611),
	(1.933800, 1.945374),
	(1.896654, 1.963491),
	(1.513891, 1.524586),
	(1.783617, 1.854338),
	(1.849737, 1.927602),
	(1.881852, 1.970945),
	(1.261186, 1.544026),
	(1.693828, 1.890894),
	(1.854093, 2.002953),
	(1.795388, 1.797967),
	(1.831472, 1.880663),
	(1.479904, 1.771516),
	(1.829553, 1.845141),
	(2.306199, 2.335310),
	(1.570205, 1.572059),
	(2.017395, 2.016006),
	(2.114387, 2.132102),
	(1.548001, 1.608863),
	(2.145476, 2.163423),
	(1.777246, 1.791297),
	(1.980084, 1.985829),
	(1.641643, 1.788124),
	(1.153357, 1.225095),
	(1.940642, 1.940907),
	(2.547906, 2.661356),
	(2.247022, 2.258527),
	(1.788491, 1.905669),
	(1.652420, 1.658077),
	(1.772939, 2.161942),
	(1.060540, 1.077195),
	(1.272011, 1.325243),
	(1.817597, 1.895109),
	(1.162976, 1.182797),
	(1.303850, 1.308206),
	(1.402132, 1.462897),
	(1.772444, 1.777269),
	(1.299856, 1.316872),
	(1.154559, 1.176763),
	(1.515685, 1.633466),
	(2.176489, 1.524880),
	(2.346517, 2.376033),
	(1.616848, 1.791758),
	(2.355282, 2.679374),
	(1.228278, 1.238410),
	(1.331963, 1.359541),
	(2.707968, 2.755216),
	(1.504362, 1.524401),
	(2.572688, 2.839572),
	(2.304355, 2.336539),
	(1.253043, 1.301596),
	(1.051973, 1.058548),
	(1.574651, 1.581604),
	(1.377014, 1.391316),
	(3.134557, 3.232853),
	(1.171827, 1.193389),
	(1.429962, 1.446871),
	(1.036725, 1.051891),
	(2.090560, 2.210339),
	(1.575183, 1.626349),
	(0.280112, 0.280112),
	(1.340367, 1.369925),
	(1.562908, 1.575139),
	(2.092219, 2.078069),
	(1.648816, 2.107155),
	(1.772586, 1.894260),
	(2.182940, 2.238143),
	(2.058135, 2.393958),
	(1.890587, 1.867601),
	(1.989763, 1.992069),
	(1.798711, 1.802870),
	(2.101905, 2.106430),
	(1.696480, 1.812243),
	(2.283578, 2.371529),
	(2.134378, 2.179697),
	(1.879130, 1.967142),
	(1.832168, 1.876348),
	(1.623926, 1.711939),
	(1.901861, 2.263083),
	(2.031832, 2.102724),
	(1.849870, 2.071222),
	(1.392554, 1.547514),
	(1.998750, 2.044422),
	(1.893699, 1.969349),
	(2.050183, 2.252188),
	(2.206681, 2.284764),
	(2.393489, 2.489517),
	(1.480314, 1.503657),
	(1.717090, 1.730161),
	(2.346541, 2.635364),
	(1.983615, 2.124936),
	(1.849063, 1.969784),
	(1.909869, 1.924919),
	(1.213367, 1.229835),
	(2.505967, 2.561979),
	(2.118378, 2.122305),
	(2.313152, 2.352855),
	(2.763797, 2.793730),
	(1.965494, 1.965967),
	(1.878462, 2.011704),
	(1.227904, 1.315237),
	(2.201716, 2.527642),
	(1.466313, 1.495667),
	(1.630454, 1.674587),
	(1.325795, 1.342955),
	(1.372929, 1.439132),
	(1.521653, 1.523548),
	(1.567786, 1.654166),
	(0.911701, 0.916913),
	(1.545672, 1.668971),
	(1.178899, 1.209515),
	(1.384740, 1.390191),
	(2.085506, 2.193040),
	(1.514841, 1.536779),
	(1.409673, 1.424210),
	(1.991162, 2.009253),
	(2.368085, 2.385389),
	(1.039881, 1.041705),
	(1.606614, 1.494771),
	(2.290779, 2.300907),
	(2.835160, 2.980770),
	(0.926814, 1.214446),
	(2.062145, 2.157710),
	(1.673108, 1.682132),
	(1.736780, 1.787293),
	(2.797087, 2.799451),
	(0.906042, 0.937365),
	(2.043012, 2.093888),
	(2.174230, 2.183958),
	(1.329249, 1.367908),
	(1.560969, 1.596560),
	(1.937046, 1.957596),
	(1.853702, 1.862027),
	(2.772275, 2.789159),
	(2.203896, 2.313731),
	(0.902256, 0.902256),
	(1.724138, 1.736453),
	(1.265408, 1.364948),
	(1.450768, 1.556410),
	(1.547633, 1.562474),
	(1.501233, 1.510074),
	(1.962608, 1.979731),
	(2.138215, 2.183313),
	(2.456187, 2.615507),
	(1.372267, 1.421625),
	(2.397984, 2.444485),
	(1.687873, 1.717086),
	(1.774753, 1.785043),
	(1.343794, 1.342380),
	(1.213325, 1.226423),
	(1.493343, 1.505712),
	(3.932592, 3.935914),
	(1.878264, 1.886057),
	(2.006544, 2.020036),
	(2.379380, 2.605763),
	(2.490602, 2.514098),
	(1.676300, 1.993174),
	(1.758741, 1.783338),
	(1.147471, 1.241576),
	(1.436561, 1.455996),
	(2.159021, 2.266780),
	(0.867188, 0.873373),
	(1.213495, 1.235051),
	(2.406319, 2.465414),
	(2.110107, 2.128992),
	(2.555130, 2.637773),
	(2.310924, 2.335363),
	(1.183844, 1.186956),
	(1.380617, 1.442494),
	(1.467927, 1.510831),
	(1.304580, 1.312175),
	(1.697260, 1.709505),
	(1.291531, 1.299640),
	(0.976240, 0.992400),
	(3.159460, 3.202825),
	(1.902555, 1.950448),
	(2.387880, 2.390703),
	(2.247805, 2.255786),
	(1.116554, 1.119892),
	(2.572406, 2.691556),
	(1.610759, 1.614669),
	(1.362183, 1.365348),
	(1.597926, 1.610770),
	(1.575701, 1.674147),
	(1.478795, 1.702009),
	(1.460925, 1.476341),
	(2.391881, 2.564403),
	(1.361958, 1.378588),
	(1.120691, 1.121476),
	(1.467319, 1.489097),
	(4.672151, 4.683293),
	(1.528951, 1.566165),
	(1.554970, 1.555348),
	(1.894891, 1.908819),
	(2.311082, 2.378093),
	(1.157387, 1.192900),
	(1.479758, 1.494070),
	(1.175961, 1.188644),
	(1.339698, 1.371453),
	(2.457955, 2.466654),
	(1.236044, 1.244483),
	(2.000414, 2.004557),
	(1.373242, 1.376854),
	(0.760425, 0.769216),
	(1.036132, 1.061715),
	(1.341753, 1.348672),
	(1.390107, 1.428627),
	(0.769962, 0.785401),
	(1.208047, 1.212964),
	(1.257759, 1.273444),
	(0.830060, 0.832470),
	(1.299411, 1.302211),
	(1.110715, 1.134739),
	(1.780798, 1.786417),
	(3.143389, 3.161387),
	(1.089676, 1.099353),
	(1.451773, 1.470415),
	(1.375346, 1.377475),
	(1.254443, 1.277873),
	(1.633536, 1.650159),
	(0.977969, 0.979379),
	(1.153417, 1.159999),
	(0.971210, 0.982292),
	(1.213915, 1.243570),
	(2.612147, 2.648283),
	(2.784731, 2.789201),
	(0.919622, 0.937811),
	(1.261426, 1.263335),
	(1.224602, 1.226493),
	(1.034082, 1.110657),
	(0.826802, 0.839413),
	(2.589468, 3.042867),
	(1.793877, 1.811448),
	(1.374330, 1.389331),
	(2.666914, 2.759495),
	(1.318575, 1.335390),
	(1.103767, 1.113074),
	(0.976598, 0.982039),
	(0.786973, 0.790781),
	(1.042000, 1.203548),
	(1.263201, 1.265788),
	(1.702189, 1.795092),
	(0.857639, 0.895446),
	(1.024628, 1.053475),
	(2.147729, 2.278884),
	(2.281822, 2.287931),
	(1.513813, 1.589207),
	(2.186309, 2.201968),
	(2.230842, 2.267998),
	(1.651383, 1.661558),
	(4.372778, 4.431567),
	(1.423293, 1.431956),
	(1.306790, 1.308565),
	(0.870040, 0.878465),
	(1.314863, 1.355689),
	(1.852380, 1.854164),
	(1.225205, 1.262126),
	(1.118317, 1.131213),
	(1.708601, 1.721521),
	(1.589145, 1.765181),
	(2.126998, 2.380434),
	(2.914077, 3.078018),
	(1.541670, 1.548534),
	(1.110520, 1.117626),
	(2.771855, 2.523099),
} \fill \pos circle(0.03);

\draw (0,0) -- (5, 5);

\end{tikzpicture}


  \column{0.55\textwidth}
  \centering
  \only<+>{\begin{tikzpicture}

\draw (0,0) -- (5,0);
\node at (2.5,-0.6) {LUnivRPI};
\node [anchor=north] at (0.714285714285714,0) {\small 0.1};
\draw (0.714285714285714,0) -- (0.714285714285714,0.1);
\draw [style=help lines] (0.714285714285714,0) -- (0.714285714285714,5);
\node [anchor=north] at (1.42857142857143,0) {\small 0.2};
\draw (1.42857142857143,0) -- (1.42857142857143,0.1);
\draw [style=help lines] (1.42857142857143,0) -- (1.42857142857143,5);
\node [anchor=north] at (2.14285714285714,0) {\small 0.3};
\draw (2.14285714285714,0) -- (2.14285714285714,0.1);
\draw [style=help lines] (2.14285714285714,0) -- (2.14285714285714,5);
\node [anchor=north] at (2.85714285714286,0) {\small 0.4};
\draw (2.85714285714286,0) -- (2.85714285714286,0.1);
\draw [style=help lines] (2.85714285714286,0) -- (2.85714285714286,5);
\node [anchor=north] at (3.57142857142857,0) {\small 0.5};
\draw (3.57142857142857,0) -- (3.57142857142857,0.1);
\draw [style=help lines] (3.57142857142857,0) -- (3.57142857142857,5);
\node [anchor=north] at (4.28571428571429,0) {\small 0.6};
\draw (4.28571428571429,0) -- (4.28571428571429,0.1);
\draw [style=help lines] (4.28571428571429,0) -- (4.28571428571429,5);
\node [anchor=north] at (5,0) {\small 0.7};
\draw (5,0) -- (5,0.1);
\draw [style=help lines] (5,0) -- (5,5);
\draw (0,0) -- (0,5);
\node [rotate=90] at (-2.5em,2.5) {RPI[3]LU};
\node [anchor=east] at (0,0.713285714285714) {\small 0.1};
\draw (0,0.713285714285714) -- (0.1,0.713285714285714);
\draw [style=help lines] (0,0.713285714285714) -- (5,0.713285714285714);
\node [anchor=east] at (0,1.42657142857143) {\small 0.2};
\draw (0,1.42657142857143) -- (0.1,1.42657142857143);
\draw [style=help lines] (0,1.42657142857143) -- (5,1.42657142857143);
\node [anchor=east] at (0,2.13985714285714) {\small 0.3};
\draw (0,2.13985714285714) -- (0.1,2.13985714285714);
\draw [style=help lines] (0,2.13985714285714) -- (5,2.13985714285714);
\node [anchor=east] at (0,2.85314285714286) {\small 0.4};
\draw (0,2.85314285714286) -- (0.1,2.85314285714286);
\draw [style=help lines] (0,2.85314285714286) -- (5,2.85314285714286);
\node [anchor=east] at (0,3.56642857142857) {\small 0.5};
\draw (0,3.56642857142857) -- (0.1,3.56642857142857);
\draw [style=help lines] (0,3.56642857142857) -- (5,3.56642857142857);
\node [anchor=east] at (0,4.27971428571429) {\small 0.6};
\draw (0,4.27971428571429) -- (0.1,4.27971428571429);
\draw [style=help lines] (0,4.27971428571429) -- (5,4.27971428571429);
\node [anchor=east] at (0,4.993) {\small 0.7};
\draw (0,4.993) -- (0.1,4.993);
\draw [style=help lines] (0,4.993) -- (5,4.993);

\foreach \pos in {
	(1.559731, 1.556186),
	(2.360406, 2.342277),
	(2.046691, 2.062508),
	(1.167770, 1.158038),
	(1.281639, 1.201201),
	(0.484526, 0.484526),
	(1.568389, 1.501520),
	(0.250054, 0.250054),
	(2.312593, 2.046426),
	(1.687598, 1.732451),
	(1.427291, 1.420891),
	(1.802097, 1.802097),
	(1.080477, 1.165638),
	(2.232143, 2.242647),
	(1.598280, 1.604737),
	(0.898964, 0.908736),
	(1.950998, 1.756547),
	(2.323961, 2.026342),
	(1.798202, 1.684679),
	(3.070307, 3.246073),
	(2.120202, 2.115927),
	(1.767990, 1.878766),
	(2.047748, 1.926436),
	(2.199977, 2.245221),
	(1.964637, 2.002058),
	(1.861827, 1.861827),
	(1.454545, 1.474026),
	(2.375626, 2.173218),
	(1.565217, 1.310559),
	(1.882204, 1.903352),
	(1.926469, 1.729752),
	(2.167488, 2.151067),
	(1.821531, 1.794242),
	(1.428571, 1.410488),
	(1.779762, 1.755952),
	(2.059437, 2.075078),
	(1.950138, 1.827776),
	(2.146109, 1.560806),
	(1.321892, 1.321892),
	(2.117621, 1.634847),
	(1.104901, 1.104901),
	(2.193929, 2.208315),
	(1.996086, 2.100457),
	(1.557465, 1.450054),
	(1.389150, 1.210813),
	(2.868217, 1.860465),
	(2.181025, 2.167394),
	(1.407490, 1.444692),
	(2.072704, 2.032844),
	(1.464543, 1.464543),
	(1.627940, 1.627940),
	(1.657887, 1.634645),
	(2.711896, 2.307409),
	(1.404151, 1.613466),
	(1.392857, 1.585714),
	(2.213658, 1.971680),
	(1.611722, 1.636142),
	(0.134983, 0.134983),
	(0.474354, 0.474354),
	(2.271517, 2.236025),
	(2.153625, 1.920316),
	(1.783343, 1.811800),
	(1.569990, 1.605944),
	(1.727484, 1.727484),
	(1.270053, 1.250955),
	(1.872120, 1.800115),
	(1.627950, 1.463325),
	(0.963880, 0.963880),
	(1.828231, 1.530612),
	(1.602632, 1.560178),
	(2.857143, 1.867044),
	(1.997579, 1.997579),
	(1.566620, 1.713975),
	(2.157327, 1.710076),
	(1.680000, 1.577143),
	(1.535714, 1.464286),
	(1.700680, 1.575964),
	(2.122088, 1.887304),
	(1.845523, 1.822739),
	(1.251476, 1.227863),
	(1.503759, 1.194162),
	(1.755562, 1.755562),
	(1.813616, 1.813616),
	(1.925841, 1.925841),
	(2.537594, 2.537594),
	(1.628264, 1.628264),
	(1.720210, 1.742994),
	(1.829268, 1.916376),
	(2.253401, 2.097506),
	(1.596841, 1.596841),
	(0.267681, 0.267681),
	(2.186998, 1.926164),
	(0.287698, 0.287698),
	(0.287698, 0.287698),
	(1.294778, 1.273198),
	(1.071429, 1.071429),
	(1.916980, 1.916980),
	(0.850181, 0.853528),
	(0.789474, 0.789474),
	(0.921659, 0.875576),
	(0.115830, 0.115830),
	(1.892857, 1.392857),
	(1.728111, 1.344086),
	(1.728111, 1.344086),
	(0.418443, 0.418443),
	(1.750237, 1.655629),
	(0.315770, 0.324863),
	(0.127767, 0.127767),
	(0.127767, 0.129112),
	(0.545889, 0.548326),
	(0.369922, 0.410277),
	(1.051051, 1.099314),
	(0.294922, 0.294922),
	(1.059771, 1.066836),
	(1.378245, 1.355370),
	(2.792823, 3.368314),
	(0.527024, 0.527024),
	(0.708075, 0.708075),
	(1.433271, 1.409774),
	(1.235231, 1.235231),
	(0.253593, 0.253593),
	(1.751152, 1.751152),
	(1.927438, 1.927438),
	(0.992063, 0.396825),
	(1.127820, 0.751880),
	(0.480769, 0.480769),
	(0.836551, 0.772201),
	(1.749271, 1.749271),
	(1.512605, 1.512605),
	(1.632653, 1.632653),
	(0.000000, 0.000000),
	(0.347222, 0.148810),
	(1.020408, 1.020408),
	(1.219512, 1.219512),
	(0.744048, 0.744048),
	(0.533049, 0.533049),
	(1.166181, 1.166181),
	(1.405152, 1.522248),
	(0.840336, 0.840336),
	(0.000000, 0.000000),
	(2.142857, 2.142857),
	(3.125000, 2.678571),
	(4.395604, 4.395604),
	(1.836897, 1.899454),
	(2.261947, 1.864130),
	(1.851852, 1.949833),
	(1.242989, 1.371096),
	(1.535905, 1.728143),
	(1.644669, 1.692539),
	(2.140851, 2.120792),
	(1.380820, 1.388778),
	(1.578703, 1.547748),
	(1.924430, 1.940670),
	(1.787746, 1.836503),
	(1.765904, 1.792318),
	(1.546519, 1.542395),
	(1.573535, 1.455830),
	(1.236199, 1.283614),
	(2.747253, 2.816901),
	(1.447544, 1.563488),
	(1.969445, 2.038230),
	(1.884921, 1.884921),
	(1.733136, 1.701979),
	(1.542675, 1.538110),
	(1.636106, 1.709805),
	(2.047464, 2.051342),
	(1.144031, 1.078029),
	(1.837379, 1.837379),
	(2.189246, 2.205523),
	(2.601132, 2.646262),
	(1.667896, 1.651504),
	(2.142254, 2.134208),
	(1.547956, 1.537839),
	(2.206890, 2.164359),
	(2.109326, 2.105427),
	(1.547762, 1.607786),
	(2.425505, 2.386194),
	(1.620961, 1.597045),
	(2.446886, 1.853480),
	(1.591711, 1.591711),
	(2.261660, 2.340862),
	(2.047508, 2.134493),
	(2.795241, 2.809749),
	(1.733012, 1.713848),
	(2.090695, 2.082282),
	(1.784829, 1.791912),
	(1.753752, 1.806559),
	(2.185472, 2.099460),
	(1.881302, 2.108006),
	(2.821204, 2.416891),
	(1.868721, 1.890005),
	(1.875477, 1.791444),
	(1.779997, 1.802866),
	(1.990434, 1.985603),
	(1.472413, 1.480685),
	(1.980792, 1.984127),
	(1.601960, 1.470034),
	(2.829918, 2.335578),
	(1.254181, 1.230291),
	(2.532891, 1.419248),
	(1.951425, 1.936585),
	(2.521008, 2.496025),
	(1.567176, 1.539613),
	(1.831388, 1.813564),
	(1.981474, 2.029221),
	(1.936119, 1.936119),
	(1.402088, 1.417666),
	(2.120105, 2.261445),
	(2.128773, 2.100604),
	(2.289738, 2.305835),
	(1.597386, 1.443093),
	(1.455745, 1.378106),
	(1.646825, 1.661706),
	(2.464418, 2.477596),
	(2.025629, 1.988277),
	(1.923284, 1.901705),
	(0.113773, 0.114527),
	(1.627424, 1.597744),
	(1.095530, 1.139351),
	(1.455203, 1.464714),
	(1.915403, 1.900439),
	(1.408738, 1.493321),
	(1.814901, 1.793674),
	(1.520147, 1.575092),
	(2.225511, 2.292528),
	(1.729043, 1.675705),
	(1.879391, 1.812061),
	(1.141527, 1.163693),
	(2.572608, 1.842842),
	(1.519991, 1.508977),
	(1.280512, 1.293851),
	(2.194211, 2.131964),
	(1.702154, 1.850706),
	(2.214369, 2.275112),
	(1.830956, 1.873536),
	(1.624462, 1.391543),
	(2.210814, 2.175203),
	(1.826299, 1.808905),
	(1.651376, 1.627785),
	(1.804021, 1.810678),
	(2.228958, 2.228958),
	(2.006173, 2.037667),
	(1.595509, 1.440390),
	(2.411874, 2.481447),
	(1.923310, 1.814443),
	(1.515322, 1.464624),
	(1.666890, 1.566475),
	(1.498501, 1.467283),
	(1.588160, 1.575290),
	(1.548225, 1.564395),
	(1.648352, 1.684185),
	(1.955850, 1.772847),
	(2.092389, 2.188168),
	(1.593683, 1.514716),
	(1.594286, 1.634286),
	(1.604295, 1.539212),
	(1.723366, 1.710577),
	(2.523762, 2.599366),
	(1.414901, 1.196172),
	(1.774644, 1.784863),
	(1.256026, 1.243339),
	(2.135427, 2.161272),
	(2.002165, 1.897547),
	(1.182831, 1.215596),
	(1.684695, 1.647395),
	(1.088227, 1.080923),
	(1.515152, 1.507135),
	(1.675618, 1.672037),
	(2.366468, 2.356441),
	(1.599081, 1.515152),
	(2.192439, 1.576542),
	(1.934813, 1.782247),
	(1.351570, 1.319261),
	(2.369959, 2.336977),
	(3.129637, 2.977877),
	(2.051546, 2.221519),
	(3.339212, 3.409759),
	(2.622806, 2.470830),
	(2.716776, 2.705456),
	(1.867401, 1.863602),
	(2.310783, 2.076169),
	(1.597294, 1.622987),
	(1.209476, 1.339508),
	(2.208546, 2.223823),
	(1.870997, 1.881462),
	(2.168135, 2.241101),
	(2.048902, 2.145121),
	(2.182314, 2.344689),
	(1.721723, 1.728967),
	(1.442469, 1.440073),
	(1.819446, 1.950282),
	(1.526544, 1.677489),
	(2.119993, 2.165722),
	(1.796771, 1.818885),
	(1.515072, 1.643512),
	(1.824152, 1.808967),
	(2.457075, 2.429586),
	(1.473923, 1.480135),
	(1.490666, 1.488676),
	(2.516020, 2.212590),
	(1.935694, 1.917962),
	(1.711586, 1.713996),
	(1.711953, 1.821737),
	(1.844660, 1.864078),
	(2.093168, 2.105590),
	(1.719786, 1.845992),
	(1.839373, 1.791954),
	(2.267827, 2.164018),
	(2.198605, 2.209045),
	(1.869330, 1.887657),
	(2.373973, 2.363503),
	(2.976190, 2.415675),
	(1.934860, 1.932172),
	(2.606332, 2.665723),
	(1.679623, 1.676057),
	(2.174532, 2.188768),
	(2.192862, 1.536838),
	(1.993895, 1.957185),
	(1.666381, 1.660664),
	(1.202490, 1.376147),
	(1.712625, 1.710821),
	(1.247036, 1.238858),
	(1.746839, 1.759919),
	(3.119781, 2.732655),
	(1.874063, 1.858000),
	(1.822371, 1.849303),
	(1.423978, 1.351542),
	(2.109361, 1.946406),
	(2.121084, 2.196001),
	(2.866488, 2.862425),
	(1.691012, 1.606158),
	(1.496400, 1.491578),
	(2.099575, 2.102264),
	(1.857622, 1.905561),
	(2.071713, 2.046101),
	(1.645222, 1.622336),
	(1.630581, 1.670875),
	(1.710331, 1.586752),
	(1.827708, 1.833707),
	(1.689339, 1.684428),
	(1.855848, 1.872033),
	(1.952705, 1.865691),
	(2.534063, 2.477291),
	(1.854178, 1.874278),
	(1.892525, 1.881448),
	(2.384161, 2.445129),
	(1.974892, 2.079095),
	(2.581179, 2.028380),
	(1.564990, 1.545866),
	(1.248413, 1.130524),
	(2.224019, 2.212466),
	(1.523571, 1.532533),
	(1.254010, 1.254010),
	(2.270059, 2.289628),
	(2.328344, 2.310611),
	(1.470293, 1.470293),
	(1.498304, 1.451187),
	(1.796925, 1.832159),
	(1.583073, 1.624933),
	(1.634781, 1.717108),
	(2.486643, 2.489140),
	(1.485414, 1.544402),
	(1.532718, 1.535993),
	(1.710501, 1.632923),
	(2.210366, 2.217625),
	(1.800174, 1.818248),
	(1.358885, 1.379791),
	(1.536098, 1.559731),
	(1.946050, 1.929535),
	(1.644010, 1.625194),
	(1.557589, 1.499697),
	(2.161062, 2.098422),
	(2.292390, 2.179985),
	(1.881500, 1.897464),
	(1.868635, 1.793890),
	(1.412769, 1.396966),
	(1.361139, 1.402764),
	(1.797161, 1.817193),
	(1.848412, 1.866989),
	(1.059639, 1.059639),
	(1.675424, 1.670467),
	(1.559604, 1.550016),
	(1.698205, 1.701670),
	(1.831502, 1.872202),
	(1.904037, 1.523229),
	(1.520485, 1.522930),
	(1.625888, 1.515391),
	(1.670822, 1.661519),
	(1.506810, 1.470588),
	(1.601620, 1.564801),
	(1.825225, 1.803533),
	(2.374586, 2.371403),
	(1.565008, 1.582843),
	(1.924886, 1.897444),
	(1.859197, 1.905234),
	(2.358907, 2.333186),
	(2.285347, 2.209965),
	(3.022693, 2.622768),
	(2.721829, 2.445758),
	(1.823332, 1.887060),
	(2.664652, 2.569201),
	(1.530612, 1.439037),
	(2.680488, 2.527142),
	(1.259716, 1.447333),
	(2.796733, 2.304508),
	(1.806438, 1.941144),
	(1.930132, 1.648141),
	(1.582080, 1.625157),
	(2.231571, 2.361441),
	(1.700168, 1.709126),
	(1.869116, 1.896536),
	(1.818611, 1.947713),
	(1.111295, 1.127820),
	(2.190894, 2.113921),
	(2.330937, 2.351874),
	(1.851146, 1.873678),
	(1.780444, 1.790174),
	(1.482509, 1.493174),
	(2.142857, 2.162884),
	(1.737503, 1.712924),
	(1.891234, 1.816153),
	(1.954532, 2.037065),
	(1.846326, 1.609474),
	(1.674391, 1.665304),
	(1.866180, 1.858253),
	(1.821692, 1.797122),
	(2.111287, 2.178374),
	(1.953000, 1.962605),
	(1.728984, 1.762656),
	(2.320062, 2.309133),
	(1.574601, 1.590930),
	(1.724508, 1.711998),
	(3.210273, 2.885598),
	(2.359149, 2.389674),
	(1.805679, 1.867791),
	(1.696880, 1.696880),
	(1.481070, 1.486818),
	(1.254440, 1.236996),
	(1.281723, 1.270211),
	(1.942054, 1.942054),
	(1.853493, 2.009906),
	(1.465778, 1.527108),
	(2.594445, 2.515148),
	(1.898485, 1.966366),
	(2.194976, 1.800064),
	(2.126266, 2.189574),
	(2.564987, 2.453800),
	(2.256264, 2.668923),
	(2.257901, 2.252845),
	(1.765151, 1.747366),
	(1.954774, 1.933306),
	(1.798302, 1.721575),
	(1.824308, 1.824308),
	(1.456595, 1.504327),
	(1.384914, 1.489207),
	(2.407950, 2.087097),
	(1.739801, 1.744720),
	(2.137960, 2.137960),
	(1.851914, 1.848541),
	(2.003562, 1.948437),
	(1.712850, 1.702289),
	(2.390520, 2.431524),
	(2.046549, 2.012757),
	(1.092772, 1.092772),
	(1.524020, 1.537204),
	(2.638353, 2.669490),
	(2.071587, 2.038328),
	(1.853336, 1.916784),
	(1.667262, 1.656275),
	(1.404798, 1.431813),
	(1.940928, 2.022303),
	(2.347068, 2.304147),
	(1.869397, 1.877227),
	(2.086301, 2.086301),
	(1.718339, 1.705403),
	(1.562833, 1.584144),
	(2.051448, 2.043359),
	(1.926465, 1.920720),
	(2.147262, 2.088533),
	(1.578329, 1.580776),
	(1.272941, 1.296918),
	(1.832942, 1.036455),
	(1.543502, 1.517222),
	(1.723340, 1.839086),
	(2.034493, 2.069906),
	(2.005082, 2.016994),
	(2.228433, 2.193807),
	(1.861520, 1.840426),
	(2.277138, 2.141729),
	(1.592246, 1.692849),
	(2.548180, 2.549709),
	(1.726952, 1.808566),
	(2.425213, 2.265094),
	(1.969516, 1.876868),
	(2.085121, 2.077319),
	(1.868064, 1.830481),
	(1.794460, 1.802692),
	(1.644094, 1.634327),
	(1.497881, 1.646809),
	(2.048287, 1.965484),
	(3.046156, 2.463927),
	(1.487326, 1.505792),
	(1.482656, 1.538605),
	(1.839990, 1.846438),
	(2.124481, 2.107295),
	(2.172919, 2.192524),
	(1.832845, 1.853791),
	(1.657517, 1.727733),
	(1.720446, 1.684760),
	(1.947062, 1.557649),
	(1.979179, 2.050429),
	(1.952956, 1.879586),
	(1.502430, 1.593780),
	(1.376419, 1.350013),
	(1.714774, 1.690364),
	(1.285565, 1.270616),
	(1.660890, 1.647164),
	(1.944837, 1.942480),
	(2.713284, 2.697350),
	(1.628239, 1.620315),
	(1.476297, 1.568565),
	(2.168831, 2.145743),
	(1.543223, 1.580278),
	(1.923277, 1.889535),
	(1.539926, 1.533717),
	(1.987871, 2.018017),
	(2.184116, 2.088706),
	(2.722884, 2.236386),
	(1.630176, 1.643182),
	(1.907368, 1.899551),
	(1.352041, 1.353635),
	(2.950647, 2.860858),
	(1.763866, 1.817758),
	(1.054300, 1.054300),
	(1.812074, 1.851307),
	(1.314928, 1.306003),
	(1.628267, 1.655820),
	(2.938883, 2.948944),
	(1.930813, 2.031376),
	(2.031565, 2.067543),
	(1.600861, 1.574539),
	(2.402306, 2.380525),
	(1.796339, 1.784897),
	(1.528635, 1.541409),
	(1.499069, 1.560458),
	(1.700812, 1.714617),
	(1.629573, 1.602625),
	(2.080265, 2.051117),
	(1.642149, 1.638216),
	(2.212935, 2.174653),
	(1.987436, 1.952237),
	(1.895878, 1.880735),
	(1.948640, 1.793473),
	(1.866254, 1.856492),
	(2.330494, 2.355136),
	(2.369547, 2.386655),
	(1.892612, 1.915977),
	(1.553500, 1.636651),
	(1.662103, 1.716863),
	(1.644493, 1.648855),
	(1.740765, 1.847009),
	(3.142744, 3.156938),
	(1.507155, 1.495426),
	(2.278673, 2.312902),
	(1.869565, 1.869565),
	(1.549994, 1.551946),
	(1.730868, 1.684088),
	(2.056202, 2.044604),
	(2.433356, 2.435065),
	(1.905893, 1.755080),
	(1.869112, 2.051832),
	(1.355262, 1.367319),
	(1.585225, 1.571632),
	(2.057942, 2.047952),
	(1.974926, 1.993499),
	(1.470659, 1.494848),
	(2.102876, 2.095187),
	(1.722733, 1.719716),
	(2.014592, 2.009885),
	(1.997099, 1.972334),
	(2.261723, 2.358564),
	(1.407074, 1.341118),
	(2.401118, 2.415721),
	(1.474608, 1.460222),
	(1.840724, 1.878661),
	(1.469459, 1.478132),
	(1.254862, 1.281505),
	(1.927208, 1.898290),
	(2.250271, 2.226366),
	(1.777498, 1.784643),
	(1.747295, 1.764136),
	(1.606976, 1.618669),
	(2.144192, 2.132176),
	(2.435209, 2.392541),
	(1.444815, 1.427717),
	(1.136935, 1.153696),
	(2.462020, 2.352079),
	(2.140896, 2.130201),
	(1.610933, 1.767048),
	(1.376756, 1.376756),
	(2.241792, 2.251245),
	(2.171529, 2.447279),
	(1.717589, 1.703826),
	(3.057897, 2.988029),
	(1.583333, 1.680556),
	(2.101377, 2.053514),
	(1.946300, 1.944159),
	(1.680811, 1.685526),
	(1.917096, 1.918436),
	(1.504724, 1.504724),
	(1.755844, 1.688312),
	(1.487864, 1.532333),
	(1.901878, 1.887458),
	(1.710262, 1.653100),
	(3.258214, 3.155542),
	(2.059055, 2.013225),
	(2.540100, 2.537594),
	(3.402788, 3.314720),
	(1.537286, 1.555688),
	(1.668446, 1.766122),
	(1.876918, 1.894530),
	(1.639275, 1.620312),
	(2.279977, 2.276268),
	(1.658933, 1.740139),
	(2.025297, 2.095135),
	(1.630225, 1.819516),
	(1.794554, 1.617751),
	(1.957136, 1.933016),
	(1.547025, 1.543063),
	(2.655821, 2.616819),
	(1.922428, 1.893735),
	(1.565144, 1.651761),
	(1.831126, 1.772530),
	(1.785714, 1.785714),
	(2.318579, 2.292769),
	(1.972035, 1.950051),
	(1.932916, 1.919381),
	(2.309618, 2.271259),
	(1.813753, 1.845297),
	(1.338718, 1.334180),
	(1.651362, 1.643535),
	(1.903325, 1.791999),
	(1.581514, 1.581514),
	(1.572886, 1.556890),
	(2.026788, 1.950194),
	(1.756691, 1.771966),
	(2.473566, 2.242031),
	(2.129413, 2.112065),
	(1.775307, 1.770681),
	(2.423930, 2.397188),
	(1.961363, 1.985402),
	(1.931302, 1.925922),
	(2.124514, 2.131358),
	(2.173913, 1.739130),
	(2.466510, 2.427833),
	(1.756483, 1.938367),
	(2.027840, 2.016897),
	(1.961593, 1.961593),
	(1.520572, 1.580203),
	(1.721474, 2.309487),
	(2.208048, 2.183714),
	(3.302277, 2.812284),
	(2.013804, 2.001198),
	(2.791030, 2.841886),
	(2.080643, 2.096017),
	(1.498724, 1.495536),
	(1.296510, 1.297558),
	(2.492169, 2.291298),
	(1.804246, 1.799798),
	(1.418846, 1.429266),
	(1.532472, 1.556139),
	(2.108324, 2.088751),
	(1.650000, 1.673214),
	(1.587027, 1.596900),
	(1.729783, 1.707956),
	(1.682533, 1.727113),
	(1.616999, 1.623680),
	(2.113783, 2.024726),
	(1.720154, 1.701979),
	(1.527785, 1.540978),
	(2.105197, 2.097318),
	(2.023239, 2.069246),
	(2.190599, 2.197535),
	(2.248339, 2.242256),
	(1.754642, 1.668129),
	(2.150974, 1.992326),
	(2.321390, 2.290846),
	(2.114511, 2.109229),
	(1.809966, 1.792775),
	(3.084605, 2.640777),
	(2.082701, 2.113032),
	(2.172223, 1.900155),
	(2.065313, 2.063796),
	(1.827994, 1.804296),
	(1.875873, 1.888702),
	(2.225636, 2.230756),
	(1.785424, 1.783102),
	(2.260259, 2.315120),
	(1.725263, 1.701528),
	(2.113228, 2.018823),
	(2.076901, 2.054098),
	(1.930401, 1.680233),
	(1.910313, 1.923212),
	(2.058111, 2.074678),
	(1.666625, 1.667869),
	(1.592392, 1.625376),
	(1.707400, 1.645312),
	(2.129086, 2.154679),
	(1.685212, 1.679828),
	(2.533231, 2.538924),
	(2.157975, 2.099988),
	(1.266399, 1.308916),
	(1.886836, 1.906371),
	(1.785714, 1.789229),
	(1.885231, 2.007410),
	(1.650241, 1.749949),
	(2.124327, 2.170652),
	(2.580087, 2.630592),
	(1.832800, 1.798372),
	(1.835087, 1.822853),
	(2.175469, 2.248383),
	(2.033493, 2.021065),
	(1.739246, 1.776863),
	(1.567707, 1.549656),
	(1.433602, 1.445034),
	(1.982359, 1.976297),
	(1.991460, 2.067158),
	(2.215636, 2.391713),
	(1.766755, 1.745688),
	(1.323370, 1.352592),
	(2.239637, 2.225363),
	(2.458366, 2.444205),
	(1.817101, 1.801683),
	(1.578561, 1.578561),
	(1.850011, 1.848769),
	(2.411339, 2.362637),
	(1.958940, 1.973788),
	(1.434367, 1.522413),
	(2.087339, 2.051672),
	(2.241249, 2.317088),
	(1.383880, 1.697507),
	(2.059202, 2.072365),
	(2.406114, 2.371067),
	(1.807501, 1.726809),
	(1.832118, 1.808097),
	(2.374823, 2.267327),
	(1.799972, 1.797190),
	(1.622793, 1.622793),
	(2.260343, 2.276200),
	(2.030793, 2.161100),
	(1.539751, 1.538281),
	(1.976289, 1.975052),
	(1.540968, 1.536081),
	(1.615059, 1.636425),
	(1.208387, 1.208387),
	(4.247104, 3.639874),
	(1.555671, 1.550483),
	(0.473381, 0.474155),
	(1.910664, 1.849322),
	(1.655907, 1.664890),
	(1.554334, 1.549277),
	(1.416229, 1.416229),
	(0.754647, 0.755194),
	(1.351280, 1.308831),
	(2.812746, 2.688172),
	(3.489482, 2.284598),
	(1.694346, 1.687628),
	(1.521091, 1.579833),
	(2.148042, 2.152980),
	(1.510873, 1.510873),
	(1.916602, 1.932046),
	(1.808629, 1.808629),
	(1.260388, 1.255111),
	(1.711772, 1.711195),
	(1.807884, 1.861663),
	(1.660545, 1.707801),
	(1.363673, 1.363673),
	(1.343085, 1.364615),
	(2.027209, 2.027209),
	(1.410049, 1.413438),
	(1.363757, 1.372354),
	(1.060665, 1.062902),
	(2.160056, 2.156643),
	(1.662601, 1.665612),
	(2.122037, 2.077194),
	(3.075283, 2.825160),
	(1.049302, 1.307738),
	(1.291017, 1.287174),
	(2.759473, 2.727112),
	(2.246831, 2.262869),
	(1.971610, 1.974911),
	(2.455319, 2.132380),
	(1.168170, 1.160151),
	(2.171139, 2.149875),
	(2.811018, 2.811018),
	(2.383156, 2.128644),
	(1.871162, 1.798749),
	(1.905049, 1.914527),
	(2.498821, 2.613744),
	(1.887502, 1.703841),
	(1.852782, 1.861825),
	(3.117300, 2.857525),
	(1.520898, 1.527374),
	(2.594112, 2.187902),
	(1.575585, 1.587991),
	(2.812921, 2.820611),
	(1.979042, 1.945374),
	(1.939386, 1.963491),
	(1.528151, 1.524586),
	(1.847146, 1.835159),
	(1.917220, 1.927602),
	(1.922762, 1.968218),
	(1.363194, 1.544026),
	(1.806437, 1.890894),
	(1.988430, 2.002953),
	(1.808286, 1.797967),
	(1.862359, 1.880663),
	(1.688510, 1.771516),
	(1.845141, 1.845141),
	(2.312668, 2.335310),
	(1.575767, 1.572059),
	(2.042404, 2.016006),
	(2.185246, 2.129571),
	(1.615919, 1.608863),
	(2.154302, 2.163423),
	(1.797068, 1.791297),
	(2.056045, 1.985829),
	(1.655135, 1.777764),
	(1.162487, 1.225095),
	(1.945686, 1.940907),
	(2.553356, 2.655411),
	(2.270917, 2.258527),
	(1.810149, 1.902892),
	(1.661189, 1.658077),
	(1.904345, 2.161942),
	(1.068710, 1.076252),
	(1.314793, 1.325243),
	(1.829297, 1.880192),
	(1.174302, 1.182797),
	(1.314119, 1.305095),
	(1.409889, 1.462897),
	(1.788624, 1.776702),
	(1.305951, 1.316872),
	(1.160110, 1.176763),
	(1.582149, 1.602243),
	(2.277765, 1.524880),
	(2.591499, 2.376033),
	(1.665197, 1.791758),
	(3.035554, 2.679374),
	(1.237522, 1.229701),
	(1.356867, 1.359541),
	(2.885810, 2.755216),
	(1.522075, 1.524043),
	(2.855817, 2.839572),
	(2.385673, 2.304999),
	(1.261311, 1.301244),
	(1.058349, 1.058548),
	(1.580214, 1.580214),
	(1.388351, 1.390967),
	(3.690008, 3.232853),
	(1.188396, 1.192935),
	(1.444429, 1.444616),
	(1.049284, 1.051891),
	(2.114778, 2.210339),
	(1.608845, 1.590956),
	(0.280112, 0.280112),
	(1.362716, 1.358570),
	(1.581773, 1.574725),
	(2.129617, 2.078069),
	(1.847831, 2.107155),
	(1.817413, 1.863521),
	(2.228543, 2.238143),
	(2.293629, 2.393958),
	(1.915489, 1.867601),
	(2.011667, 1.992069),
	(1.807028, 1.802870),
	(2.156206, 2.106430),
	(1.813621, 1.812243),
	(2.466809, 2.371529),
	(2.185545, 2.179697),
	(1.950157, 1.967142),
	(1.882845, 1.876348),
	(1.663618, 1.711939),
	(2.041127, 2.263083),
	(2.098150, 2.060418),
	(1.992168, 2.071222),
	(1.432341, 1.547514),
	(2.017980, 2.044422),
	(2.197521, 1.969349),
	(2.264592, 2.252188),
	(2.308528, 2.256473),
	(2.551135, 2.475913),
	(1.509493, 1.503657),
	(1.745609, 1.730161),
	(2.383659, 2.635364),
	(2.048131, 2.124936),
	(1.956636, 1.969784),
	(1.938600, 1.924919),
	(1.219625, 1.229835),
	(3.540987, 2.561979),
	(2.185794, 2.122305),
	(2.539764, 2.346747),
	(2.825880, 2.793730),
	(1.967389, 1.965967),
	(2.241614, 2.011704),
	(1.294224, 1.315237),
	(2.563397, 2.527642),
	(1.476447, 1.495667),
	(1.702479, 1.674587),
	(1.343858, 1.342955),
	(1.417853, 1.439132),
	(1.532711, 1.523548),
	(1.619254, 1.653446),
	(0.916913, 0.916913),
	(1.614815, 1.666928),
	(1.211820, 1.206553),
	(1.399641, 1.390191),
	(2.341306, 2.193040),
	(1.538549, 1.536426),
	(1.447227, 1.424210),
	(2.023880, 2.009253),
	(2.442628, 2.385389),
	(1.042161, 1.041705),
	(1.629285, 1.494771),
	(2.300907, 2.298375),
	(3.094086, 2.979637),
	(1.214446, 1.214446),
	(2.095535, 2.148499),
	(1.690779, 1.682132),
	(1.788871, 1.779795),
	(3.123374, 2.799451),
	(0.939207, 0.937365),
	(2.187162, 2.093888),
	(2.582810, 2.183958),
	(1.427382, 1.367908),
	(1.570600, 1.588186),
	(1.941513, 1.957596),
	(1.870352, 1.862027),
	(2.821800, 2.789159),
	(2.389355, 2.313731),
	(1.203008, 0.902256),
	(1.724138, 1.736453),
	(1.278680, 1.364948),
	(1.492338, 1.482422),
	(1.556703, 1.562474),
	(1.514284, 1.510074),
	(2.000555, 1.971864),
	(2.276164, 2.183313),
	(2.956275, 2.615507),
	(1.410047, 1.421625),
	(2.446208, 2.442189),
	(1.699466, 1.717086),
	(1.785491, 1.783701),
	(1.346624, 1.342380),
	(1.227733, 1.226423),
	(1.504588, 1.505712),
	(3.935250, 3.935914),
	(2.330294, 1.886057),
	(2.017506, 2.020036),
	(2.735462, 2.605763),
	(2.635495, 2.514098),
	(1.971168, 1.990240),
	(1.768244, 1.783338),
	(1.281798, 1.241576),
	(1.444659, 1.455996),
	(2.295643, 2.266780),
	(0.872755, 0.873373),
	(1.226668, 1.230261),
	(2.475618, 2.464989),
	(2.165382, 2.128992),
	(2.907670, 2.636727),
	(2.402247, 2.335363),
	(1.186289, 1.186956),
	(1.429467, 1.437202),
	(1.476898, 1.510831),
	(1.315430, 1.311741),
	(1.708043, 1.709505),
	(1.294664, 1.299455),
	(0.984558, 0.991925),
	(3.489861, 3.202825),
	(2.057163, 1.950448),
	(2.387880, 2.390190),
	(2.286113, 2.255786),
	(1.500839, 1.119892),
	(2.722107, 2.691556),
	(1.627571, 1.614669),
	(1.366733, 1.365348),
	(1.627420, 1.610770),
	(1.642656, 1.669608),
	(1.869420, 1.702009),
	(1.478944, 1.476341),
	(2.657299, 2.564403),
	(1.392842, 1.378588),
	(1.128543, 1.121476),
	(1.494088, 1.489097),
	(4.698613, 4.682829),
	(1.557841, 1.566165),
	(1.555538, 1.555348),
	(1.909462, 1.908819),
	(2.512116, 2.374877),
	(1.170578, 1.192900),
	(1.496035, 1.494070),
	(1.187164, 1.185051),
	(1.353278, 1.362865),
	(2.466654, 2.466654),
	(1.255258, 1.238770),
	(2.009815, 2.003601),
	(1.378419, 1.375289),
	(0.782270, 0.766286),
	(1.044285, 1.060310),
	(1.346883, 1.348434),
	(1.402293, 1.420503),
	(0.773050, 0.779628),
	(1.215285, 1.212964),
	(1.276787, 1.272287),
	(0.830462, 0.830328),
	(1.302653, 1.300885),
	(1.143942, 1.131871),
	(1.854998, 1.782781),
	(3.337455, 3.161387),
	(1.094515, 1.097164),
	(1.455358, 1.465540),
	(1.382109, 1.375346),
	(1.277341, 1.277873),
	(1.717061, 1.645076),
	(0.979222, 0.979379),
	(1.154991, 1.158282),
	(0.977422, 0.982292),
	(1.220390, 1.241887),
	(2.663410, 2.644642),
	(3.771455, 2.789201),
	(0.924058, 0.935740),
	(1.266879, 1.263335),
	(1.229869, 1.226493),
	(1.034082, 1.109020),
	(0.835722, 0.839106),
	(3.088934, 3.042867),
	(1.803008, 1.810064),
	(1.382880, 1.389331),
	(2.791593, 2.756116),
	(1.327368, 1.329374),
	(1.116071, 1.105029),
	(0.980044, 0.978049),
	(0.786973, 0.788297),
	(1.078355, 1.203401),
	(1.265213, 1.265788),
	(1.850290, 1.744170),
	(0.861247, 0.895446),
	(1.043782, 1.052551),
	(2.332416, 2.278884),
	(2.324150, 2.287931),
	(1.529859, 1.582832),
	(2.281565, 2.200010),
	(2.331907, 2.261558),
	(1.660475, 1.656795),
	(4.446264, 4.430167),
	(1.433970, 1.423696),
	(1.308058, 1.308311),
	(0.874380, 0.878465),
	(1.325663, 1.355689),
	(1.874677, 1.854164),
	(1.263643, 1.241137),
	(1.122265, 1.131213),
	(1.717368, 1.720829),
	(1.595833, 1.758493),
	(2.610548, 2.380434),
	(3.050059, 3.078018),
	(1.550428, 1.548061),
	(1.123415, 1.117626),
	(2.771855, 2.523099),
} \fill \pos circle(0.03);

\draw (0,0) -- (5, 5);

\end{tikzpicture}
}
  \only<+>{\begin{tikzpicture}[scale=\chartscale]

\draw (0,0) -- (5,0);
\node at (2.5,-0.6) {LUnivRPI};
\node [anchor=north] at (0.714285714285714,0) {\tiny 0.1};
\draw (0.714285714285714,0) -- (0.714285714285714,0.1);
\draw [style=help lines] (0.714285714285714,0) -- (0.714285714285714,5);
\node [anchor=north] at (1.42857142857143,0) {\tiny 0.2};
\draw (1.42857142857143,0) -- (1.42857142857143,0.1);
\draw [style=help lines] (1.42857142857143,0) -- (1.42857142857143,5);
\node [anchor=north] at (2.14285714285714,0) {\tiny 0.3};
\draw (2.14285714285714,0) -- (2.14285714285714,0.1);
\draw [style=help lines] (2.14285714285714,0) -- (2.14285714285714,5);
\node [anchor=north] at (2.85714285714286,0) {\tiny 0.4};
\draw (2.85714285714286,0) -- (2.85714285714286,0.1);
\draw [style=help lines] (2.85714285714286,0) -- (2.85714285714286,5);
\node [anchor=north] at (3.57142857142857,0) {\tiny 0.5};
\draw (3.57142857142857,0) -- (3.57142857142857,0.1);
\draw [style=help lines] (3.57142857142857,0) -- (3.57142857142857,5);
\node [anchor=north] at (4.28571428571429,0) {\tiny 0.6};
\draw (4.28571428571429,0) -- (4.28571428571429,0.1);
\draw [style=help lines] (4.28571428571429,0) -- (4.28571428571429,5);
\node [anchor=north] at (5,0) {\tiny 0.7};
\draw (5,0) -- (5,0.1);
\draw [style=help lines] (5,0) -- (5,5);
\draw (0,0) -- (0,5);
\node [rotate=90] at (-2.5em,2.5) {RPI[3]LUniv};
\node [anchor=east] at (0,0.713285714285714) {\tiny 0.1};
\draw (0,0.713285714285714) -- (0.1,0.713285714285714);
\draw [style=help lines] (0,0.713285714285714) -- (5,0.713285714285714);
\node [anchor=east] at (0,1.42657142857143) {\tiny 0.2};
\draw (0,1.42657142857143) -- (0.1,1.42657142857143);
\draw [style=help lines] (0,1.42657142857143) -- (5,1.42657142857143);
\node [anchor=east] at (0,2.13985714285714) {\tiny 0.3};
\draw (0,2.13985714285714) -- (0.1,2.13985714285714);
\draw [style=help lines] (0,2.13985714285714) -- (5,2.13985714285714);
\node [anchor=east] at (0,2.85314285714286) {\tiny 0.4};
\draw (0,2.85314285714286) -- (0.1,2.85314285714286);
\draw [style=help lines] (0,2.85314285714286) -- (5,2.85314285714286);
\node [anchor=east] at (0,3.56642857142857) {\tiny 0.5};
\draw (0,3.56642857142857) -- (0.1,3.56642857142857);
\draw [style=help lines] (0,3.56642857142857) -- (5,3.56642857142857);
\node [anchor=east] at (0,4.27971428571429) {\tiny 0.6};
\draw (0,4.27971428571429) -- (0.1,4.27971428571429);
\draw [style=help lines] (0,4.27971428571429) -- (5,4.27971428571429);
\node [anchor=east] at (0,4.993) {\tiny 0.7};
\draw (0,4.993) -- (0.1,4.993);
\draw [style=help lines] (0,4.993) -- (5,4.993);
\foreach \pos in {
	(1.559731, 1.563275),
	(2.360406, 1.972444),
	(2.046691, 2.056181),
	(1.167770, 1.274815),
	(1.281639, 1.372801),
	(0.484526, 0.484526),
	(1.568389, 1.610942),
	(0.250054, 0.250054),
	(2.312593, 2.351863),
	(1.687598, 1.799731),
	(1.427291, 1.452893),
	(1.802097, 1.783374),
	(1.080477, 1.261443),
	(2.232143, 2.263655),
	(1.598280, 1.604737),
	(0.898964, 0.908736),
	(1.950998, 1.957480),
	(2.323961, 2.507599),
	(1.798202, 1.739170),
	(3.070307, 3.380703),
	(2.120202, 2.098829),
	(1.767990, 1.763559),
	(2.047748, 2.057453),
	(2.199977, 2.239566),
	(1.964637, 1.997380),
	(1.861827, 1.727166),
	(1.454545, 1.506494),
	(2.375626, 2.040055),
	(1.565217, 1.136646),
	(1.882204, 1.876917),
	(1.926469, 2.089269),
	(2.167488, 2.233169),
	(1.821531, 1.801064),
	(1.428571, 1.392405),
	(1.779762, 1.744048),
	(2.059437, 1.866528),
	(1.950138, 1.988376),
	(2.146109, 2.146109),
	(1.321892, 1.321892),
	(2.117621, 2.117621),
	(1.104901, 1.104901),
	(2.193929, 2.021292),
	(1.996086, 2.172211),
	(1.557465, 1.534448),
	(1.389150, 1.379763),
	(2.868217, 2.879291),
	(2.181025, 2.201472),
	(1.407490, 1.773313),
	(2.072704, 1.897321),
	(1.464543, 1.464543),
	(1.627940, 1.642283),
	(1.657887, 1.642392),
	(2.711896, 2.794631),
	(1.404151, 1.430316),
	(1.392857, 1.678571),
	(2.213658, 2.348091),
	(1.611722, 1.660562),
	(0.134983, 0.134983),
	(0.474354, 0.474354),
	(2.271517, 2.307010),
	(2.153625, 2.180546),
	(1.783343, 2.200721),
	(1.569990, 1.581975),
	(1.727484, 1.688665),
	(1.270053, 1.260504),
	(1.872120, 1.872120),
	(1.627950, 1.454180),
	(0.963880, 0.842127),
	(1.828231, 1.828231),
	(1.602632, 1.623859),
	(2.857143, 2.857143),
	(1.997579, 1.997579),
	(1.566620, 1.729487),
	(2.157327, 2.157327),
	(1.680000, 1.725714),
	(1.535714, 1.476190),
	(1.700680, 1.734694),
	(2.122088, 2.122088),
	(1.845523, 1.822739),
	(1.251476, 1.145218),
	(1.503759, 1.503759),
	(1.755562, 1.768963),
	(1.813616, 1.702009),
	(1.925841, 2.055188),
	(2.537594, 1.691729),
	(1.628264, 1.628264),
	(1.720210, 1.742994),
	(1.829268, 2.047038),
	(2.253401, 2.196712),
	(1.596841, 1.562500),
	(0.267681, 0.267681),
	(2.186998, 2.186998),
	(0.287698, 0.287698),
	(0.287698, 0.287698),
	(1.294778, 1.294778),
	(1.071429, 1.071429),
	(1.916980, 1.852586),
	(0.850181, 0.853528),
	(0.789474, 0.789474),
	(0.921659, 0.921659),
	(0.115830, 0.115830),
	(1.892857, 1.892857),
	(1.728111, 1.728111),
	(1.728111, 1.728111),
	(0.418443, 0.418443),
	(1.750237, 1.750237),
	(0.315770, 0.324863),
	(0.127767, 0.127767),
	(0.127767, 0.129112),
	(0.545889, 0.548326),
	(0.369922, 0.410277),
	(1.051051, 0.799013),
	(0.294922, 0.258956),
	(1.059771, 1.066836),
	(1.378245, 1.355370),
	(2.792823, 3.503724),
	(0.527024, 0.527024),
	(0.708075, 0.708075),
	(1.433271, 1.433271),
	(1.235231, 1.127820),
	(0.253593, 0.253593),
	(1.751152, 1.751152),
	(1.927438, 1.549509),
	(0.992063, 0.396825),
	(1.127820, 1.127820),
	(0.480769, 0.480769),
	(0.836551, 1.029601),
	(1.749271, 1.020408),
	(1.512605, 1.512605),
	(1.632653, 0.612245),
	(0.000000, 0.000000),
	(0.347222, 0.347222),
	(1.020408, 0.900360),
	(1.219512, 1.219512),
	(0.744048, 0.744048),
	(0.533049, 0.426439),
	(1.166181, 0.874636),
	(1.405152, 1.522248),
	(0.840336, 0.840336),
	(0.000000, 0.000000),
	(2.142857, 2.142857),
	(3.125000, 2.678571),
	(4.395604, 4.395604),
	(1.836897, 1.959167),
	(2.261947, 1.830979),
	(1.851852, 2.038017),
	(1.242989, 1.360709),
	(1.535905, 1.728143),
	(1.644669, 1.730151),
	(2.140851, 2.133557),
	(1.380820, 1.412654),
	(1.578703, 1.547748),
	(1.924430, 1.962323),
	(1.787746, 1.917764),
	(1.765904, 1.720625),
	(1.546519, 1.554767),
	(1.573535, 1.449634),
	(1.236199, 1.317483),
	(2.747253, 2.940721),
	(1.447544, 1.567002),
	(1.969445, 2.067193),
	(1.884921, 1.879085),
	(1.733136, 1.760399),
	(1.542675, 1.542675),
	(1.636106, 1.765816),
	(2.047464, 2.066853),
	(1.144031, 1.144031),
	(1.837379, 1.837379),
	(2.189246, 2.221800),
	(2.601132, 2.687290),
	(1.667896, 1.671994),
	(2.142254, 2.142254),
	(1.547956, 1.547956),
	(2.206890, 2.202164),
	(2.109326, 2.109326),
	(1.547762, 1.616361),
	(2.425505, 2.453023),
	(1.620961, 1.623618),
	(2.446886, 2.446886),
	(1.591711, 1.591711),
	(2.261660, 2.334996),
	(2.047508, 1.960522),
	(2.795241, 3.124093),
	(1.733012, 1.730274),
	(2.090695, 2.094902),
	(1.784829, 1.809618),
	(1.753752, 1.826014),
	(2.185472, 2.259754),
	(1.881302, 2.131691),
	(2.821204, 3.081761),
	(1.868721, 1.915546),
	(1.875477, 1.959511),
	(1.779997, 1.852417),
	(1.990434, 1.913136),
	(1.472413, 1.410373),
	(1.980792, 2.020808),
	(1.601960, 1.595678),
	(2.829918, 2.865740),
	(1.254181, 1.206402),
	(2.532891, 1.429607),
	(1.951425, 1.966264),
	(2.521008, 2.534635),
	(1.567176, 1.547488),
	(1.831388, 1.924962),
	(1.981474, 2.076967),
	(1.936119, 1.944323),
	(1.402088, 1.409877),
	(2.120105, 1.923607),
	(2.128773, 2.124748),
	(2.289738, 2.080483),
	(1.597386, 1.597386),
	(1.455745, 1.455745),
	(1.646825, 1.582341),
	(2.464418, 2.573142),
	(2.025629, 2.011263),
	(1.923284, 2.023090),
	(0.113773, 0.114527),
	(1.627424, 1.627424),
	(1.095530, 1.193285),
	(1.455203, 1.493247),
	(1.915403, 1.915403),
	(1.408738, 1.504987),
	(1.814901, 1.814901),
	(1.520147, 1.570513),
	(2.225511, 2.303697),
	(1.729043, 1.742377),
	(1.879391, 1.879391),
	(1.141527, 1.163693),
	(2.572608, 1.862499),
	(1.519991, 1.497962),
	(1.280512, 1.280512),
	(2.194211, 2.212885),
	(1.702154, 1.931171),
	(2.214369, 2.374510),
	(1.830956, 1.830956),
	(1.624462, 1.684185),
	(2.210814, 2.213781),
	(1.826299, 1.652365),
	(1.651376, 1.651376),
	(1.804021, 1.784050),
	(2.228958, 2.022424),
	(2.006173, 2.015621),
	(1.595509, 1.602896),
	(2.411874, 2.353896),
	(1.923310, 1.923310),
	(1.515322, 1.453357),
	(1.666890, 1.666890),
	(1.498501, 1.479770),
	(1.588160, 1.526384),
	(1.548225, 1.564395),
	(1.648352, 1.737936),
	(1.955850, 1.778566),
	(2.092389, 2.298681),
	(1.593683, 1.586504),
	(1.594286, 1.634286),
	(1.604295, 1.539212),
	(1.723366, 1.736156),
	(2.523762, 2.631768),
	(1.414901, 1.209843),
	(1.774644, 1.808706),
	(1.256026, 1.256026),
	(2.135427, 2.196808),
	(2.002165, 2.002165),
	(1.182831, 1.228702),
	(1.684695, 1.771727),
	(1.088227, 1.088227),
	(1.515152, 1.515152),
	(1.675618, 1.672037),
	(2.366468, 2.068989),
	(1.599081, 1.550490),
	(2.192439, 1.650139),
	(1.934813, 1.879334),
	(1.351570, 1.321954),
	(2.369959, 2.275726),
	(3.129637, 3.031836),
	(2.051546, 1.593011),
	(3.339212, 3.380364),
	(2.622806, 2.657123),
	(2.716776, 2.713003),
	(1.867401, 1.867401),
	(2.310783, 2.080990),
	(1.597294, 1.588729),
	(1.209476, 1.375134),
	(2.208546, 2.267470),
	(1.870997, 1.839605),
	(2.168135, 2.009173),
	(2.048902, 2.130971),
	(2.182314, 2.380411),
	(1.721723, 1.745871),
	(1.442469, 1.468826),
	(1.819446, 1.982991),
	(1.526544, 1.822739),
	(2.119993, 2.215109),
	(1.796771, 1.835471),
	(1.515072, 1.646134),
	(1.824152, 1.820356),
	(2.457075, 2.433815),
	(1.473923, 1.481689),
	(1.490666, 1.468774),
	(2.516020, 2.212590),
	(1.935694, 1.950470),
	(1.711586, 1.713996),
	(1.711953, 1.859476),
	(1.844660, 1.905687),
	(2.093168, 2.152174),
	(1.719786, 1.700949),
	(1.839373, 1.844365),
	(2.267827, 2.267827),
	(2.198605, 2.219485),
	(1.869330, 1.896820),
	(2.373973, 2.373973),
	(2.976190, 3.015873),
	(1.934860, 1.959045),
	(2.606332, 2.590342),
	(1.679623, 1.683190),
	(2.174532, 2.203004),
	(2.192862, 2.188274),
	(1.993895, 1.978438),
	(1.666381, 1.649231),
	(1.202490, 1.464613),
	(1.712625, 1.714430),
	(1.247036, 1.247036),
	(1.746839, 1.761372),
	(3.119781, 3.104600),
	(1.874063, 1.922253),
	(1.822371, 1.750554),
	(1.423978, 1.425745),
	(2.109361, 2.168206),
	(2.121084, 2.291989),
	(2.866488, 2.907118),
	(1.691012, 1.709194),
	(1.496400, 1.496400),
	(2.099575, 2.053874),
	(1.857622, 1.909556),
	(2.071713, 2.074559),
	(1.645222, 1.657936),
	(1.630581, 1.697738),
	(1.710331, 1.814137),
	(1.827708, 1.667733),
	(1.689339, 1.677061),
	(1.855848, 1.877428),
	(1.952705, 1.921994),
	(2.534063, 2.595995),
	(1.854178, 1.891865),
	(1.892525, 1.908349),
	(2.384161, 2.448338),
	(1.974892, 2.101424),
	(2.581179, 2.498477),
	(1.564990, 1.549053),
	(1.248413, 1.157729),
	(2.224019, 2.215354),
	(1.523571, 1.541495),
	(1.254010, 1.249844),
	(2.270059, 2.253285),
	(2.328344, 2.328344),
	(1.470293, 1.478317),
	(1.498304, 1.517150),
	(1.796925, 1.854580),
	(1.583073, 1.685821),
	(1.634781, 1.736710),
	(2.486643, 2.089679),
	(1.485414, 1.608752),
	(1.532718, 1.529443),
	(1.710501, 1.638600),
	(2.210366, 2.235772),
	(1.800174, 1.782099),
	(1.358885, 1.379791),
	(1.536098, 1.282051),
	(1.946050, 1.970823),
	(1.644010, 1.625194),
	(1.557589, 1.557589),
	(2.161062, 2.192381),
	(2.292390, 2.288984),
	(1.881500, 1.902025),
	(1.868635, 1.910679),
	(1.412769, 1.419090),
	(1.361139, 1.431901),
	(1.797161, 1.828640),
	(1.848412, 1.913431),
	(1.059639, 1.145631),
	(1.675424, 1.591157),
	(1.559604, 1.569191),
	(1.698205, 1.701670),
	(1.831502, 1.868810),
	(1.904037, 1.525949),
	(1.520485, 1.525374),
	(1.625888, 1.965272),
	(1.670822, 1.668961),
	(1.506810, 1.477833),
	(1.601620, 1.620029),
	(1.825225, 1.815928),
	(2.374586, 1.836644),
	(1.565008, 1.622971),
	(1.924886, 1.893524),
	(1.859197, 1.944189),
	(2.358907, 2.346046),
	(2.285347, 2.254091),
	(3.022693, 2.915737),
	(2.721829, 2.519636),
	(1.823332, 1.897681),
	(2.664652, 2.672606),
	(1.530612, 1.530612),
	(2.680488, 2.698890),
	(1.259716, 1.701957),
	(2.796733, 2.662490),
	(1.806438, 1.934236),
	(1.930132, 1.785030),
	(1.582080, 1.632989),
	(2.231571, 2.423633),
	(1.700168, 1.719875),
	(1.869116, 1.908723),
	(1.818611, 1.977506),
	(1.111295, 1.165000),
	(2.190894, 2.193745),
	(2.330937, 2.367576),
	(1.851146, 1.880612),
	(1.780444, 1.710718),
	(1.482509, 1.456607),
	(2.142857, 2.058300),
	(1.737503, 1.748847),
	(1.891234, 1.956169),
	(1.954532, 2.046444),
	(1.846326, 1.676613),
	(1.674391, 1.674391),
	(1.866180, 1.874108),
	(1.821692, 1.804142),
	(2.111287, 2.223757),
	(1.953000, 1.985016),
	(1.728984, 1.815360),
	(2.320062, 2.341920),
	(1.574601, 1.637585),
	(1.724508, 1.745952),
	(3.210273, 3.534948),
	(2.359149, 2.128031),
	(1.805679, 1.870009),
	(1.696880, 1.696880),
	(1.481070, 1.492566),
	(1.254440, 1.552588),
	(1.281723, 1.286840),
	(1.942054, 1.893287),
	(1.853493, 2.025547),
	(1.465778, 1.606836),
	(2.594445, 2.604611),
	(1.898485, 1.966366),
	(2.194976, 1.823024),
	(2.126266, 2.204855),
	(2.564987, 2.461468),
	(2.256264, 2.609548),
	(2.257901, 2.207332),
	(1.765151, 1.754035),
	(1.954774, 1.941655),
	(1.798302, 1.798302),
	(1.824308, 1.839010),
	(1.456595, 1.508946),
	(1.384914, 1.542566),
	(2.407950, 2.407950),
	(1.739801, 1.751279),
	(2.137960, 2.168098),
	(1.851914, 1.868781),
	(2.003562, 2.122291),
	(1.712850, 1.736082),
	(2.390520, 2.450659),
	(2.046549, 2.040213),
	(1.092772, 1.047240),
	(1.524020, 1.521384),
	(2.638353, 2.673641),
	(2.071587, 2.049414),
	(1.853336, 1.930141),
	(1.667262, 1.617821),
	(1.404798, 1.437216),
	(1.940928, 2.049427),
	(2.347068, 2.234119),
	(1.869397, 1.890929),
	(2.086301, 1.998324),
	(1.718339, 1.726963),
	(1.562833, 1.616111),
	(2.051448, 2.075716),
	(1.926465, 1.953275),
	(2.147262, 2.158273),
	(1.578329, 1.600352),
	(1.272941, 1.296918),
	(1.832942, 1.036455),
	(1.543502, 1.561022),
	(1.723340, 1.854519),
	(2.034493, 2.069906),
	(2.005082, 1.923688),
	(2.228433, 2.233380),
	(1.861520, 1.864157),
	(2.277138, 2.139472),
	(1.592246, 1.677372),
	(2.548180, 2.551239),
	(1.726952, 1.798773),
	(2.425213, 2.425213),
	(1.969516, 1.993425),
	(2.085121, 2.081220),
	(1.868064, 1.865853),
	(1.794460, 1.792402),
	(1.644094, 1.758041),
	(1.497881, 1.560889),
	(2.048287, 1.625556),
	(3.046156, 3.041937),
	(1.487326, 1.515864),
	(1.482656, 1.552592),
	(1.839990, 1.762607),
	(2.124481, 2.123159),
	(2.172919, 2.172919),
	(1.832845, 1.866883),
	(1.657517, 1.736244),
	(1.720446, 1.731716),
	(1.947062, 1.797288),
	(1.979179, 2.167201),
	(1.952956, 1.952956),
	(1.502430, 1.607386),
	(1.376419, 1.379720),
	(1.714774, 1.739183),
	(1.285565, 1.288056),
	(1.660890, 1.666038),
	(1.944837, 1.944837),
	(2.713284, 2.453792),
	(1.628239, 1.628239),
	(1.476297, 1.594018),
	(2.168831, 2.181818),
	(1.543223, 1.575919),
	(1.923277, 1.777928),
	(1.539926, 1.570973),
	(1.987871, 2.044616),
	(2.184116, 2.256318),
	(2.722884, 2.677629),
	(1.630176, 1.645350),
	(1.907368, 1.901505),
	(1.352041, 1.367985),
	(2.950647, 3.015667),
	(1.763866, 1.843976),
	(1.054300, 0.998425),
	(1.812074, 1.870923),
	(1.314928, 1.258404),
	(1.628267, 1.654508),
	(2.938883, 2.951459),
	(1.930813, 2.042869),
	(2.031565, 2.075938),
	(1.600861, 1.617612),
	(2.402306, 2.401025),
	(1.796339, 1.794704),
	(1.528635, 1.537151),
	(1.499069, 1.568379),
	(1.700812, 1.720139),
	(1.629573, 1.631158),
	(2.080265, 2.074129),
	(1.642149, 1.648049),
	(2.212935, 2.195921),
	(1.987436, 2.076790),
	(1.895878, 1.903450),
	(1.948640, 1.803817),
	(1.866254, 1.867882),
	(2.330494, 2.427304),
	(2.369547, 2.312518),
	(1.892612, 1.983738),
	(1.553500, 1.640707),
	(1.662103, 1.761958),
	(1.644493, 1.705562),
	(1.740765, 1.871527),
	(3.142744, 3.162616),
	(1.507155, 1.516929),
	(2.278673, 2.316569),
	(1.869565, 1.868323),
	(1.549994, 1.596845),
	(1.730868, 1.732482),
	(2.056202, 2.059516),
	(2.433356, 2.465824),
	(1.905893, 1.851223),
	(1.869112, 2.078654),
	(1.355262, 1.381788),
	(1.585225, 1.575031),
	(2.057942, 2.057942),
	(1.974926, 1.995047),
	(1.470659, 1.502104),
	(2.102876, 2.114409),
	(1.722733, 1.728768),
	(2.014592, 2.028713),
	(1.997099, 1.991792),
	(2.261723, 2.399175),
	(1.407074, 1.436389),
	(2.401118, 2.315587),
	(1.474608, 1.471011),
	(1.840724, 1.898388),
	(1.469459, 1.473176),
	(1.254862, 1.297490),
	(1.927208, 1.842518),
	(2.250271, 2.261427),
	(1.777498, 1.791787),
	(1.747295, 1.757821),
	(1.606976, 1.628692),
	(2.144192, 2.190921),
	(2.435209, 2.398862),
	(1.444815, 1.417030),
	(1.136935, 1.184424),
	(2.462020, 2.370402),
	(2.140896, 1.750508),
	(1.610933, 1.778361),
	(1.376756, 1.381038),
	(2.241792, 2.287479),
	(2.171529, 2.481747),
	(1.717589, 1.731352),
	(3.057897, 3.057897),
	(1.583333, 1.686508),
	(2.101377, 2.117331),
	(1.946300, 1.946300),
	(1.680811, 1.706742),
	(1.917096, 1.929161),
	(1.504724, 1.506760),
	(1.755844, 1.779221),
	(1.487864, 1.532333),
	(1.901878, 1.776902),
	(1.710262, 1.472471),
	(3.258214, 3.155542),
	(2.059055, 2.055781),
	(2.540100, 2.532581),
	(3.402788, 3.412157),
	(1.537286, 1.572674),
	(1.668446, 1.787702),
	(1.876918, 1.859307),
	(1.639275, 1.710914),
	(2.279977, 2.294814),
	(1.658933, 1.784886),
	(2.025297, 1.837123),
	(1.630225, 1.912231),
	(1.794554, 1.896216),
	(1.957136, 1.958859),
	(1.547025, 1.564852),
	(2.655821, 2.657678),
	(1.922428, 1.886984),
	(1.565144, 1.653775),
	(1.831126, 1.828196),
	(1.785714, 1.785714),
	(2.318579, 2.335785),
	(1.972035, 1.972035),
	(1.932916, 1.909906),
	(2.309618, 2.309618),
	(1.813753, 1.852306),
	(1.338718, 1.325104),
	(1.651362, 1.669623),
	(1.903325, 1.809955),
	(1.581514, 1.583597),
	(1.572886, 1.585327),
	(2.026788, 1.985545),
	(1.756691, 1.761273),
	(2.473566, 2.546886),
	(2.129413, 1.721745),
	(1.775307, 1.777157),
	(2.423930, 2.413424),
	(1.961363, 2.000699),
	(1.931302, 1.942061),
	(2.124514, 2.138203),
	(2.173913, 2.178350),
	(2.466510, 2.478816),
	(1.756483, 2.005924),
	(2.027840, 2.053011),
	(1.961593, 1.962923),
	(1.520572, 1.618537),
	(1.721474, 2.311063),
	(2.208048, 2.213171),
	(3.302277, 3.336784),
	(2.013804, 2.015379),
	(2.791030, 2.889110),
	(2.080643, 2.127790),
	(1.498724, 1.503508),
	(1.296510, 1.301750),
	(2.492169, 2.485360),
	(1.804246, 1.820554),
	(1.418846, 1.477892),
	(1.532472, 1.573890),
	(2.108324, 2.126499),
	(1.650000, 1.698214),
	(1.587027, 1.761033),
	(1.729783, 1.753792),
	(1.682533, 1.728551),
	(1.616999, 1.628135),
	(2.113783, 2.077112),
	(1.720154, 1.718855),
	(1.527785, 1.620138),
	(2.105197, 2.136712),
	(2.023239, 2.060686),
	(2.190599, 2.205627),
	(2.248339, 2.260506),
	(1.754642, 1.704684),
	(2.150974, 2.040289),
	(2.321390, 2.344299),
	(2.114511, 2.126835),
	(1.809966, 1.819789),
	(3.084605, 3.159501),
	(2.082701, 2.138308),
	(2.172223, 2.375194),
	(2.065313, 2.085041),
	(1.827994, 1.837689),
	(1.875873, 1.910084),
	(2.225636, 2.293907),
	(1.785424, 1.799355),
	(2.260259, 2.318777),
	(1.725263, 1.734164),
	(2.113228, 2.028990),
	(2.076901, 2.062868),
	(1.930401, 1.967740),
	(1.910313, 1.933766),
	(2.058111, 2.078501),
	(1.666625, 1.671600),
	(1.592392, 1.629041),
	(1.707400, 1.700344),
	(2.129086, 2.098618),
	(1.685212, 1.690596),
	(2.533231, 2.544616),
	(2.157975, 2.066852),
	(1.266399, 1.311953),
	(1.886836, 1.923607),
	(1.785714, 1.840200),
	(1.885231, 2.020547),
	(1.650241, 1.778850),
	(2.124327, 2.135357),
	(2.580087, 2.646465),
	(1.832800, 1.966463),
	(1.835087, 1.845573),
	(2.175469, 2.269594),
	(2.033493, 2.066116),
	(1.739246, 1.787927),
	(1.567707, 1.564930),
	(1.433602, 1.445034),
	(1.982359, 2.002061),
	(1.991460, 2.104037),
	(2.215636, 2.391713),
	(1.766755, 1.752711),
	(1.323370, 1.369291),
	(2.239637, 2.292452),
	(2.458366, 2.482440),
	(1.817101, 1.865557),
	(1.578561, 1.579610),
	(1.850011, 1.850011),
	(2.411339, 2.376374),
	(1.958940, 1.974778),
	(1.434367, 1.524642),
	(2.087339, 2.078035),
	(2.241249, 2.293139),
	(1.383880, 1.709268),
	(2.059202, 2.081140),
	(2.406114, 2.381851),
	(1.807501, 1.767155),
	(1.832118, 1.884526),
	(2.374823, 2.578501),
	(1.799972, 1.808318),
	(1.622793, 1.622793),
	(2.260343, 2.136370),
	(2.030793, 2.219237),
	(1.539751, 1.548575),
	(1.976289, 1.987427),
	(1.540968, 1.545854),
	(1.615059, 1.651765),
	(1.208387, 1.208387),
	(4.247104, 4.099684),
	(1.555671, 1.555022),
	(0.473381, 0.474155),
	(1.910664, 1.894631),
	(1.655907, 1.666687),
	(1.554334, 1.552086),
	(1.416229, 1.416229),
	(0.754647, 0.754647),
	(1.351280, 1.349511),
	(2.812746, 2.898917),
	(3.489482, 2.289100),
	(1.694346, 1.700392),
	(1.521091, 1.584352),
	(2.148042, 2.264086),
	(1.510873, 1.510873),
	(1.916602, 1.927413),
	(1.808629, 1.821809),
	(1.260388, 1.257750),
	(1.711772, 1.610876),
	(1.807884, 1.865667),
	(1.660545, 1.711391),
	(1.363673, 1.361682),
	(1.343085, 1.364615),
	(2.027209, 2.027209),
	(1.410049, 1.414567),
	(1.363757, 1.379630),
	(1.060665, 1.062902),
	(2.160056, 2.177118),
	(1.662601, 1.667871),
	(2.122037, 2.155669),
	(3.075283, 3.034279),
	(1.049302, 1.328482),
	(1.291017, 1.301263),
	(2.759473, 2.808014),
	(2.246831, 2.262869),
	(1.971610, 1.967484),
	(2.455319, 2.137310),
	(1.168170, 1.170020),
	(2.171139, 2.111598),
	(2.811018, 1.944973),
	(2.383156, 2.436042),
	(1.871162, 1.798749),
	(1.905049, 1.919697),
	(2.498821, 2.737506),
	(1.887502, 2.095504),
	(1.852782, 1.856801),
	(3.117300, 3.096289),
	(1.520898, 1.527374),
	(2.594112, 2.594112),
	(1.575585, 1.589542),
	(2.812921, 2.957124),
	(1.979042, 1.982198),
	(1.939386, 1.993075),
	(1.528151, 1.537657),
	(1.847146, 1.857934),
	(1.917220, 1.943176),
	(1.922762, 1.971854),
	(1.363194, 1.553299),
	(1.806437, 1.904970),
	(1.988430, 2.005373),
	(1.808286, 1.809145),
	(1.862359, 1.896678),
	(1.688510, 1.771516),
	(1.845141, 1.848063),
	(2.312668, 2.291105),
	(1.575767, 1.579474),
	(2.042404, 2.018785),
	(2.185246, 2.146020),
	(1.615919, 1.617683),
	(2.154302, 2.169308),
	(1.797068, 1.792050),
	(2.056045, 2.052215),
	(1.655135, 1.787883),
	(1.162487, 1.225095),
	(1.945686, 1.943828),
	(2.553356, 2.667301),
	(2.270917, 2.274457),
	(1.810149, 1.905669),
	(1.661189, 1.661754),
	(1.904345, 2.198444),
	(1.068710, 1.077195),
	(1.314793, 1.334061),
	(1.829297, 1.883409),
	(1.174302, 1.185371),
	(1.314119, 1.300427),
	(1.409889, 1.465160),
	(1.788624, 1.780108),
	(1.305951, 1.317126),
	(1.160110, 1.178395),
	(1.582149, 1.603789),
	(2.277765, 1.517236),
	(2.591499, 2.591499),
	(1.665197, 1.790336),
	(3.035554, 3.027532),
	(1.237522, 1.229701),
	(1.356867, 1.360544),
	(2.885810, 2.904390),
	(1.522075, 1.526011),
	(2.855817, 2.901237),
	(2.385673, 2.385673),
	(1.261311, 1.302475),
	(1.058349, 1.059743),
	(1.580214, 1.469489),
	(1.388351, 1.393758),
	(3.690008, 3.690008),
	(1.188396, 1.192935),
	(1.444429, 1.444992),
	(1.049284, 1.051891),
	(2.114778, 2.055216),
	(1.608845, 1.600189),
	(0.280112, 0.280112),
	(1.362716, 1.358751),
	(1.581773, 1.581981),
	(2.129617, 2.096262),
	(1.847831, 2.138515),
	(1.817413, 1.900663),
	(2.228543, 2.262145),
	(2.293629, 2.412073),
	(1.915489, 1.873348),
	(2.011667, 2.007055),
	(1.807028, 1.795592),
	(2.156206, 2.178832),
	(1.813621, 1.848075),
	(2.466809, 2.378858),
	(2.185545, 2.211859),
	(1.950157, 2.002656),
	(1.882845, 1.882845),
	(1.663618, 1.739551),
	(2.041127, 2.266348),
	(2.098150, 2.078712),
	(1.992168, 2.095546),
	(1.432341, 1.557985),
	(2.017980, 2.064854),
	(2.197521, 1.996193),
	(2.264592, 2.300032),
	(2.308528, 1.940748),
	(2.551135, 2.491918),
	(1.509493, 1.517274),
	(1.745609, 1.736103),
	(2.383659, 2.674802),
	(2.048131, 2.142345),
	(1.956636, 1.990103),
	(1.938600, 1.942704),
	(1.219625, 1.230164),
	(3.540987, 3.389996),
	(2.185794, 2.185140),
	(2.539764, 2.517164),
	(2.825880, 2.871333),
	(1.967389, 1.965967),
	(2.241614, 2.220713),
	(1.294224, 1.315237),
	(2.563397, 2.689917),
	(1.476447, 1.497763),
	(1.702479, 1.701067),
	(1.343858, 1.343858),
	(1.417853, 1.439132),
	(1.532711, 1.533659),
	(1.619254, 1.654166),
	(0.916913, 0.916913),
	(1.614815, 1.669653),
	(1.211820, 1.211820),
	(1.399641, 1.395280),
	(2.341306, 2.354340),
	(1.538549, 1.536072),
	(1.447227, 1.470648),
	(2.023880, 2.012718),
	(2.442628, 2.455940),
	(1.042161, 1.044896),
	(1.629285, 1.508373),
	(2.300907, 2.298375),
	(3.094086, 3.009666),
	(1.214446, 1.214446),
	(2.095535, 2.151569),
	(1.690779, 1.690779),
	(1.788871, 1.822810),
	(3.123374, 3.123374),
	(0.939207, 0.937365),
	(2.187162, 2.220020),
	(2.582810, 2.616859),
	(1.427382, 1.350065),
	(1.570600, 1.591954),
	(1.941513, 1.962957),
	(1.870352, 1.818023),
	(2.821800, 2.830805),
	(2.389355, 2.385754),
	(1.203008, 1.203008),
	(1.724138, 1.736453),
	(1.278680, 1.361630),
	(1.492338, 1.528188),
	(1.556703, 1.563299),
	(1.514284, 1.526914),
	(2.000555, 1.976491),
	(2.276164, 2.155458),
	(2.956275, 2.980616),
	(1.410047, 1.422234),
	(2.446208, 2.454245),
	(1.699466, 1.717550),
	(1.785491, 1.789517),
	(1.346624, 1.343794),
	(1.227733, 1.228606),
	(1.504588, 1.505712),
	(3.935250, 3.928605),
	(2.330294, 2.330294),
	(2.017506, 2.023409),
	(2.735462, 2.808565),
	(2.635495, 2.647243),
	(1.971168, 1.990240),
	(1.768244, 1.791165),
	(1.281798, 1.240058),
	(1.444659, 1.448438),
	(2.295643, 2.301416),
	(0.872755, 0.873373),
	(1.226668, 1.230860),
	(2.475618, 2.521108),
	(2.165382, 2.150642),
	(2.907670, 2.938530),
	(2.402247, 2.420254),
	(1.186289, 1.188512),
	(1.429467, 1.437405),
	(1.476898, 1.512197),
	(1.315430, 1.315213),
	(1.708043, 1.713891),
	(1.294664, 1.298349),
	(0.984558, 0.992400),
	(3.489861, 3.493991),
	(2.057163, 2.085771),
	(2.387880, 2.390190),
	(2.286113, 2.271748),
	(1.500839, 1.543398),
	(2.722107, 2.810705),
	(1.627571, 1.619751),
	(1.366733, 1.364557),
	(1.627420, 1.629085),
	(1.642656, 1.678403),
	(1.869420, 1.869420),
	(1.478944, 1.477543),
	(2.657299, 2.778298),
	(1.392842, 1.381201),
	(1.128543, 1.122654),
	(1.494088, 1.503162),
	(4.698613, 4.694899),
	(1.557841, 1.566410),
	(1.555538, 1.555917),
	(1.909462, 1.911819),
	(2.512116, 2.447785),
	(1.170578, 1.192900),
	(1.496035, 1.499122),
	(1.187164, 1.185262),
	(1.353278, 1.366260),
	(2.466654, 2.466654),
	(1.255258, 1.242535),
	(2.009815, 2.005991),
	(1.378419, 1.377215),
	(0.782270, 0.766286),
	(1.044285, 1.060872),
	(1.346883, 1.351297),
	(1.402293, 1.423865),
	(0.773050, 0.779226),
	(1.215285, 1.214056),
	(1.276787, 1.275115),
	(0.830462, 0.830328),
	(1.302653, 1.300885),
	(1.143942, 1.130915),
	(1.854998, 1.850866),
	(3.337455, 3.312414),
	(1.094515, 1.097740),
	(1.455358, 1.466400),
	(1.382109, 1.378101),
	(1.277341, 1.283464),
	(1.717061, 1.648510),
	(0.979222, 0.978596),
	(1.154991, 1.158282),
	(0.977422, 0.982292),
	(1.220390, 1.244477),
	(2.663410, 2.683579),
	(3.771455, 3.772573),
	(0.924058, 0.940324),
	(1.266879, 1.264970),
	(1.229869, 1.230544),
	(1.034082, 1.109020),
	(0.835722, 0.839260),
	(3.088934, 3.105906),
	(1.803008, 1.459203),
	(1.382880, 1.390831),
	(2.791593, 2.815921),
	(1.327368, 1.330454),
	(1.116071, 1.109919),
	(0.980044, 0.978411),
	(0.786973, 0.788297),
	(1.078355, 1.206626),
	(1.265213, 1.267082),
	(1.850290, 1.838240),
	(0.861247, 0.895446),
	(1.043782, 1.061090),
	(2.332416, 2.328134),
	(2.324150, 2.290331),
	(1.529859, 1.585250),
	(2.281565, 2.278303),
	(2.331907, 2.340824),
	(1.660475, 1.656795),
	(4.446264, 4.437866),
	(1.433970, 1.426114),
	(1.308058, 1.308311),
	(0.874380, 0.878720),
	(1.325663, 1.353961),
	(1.874677, 1.871555),
	(1.263643, 1.245436),
	(1.122265, 1.131213),
	(1.717368, 1.723366),
	(1.595833, 1.763576),
	(2.610548, 2.607438),
	(3.050059, 3.126946),
	(1.550428, 1.551375),
	(1.123415, 1.117626),
	(2.771855, 2.487562),
} \fill \pos circle(0.03);
\draw (0,0) -- (5, 5);
\end{tikzpicture}
}
  
\end{columns}
\end{frame}

\begin{frame}{References}

\begin{subpart}{Skeptik}
\item http://github.com/Paradoxika/Skeptik
\end{subpart}

\begin{subpart}{Bibliography}
\item Fontaine, P., Merz, S., Woltzenlogel Paleo, B.: Compression of
  propositional resolution proofs via partial regularization. In: CADE. Lecture
  Notes in Computer Science, vol. 6803, pp. 237--251. Springer (2011)
\end{subpart}

\end{frame}

\end{document}
% vim: tw=100
